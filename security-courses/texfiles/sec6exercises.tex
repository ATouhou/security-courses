% Henrik Lund, spring/summer 1998
% My own packages and commands
% included by 
% command: \input{stdpack.tex}

%\usepackage{fancyheadings}
\usepackage{fancyhdr}
\usepackage{isolatin1}
\usepackage{times}
\usepackage[rflt]{floatflt}
\usepackage{float}
\usepackage{graphicx}
%\usepackage[dvips]{epsfig}
%BASE URL's for RFC repositories:
%ftp://ftp.ietf.org/rfc - the home
%ftp://sunsite.dk/mirrors/rfc - danish mirror
\usepackage[pdftitle={Designing Internetworks with
IPv6},pdfauthor={Henrik Lund Kramshoej},colorlinks,
baseurl=ftp://sunsite.dk/mirrors/rfc/,
urlcolor={blue},menucolor={blue},citecolor={blue},
pagecolor={blue},linkcolor={blue}]{hyperref}   
\usepackage{makeidx}
\usepackage{calc}
%\usepackage{layout}
\usepackage{amsmath}

\usepackage{hhline}
\usepackage{tabularx}
\usepackage{array}
\usepackage{multirow}
\usepackage{dcolumn}     
\newcolumntype{.}{D{.}{.}{-1}}
\usepackage[danish,english]{babel}
\usepackage{wasysym}
\usepackage{fancyvrb}
\usepackage{alltt}
\usepackage{microtype}
%\usepackage{ulem}

\makeindex
% LaTeX companion page 93
% page layout
% A4 210mm * 297mm
% left and right margin 25 mm
%\setlength{\hoffset}{-1 inch +2.5 cm }
\setlength{\hoffset}{-5.4mm}
\setlength{\oddsidemargin}{0mm}   
\setlength{\textwidth}{17cm}

% top and bottom margin 25 mm
\setlength{\voffset}{-15.4mm}
\setlength{\topmargin}{0mm}
\setlength{\topmargin}{0mm}
\setlength{\marginparwidth}{0mm}
\setlength{\marginparsep}{3mm}
% \setlength{\marginparpush}{0mm}

%\setlength{\textheight}{24.7cm - 96pt}
\setlength{\textheight}{\paperheight-\headheight-\headsep-\footskip-2cm}
\setlength{\parindent}{0mm}

\pagestyle{fancy}
\renewcommand{\chaptermark}[1]%
                  {\markboth{#1}{}}
\renewcommand{\sectionmark}[1]%
                  {\markright{\thesection\ #1}}
% header
\lhead{\fancyplain{}{\bfseries\rightmark}}
\chead{}{}
\rhead{}{}
%\rhead{\fancyplain{}{}}
%\lfoot{\fancyplain{}{\bfseries\leftmark}}
\cfoot{\small Copyright \copyright\ 2003-2006, Henrik
  Lund Kramshj} 
\rfoot{\bfseries\thepage}

% plain also redefined so chapter pages look alright
\fancypagestyle{plain}{%
\fancyhf{}%
%\lfoot{\bfseries \leftmark}
\cfoot{\small Copyright \copyright\ 2003-2006, Henrik
  Lund Kramsh�j} 
%\cfoot{\small Copyright \copyright\ 2003 Henrik Lund Kramsh�j}
\rfoot{\bfseries\thepage}
\renewcommand{\headrulewidth}{0pt}}


% std sizes: tiny scriptsize footnotesize small normalsize large Large
% LARGE huge Huge

% Usefull commands
\newcommand{\Index}[1]{#1\index{#1}}
\newcommand{\titel}[1]{{\emph{#1}}}
\newcommand{\command}[1]{{\tt #1}}
\newcommand{\update}[1]{{\bf \emph{\lightning\ #1 \lightning\ }\\}}
\newcommand{\link}[1]{\href{#1}{#1}}
\newcommand{\rfclink}[1]{\href{rfc#1.txt}{RFC-#1}}
\newcommand{\file}[1]{{\scriptsize\verbatimtabinput{#1}}} 

\newcommand{\hlkimage}[2]{\begin{center}\colorbox{white}{\includegraphics[width=#1]{#2}}\end{center}}


% fancyvrb package stuff
%\fvset{frame=single,numbers=left, numbersep=3pt}
\fvset{frame=single,%xleftmargin=10mm,xrightmargin=10mm,
fontsize=\scriptsize,boxwidth=auto}



\newcommand{\unix}{{\sc unix} }
\def\unix{{\sc unix}\footnote{{\sc unix} is trademark of X/Open in
the USA and other countries. I use the term to describe a generic family of
operating systems with a common characteristics.}\gdef\unix{{\sc unix}
}} 

\newcommand{\diku}{DIKU}
\def\diku{DIKU, Department of Computer Science University of
Copenhagen, Denmark\gdef\diku{DIKU}}  

\newcommand{\dikudk}{DIKU}
\def\dikudk{DIKU, Datalogisk Institut ved K�benhavns
Universitet\gdef\dikudk{DIKU}} 

% Line spacing
%\renewcommand{\baselinestretch}{1}
\renewcommand{\baselinestretch}{1.5}
%\renewcommand{\baselinestretch}{2}

\newcommand{\linieafstand}[1]{\renewcommand{\baselinestretch}{#1}\normalsize}
\newcommand{\normal}{\renewcommand{\baselinestretch}{1}\normalsize}
\newcommand{\double}{\renewcommand{\baselinestretch}{1.5}\normalsize}

\addto\captionsdanish{%
  \renewcommand{\chaptername}%
    {�velse}%
}

%%% Local Variables: 
%%% mode: plain-tex
%%% TeX-master: "mac-osx-bog"
%%% End: 
