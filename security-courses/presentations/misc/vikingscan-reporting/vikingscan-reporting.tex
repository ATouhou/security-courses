\documentclass[18pt,landscape,a4paper]{foils}
\usepackage{sec6slides}

\begin{document}

\mytitlepage{VikingScan reporting}
{Hack.lu}


\slide{What is pentesting}

\begin{list1}
\item Running a lot of tools
\item Running some more specialized tools
\item then doing some thinking and running more tools
\item Add some manual stuff like writing customized exploits
\end{list1}

\slide{... but what about reporting results}

\begin{list1}
\item Not very technical
\item Value add from companies - like mine :-)  
\end{list1}

\slide{Proposal}

\begin{list1}
\item Create some tools to allow more or less automated reporting  
\item Based on output from tools compile a report that can be edited
\end{list1}

\slide{VikingScan}

\begin{list1}
\item I have created the project VikingScan on Sourceforge, and
  started added and refining my various pieces   
\end{list1}


\slide{Feedback needed}

\begin{list1}
\item Currently it uses Docbook XML, has various good things:
\begin{list2}
\item Interoperable and can create RTF, HTML and PDF easily    
\end{list2}
\item and some bad things:
\begin{list2}
\item Toolchain and templates/stylesheets can be pure hell    
\end{list2}
\end{list1}

\slide{Current directions}

\begin{list1}
\item Make easy offline generation from Nmap XML output into report
  working again
\item Go more into the database and provide solution based on:
\begin{list2}
\item Postgresql - just one database for now
\item Openjade
\item LaTeX - only used for PDF currently    
\item Adding web fronted, proof of concept using Ravenous has initial
  steps and another project done by some students have more wizard
  like features. Probably needs complete rewrite, but WONT be PHP ;-)
\end{list2}
\item Features needed:
\begin{list2}
\item Basic import OK - proof of concept, data can be imported
\item Extend database schema to allow merging data from multiple scans
\item Extend database to allow integration with OSVDB (and nessus scans?)
\end{list2}

\end{list1}

\slide{Demo}

\begin{list1}
\item Kind of a demo, showing the files and stuff
\end{list1}

\slide{Tool writers}

\begin{list1}
\item Consider making your tool output in XML
\item Yes, XML can be hard to decipher but has clear boundaries
  between fields  
\end{list1}

\end{document}
