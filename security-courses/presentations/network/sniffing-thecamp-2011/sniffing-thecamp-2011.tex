\documentclass[20pt,landscape,a4paper,footrule]{foils}
\usepackage{solido-network-slides}


% Beskrivelse: Det kunne f.eks. handle om hvordan man undg�r at afsl�re sine brugernavne og adgangskoder n�r man sidder p� et offentligt netv�rk, og evt. andre ting man skal v�re opm�rksom p� som internet-bruger. Herunder en demonstration af hvor nemt det er at sniffe p� netv�rket.
% Type: Foredrag
% Forslagstiller: Jette Derriche
% Foredragsholder: Henrik Kramsh�j (hlkv6), Flemming Jacobsen (F3), Thomas Rasmussen (Tykling)
% Grundet sidste �rs diskussion omkring snifning af data g�res alle opm�rksomme p� risikoen ved at bruge usikre protokoller som FTP, SMTP, HTTP (uden S) osv.
% Der ops�ttes testnetv�rk som sender traffik der kan sniffes og der snakkes om programmer og metoder til opsamling af hemmeligheder.
% Det anbefales at alle der konfigurerer egen webmail osv. sp�rger om hj�lp til ops�tning af SSL ;-)
% Det er meningen at vi her under kontrollerede former og venlige sj�le g�r alle opm�rksomme p� problemer - s� vi alle fremover, ude i den store verden benytter bedre protokoller.



\begin{document}
\selectlanguage{english}
\mytitlepage{Sikkerhed - sniffning p� netv�rket}


\vskip 2cm
\centerline{\footnotesize Slides are available as PDF}

\slide{Goal}
%\hlkimage{6cm}{kame-noanime-small.png}

\begin{list1}
\item Hvorfor? Sidste �r :-)
\item Sniffe - demoer hvordan man henter post med POP3 og en FTP server, update af webhotel scenarier

\item BackTrack 5 bruger vi til at sniffe, vi bruger Wireshark til prim�re demo, n�vner Ettercap osv.


\item Demo: l�sninger,
POP3S, SSL/TLS, SMTP over TLS osv. 

\item Mere avanceret: 
OpenSSH tunnel, SSL VPN - generelt, eksempel med OpenVPN, FileZilla - underst�tter vist SFTP/SCP

\item sniffe igen

\item Advanced sniffing:
\item Tcpdump wizardry Tyklol

\item HLK: netflow opsamling

\end{list1}

%\centerline{Participate and demonstrate IPv6 working - make the turtle dance!}

\slide{Internet today}

\hlkimage{14cm}{images/server-client.pdf}

\begin{list1}
\item Clients and servers
\item Rooted in academic networks
\item Protocols which are more than 20 years old, moved to TCP/IP in 1981 
\end{list1}




\myquestionspage


\slide{More Information}

\hlkimage{18cm}{twitter-security-feed.png}

\begin{list1}
\item Twitter has become an important new ressource for lots of stuff
\item Twitter has replaced RSS for me 
\end{list1}

\slide{Ressources}

\begin{list1}
\item \emph{Guidelines for the Secure Deployment of IPv6}, SP800-119, NIST\\
\link{http://csrc.nist.gov/publications/nistpubs/800-119/sp800-119.pdf}
\item \emph{The Second Internet: Reinventing Computer Networks with IPv6}, Lawrence E. Hughes, October 2010,\\ \link{http://www.secondinternet.org/}
\item \emph{IPv6 Network Administration}
af David Malone og Niall Richard Murphy
 - god til real-life admins, typisk
O'Reilly bog
\item \emph{IPv6 Essentials} af Silvia Hagen, O'Reilly 2nd edition (May 17, 2006)
	god reference om emnet
\item \emph{IPv6 Core Protocols Implementation}
af Qing Li, Tatuya Jinmei og Keiichi Shima
\item \emph{IPv6 Advanced Protocols Implementation}
af Qing Li, Jinmei Tatuya og Keiichi Shima
\item - flere andre
\end{list1}



\end{document}
