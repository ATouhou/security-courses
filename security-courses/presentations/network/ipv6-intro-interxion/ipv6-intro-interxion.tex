\documentclass[20pt,landscape,a4paper,footrule]{foils}
\usepackage{solido-network-slides}

\begin{document}
\selectlanguage{english}
\mytitlepage{Er dit netv�rk klar til IPv6?}

\vskip 2cm
\centerline{\footnotesize Slides are available as PDF}

\slide{Goal}
\hlkimage{8cm}{kame-noanime-small.png}

\begin{list1}
\item Introduce IPv6 
\item The future is here
\item Denmark is falling behind on IPv6 
\item Why you should implement IPv6
\item Ressources
\end{list1}


\slide{Internetworking: history}

\begin{list2}  
\item[1960s]  L. Kleinrock, MIT packet-switching theory,
 J. C. R. Licklider,MIT - notes , \\
Paul Baran: On Distributed Communications
\item[1969]  ARPANET 4 nodes
\item[1971]  14 nodes
\item[1973]  Design of Internet Protocols started
\item[1973]  Email is about 75\% of all ARPANET traffic
\item[1974]  TCP/IP: Cerf/Kahn: A protocol for Packet
        Network Interconnection
\item[1983]  EUUG $\rightarrow$ DKUUG/DIKU forbindelse
\item[1988]  About 60.000 systems on the internet - 
        The Morris Worm hits about 10\%
\item[2010] IANA reserved blocks 8\% (March 2010) - \link{http://www.potaroo.net/tools/ipv4/}
\item[2011] IANA Unallocated Address Pool Exhaustion (February 3)
\end{list2}

\slide{Status idag p� internet}

\hlkimage{18cm}{ipv4-address-report-2012.png}

Kilde: \link{http://www.potaroo.net/tools/ipv4/}


\slide{Why IPv6}

\hlkimage{5cm}{ipv4-run-out.png}
	
\slide{OSI \& Internet Protocols}

\hlkimage{14cm,angle=90}{images/compare-osi-ip.pdf}

\slide{IPv6: Internet redesigned? - no!}
 
\begin{list1}
\item Preserve the good stuff
\item back to basics, internet as it used to be!
\item fate sharing - connection rely on end points, not intermediary NAT boxes
\item end-to-end transparency - you have an address and I have an address
\item Wants: bandwidth +10G, low latency/predictable latency, Quality of Service, Security
\end{list1}

\vskip 5mm
\centerline{\color{titlecolor}\LARGE \bf IPv6 is evolution, not revolution}
\vskip 5mm


\slide{How to use IPv6}

\begin{center}
\vskip 3 cm
\hlkbig
www.solidonetworks.com

hlk@solidonetworks.com
\end{center}

\slide{Really how to use IPv6?}

\begin{list1}
\item Get IPv6 address and routing
\item Add AAAA (quad A) records to your DNS
\item Done
\end{list1}

\begin{alltt}
\LARGE
www     IN	A       91.102.95.20
        IN	AAAA    2a02:9d0:10::9
\end{alltt}





\slide{Allocating IPv6 addresses} 

\begin{list1}
\item You have plenty!
\item Providers will typically get /32
\item Providers will typically give you /48 or /56
\item Your /48 can be used for:
\begin{list2}
\item 65536 subnets
\item Each subnet has $2^{64}$ addresses
\end{list2}
\end{list1}

\vskip 2cm
\centerline{Notice: you probably already have IPv6 traffic in your network!}

\slide{The future is here}

What can we use IPv6 for?

\begin{list1}
\item Connectivity - no address conflicts
\item End to end transparency - logging is easier, no NAT
\item Fate sharing - connection rely on end points
\item Two way communication - think chat protocols, file transfer, p2p services
\item Easier redundancy, no NAT and less state in the network
\item Easier security - flat networks, simpler rulesets
\item High performance - bigger packets, and NO carrier grade NAT
\end{list1}


\slide{New applications}

\begin{list1}
\item Who would have guessed the applications?
\item World Wide Web 
\item World Wide chatting - MSN, IRC, Jabber etc.
\item Distribution of software - peer to peer
\item Facebook
\item Twittter
\item Foursquare
\item Whats next?
\item Smart internet devices + GPS + video + users = fun and business!
\item Sometimes named the Internet of Things
\end{list1}

\slide{Interxion clients}
\begin{list1}
\item Now we can connect
\item We can make things happen that would be harder before
\item Peer to multiple peers
\item Use services directly at each others cages
\item Peer using IPv6 at Interxion via DIX
\item Restructure our networks
\item Use IPv6 for testing network changes ;-)
\end{list1}


\slide{IPv6 business case}

\begin{list2}
\item An almost unlimited scalability with a very large IPv6 address space ($2^{128}$ addresses)\\
enabling IP addresses to each and every device.

\item Address self-configuration mechanisms, easing the deployment.

\item Improved security and authentication features, such as mandatory IPSec capacities and the possibility to use of the address space to include encryption keys.

\item Peer-to-peer connectivity, solving the NAT barrier with specific and permanent IP addresses for any device and/or user of the Internet.

\item Mobility features, enabling a seamless connexion when moving from one access point to another access point on the Internet.

\item Multi cast and any cast functionalities.

\item IPv6 will provide an easier remote interaction with each and every device with a {\bfseries direct integration to the Internet.} In other words, IPv6 will make possible to move from a network of servers, to a network of things.

\end{list2}

\centerline{ Business case for IPv6 is {\bf continuity}}


{\footnotesize Partial quote from http://www.smartipv6building.org/index.php/en/ipv6-potential}




\slide{IPv6 ripeness}

\hlkimage{28cm}{userfiles-v6-ripeness.png}

\centerline{IPv6 ripeness from \link{http://labs.ripe.net/}}

\slide{Curent status Denmark}

\begin{list1}
\item Too little interest - less than 100 people thinking about IPv6?
\item Some providers have some IPv6 connectivity
\item NO ISPs have IPv6 to consumers
\item NO ISPs market IPv6 as a product, except me perhaps :-)
\item Perceived NO NEEED
\vskip 2 cm
\pause
\item Free, a major French ISP rolled-out IPv6 at end of year 2007
\item XS4All As of August 2010 native IPv6 DSL connections became available to almost all their customers.
\end{list1}

Source: \link{http://en.wikipedia.org/wiki/IPv6_deployment}

\slide{Danish sites with IPv6}
\begin{list1}
\item Name servers for .dk\\
p.nic.dk has IPv6 address 2001:500:14:6036:ad::1\\
s.nic.dk has IPv6 address 2a01:3f0:0:303::53\\
b.nic.dk has IPv6 address 2a01:630:0:80::53
\item ns1.gratisdns.dk has IPv6 address 2a02:9d0:3002:1::2
\item ns1.censurfridns.dk has IPv6 address 2002:d596:2a92:1:71:53::
\item www.solidonetworks.com has IPv6 address 2a02:9d0:10::9
\item Most others have no IPv6 address
\end{list1}


\slide{Danish resources - get involved}

\hlkimage{10cm}{taskforce-logo.jpg}

\begin{center}
Danish IPv6 task force - unofficial\\
\link{http://www.ipv6tf.dk}
\end{center}


\myquestionspage


\slide{Books on IPv6}

\begin{list1}
\item \emph{The Second Internet: Reinventing Computer Networks with IPv6}\\ \link{http://www.secondinternet.org/}

\item \emph{Preparing an IPv6 Addressing Plan}\\ {\small \link{https://labs.ripe.net/Members/steffann/preparing-an-ipv6-addressing-plan}}

\item \emph{Guidelines for the Secure Deployment of IPv6} NIST SP 800-119\\
\link{http://csrc.nist.gov/publications/nistpubs/800-119/sp800-119.pdf}

\item \emph{IPv6 Network Administration}
David Malone and Niall Richard Murphy
\item \emph{IPv6 Core Protocols Implementation}
af Qing Li, Tatuya Jinmei og Keiichi Shima
\item \emph{IPv6 Advanced Protocols Implementation}
af Qing Li, Jinmei Tatuya og Keiichi Shima
\item - flere andre se reviews p� \link{http://getipv6.info/index.php/Book_Reviews}
\item \emph{IPv6 Essentials} Silvia Hagen, O'Reilly 2nd edition (May 17, 2006)
\end{list1}

\hlkprofiluk

\end{document}
