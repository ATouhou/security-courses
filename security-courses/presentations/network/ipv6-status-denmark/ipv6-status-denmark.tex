\documentclass[20pt,landscape,a4paper,footrule]{foils}
\usepackage{solido-network-slides}

% 

%Dette indl�g giver en kort status over projekter som er indf�rt i Danmark med udgangspunkt i managed hosting for store danske websteder og infrastrukturer for danske virksomheder. Henrik Kramsh�j har arbejdet med IPv6 siden midten af 1990'erne og har opdateret flere netv�rk til at underst�tte IPv6. Forh�bentlig kan der ogs� afsl�res store danske websites som i anledning af "IPv6 World day" den 8. juni k�rer IPv6.
%v/ Henrik Lund Kramsh�j, Solido Networks ApS


\begin{document}
\selectlanguage{english}
\mytitlepage{IPv6 Status Denmark}


\vskip 2cm
\centerline{\footnotesize Slides are available as PDF}

\slide{Goal}
\hlkimage{6cm}{kame-noanime-small.png}

\begin{list1}
\item Introduce IPv6 - facts and features
\item IPv6 Status Denmark
\item Enabled providers and sites
\item How to get your site on IPv6
\end{list1}

%\centerline{Participate and demonstrate IPv6 working - make the turtle dance!}

\slide{Internet today}

\hlkimage{14cm}{images/server-client.pdf}

\begin{list1}
\item Clients and servers
\item Rooted in academic networks
\item Protocols which are more than 20 years old, moved to TCP/IP in 1981 
\end{list1}


%%%%%%%%%%%%%%%%%%%%%%%%%%%%%%%%%%%%%%%%%%%%%%%%%%%%%%%%%%%%%%%%%%%%%%%
%% Node 1: UCLA (30 August, hooked up 2 September) 
%% Node 2: Stanford Research Institute (SRI) (1 October) 
%% Node 3: University of California Santa Barbara (UCSB) (1 November) 
%% Node 4: University of Utah (December) 
%% RFC 4: Network Timetable
%% http://www.zakon.org/robert/internet/timeline/

\slide{Internetworking: history}

\begin{list2}  
\item[1960s]  L. Kleinrock, MIT packet-switching theory,  J. C. R. Licklider, MIT - notes 
  Paul Baran: On Distributed Communications
\item[1969]  ARPANET 4 nodes
\item[1971]  14 nodes
\item[1974]  TCP/IP: Cerf/Kahn: A protocol for Packet
        Network Interconnection
\item[1983]  Switching from NCP to IP/TCP
\item[1983]  EUUG $\rightarrow$ DKUUG/DIKU forbindelse
\item[1988]  About 60.000 systems on the internet - 
        The Morris Worm hits about 10\%
\item[2010] IANA reserved blocks 7\% (Maj 2010) - \link{http://www.potaroo.net/tools/ipv4/}
\item[2011] February 3 IANA pool ran out - last 5 /8 allocated to RIRs
\item[2011] April 15 APNIC ran into their last /8 and started a more restrictive policy
\end{list2}

\slide{Future - 2010 and beyond}

\begin{quote}
{\bf The Mobile Network in 2010 and 2011}\\
Global mobile data traffic grew 2.6-fold in 2010, nearly tripling for the third year in a row. The 2010 mobile data traffic growth rate was higher than anticipated. Last year's forecast projected that the growth rate would be 149 percent. This year's estimate is that global mobile data traffic grew 159 percent in 2010.\\
...\\
Last year's mobile data traffic was three times the size of the entire global Internet in 2000. Global mobile data traffic in 2010 (237 petabytes per month) was over three times greater than the total global Internet traffic in 2000 (75 petabytes per month).\\
...\\
There will be 788 million mobile-only Internet users by 2015. The mobile-only Internet population will grow 56-fold from 14 million at the end of 2010 to 788 million by the end of 2015.

\end{quote}

Kilde:
\emph{Cisco Visual Networking Index: Global Mobile Data Traffic Forecast Update, 2010 - 2015}
%\link{http://www.cisco.com/en/US/solutions/collateral/ns341/ns525/ns537/ns705/ns827/white_paper_c11-520862.html}


\slide{How to use IPv6}

\begin{center}
\vskip 3 cm
\hlkbig
www.solidonetworks.com

hlk@solidonetworks.com
\end{center}

\slide{Really how to use IPv6?}

\begin{list1}
\item Get IPv6 address and routing
\item Add AAAA (quad A) records to your DNS
\item Done
\end{list1}
\vskip 1cm
\centerline{\Large www.solidonetworks.com}

\begin{alltt}
\LARGE
www     IN	A       91.102.95.20
        IN	AAAA    2a02:9d0:10::9
\end{alltt}





\slide{IPv6 Status Denmark}

\begin{list1}
\item IT- og Telestyrelsen are becoming more active
\item Unofficial IPv6 task force at \link{http://www.ipv6tf.dk/}
\item Other initiatives \link{http://world-ipv6-day.dk/}
\item Major providers are ready on back bones
\item Internet Providers are increasingly becoming ready
\end{list1}

\slide{IPv6 in the Nordic region}

\hlkimage{14cm}{ipv6-nordic.png}

\link{http://v6asns.ripe.net/v/6?s=_ALL;s=DK;s=SE;s=NO;s=NL}


\slide{Current status Denmark}

\begin{list1}
\item Too little interest - less than 100 people thinking about IPv6?
\item Some providers have some IPv6 connectivity
\item Perceived NO NEEED
\vskip 2 cm
\pause
\item Free, a major French ISP rolled-out IPv6 at end of year 2007
\item XS4All As of August 2010 native IPv6 DSL connections became available to almost all their customers.
\end{list1}

Source: \link{http://en.wikipedia.org/wiki/IPv6_deployment}

\slide{Enabled providers and sites}

\begin{list1}
\item The ones we know of who support IPv6:\\
Nianet, TDC, Netgroup, Lynero, Solido, Gratisdns, DK-hostmaster
\item The missing in action - what are they doing?\\
Telenor, Telia
\item The ones we think are ignoring IPv6:
Jaynet, 
\vskip 1 cm
\item Enabled sites: \link{http://www.tdc.dk}, \link{http://www.lynero.dk},\\ \link{http://www.solidohosting.com},\link{www.feriebolig-spanien.dk},\\
\link{http://www.dk-hostmaster.dk},\link{http://mirrors.dotsrc.org
}
\end{list1}





\slide{How to get your site on IPv6}

Practical information for your network

\begin{list1}
\item Strategy and actions points
\begin{list2}
\item Collect information about IPv6 
\item Collect information about your network
\item Collect information about your hosts and services
\item Ask your providers for IPv6 plans
\item Experiment with IPv6 - today
\item Implement small proof of concept, in production!
\item Expand coverage
\end{list2}
\end{list1}

\slide{Implications}

\hlkimage{2cm}{IPv6ready.png}

\begin{list1}
\item For an IPv4 enterprise network, the existence of an IPv6 overlay network has several of implications:
\begin{list2}
\item The IPv4 firewalls can be bypassed by the IPv6 traffic, and leave the security door wide open.
\item Intrusion detection mechanisms not expecting IPv6 traffic may be confused and allow intrusion
\item In some cases (for example, with the IPv6 transition technology known as 6to4), an internal PC can communicate directly with another internal PC and evade all intrusion protection and detection systems (IPS/IDS). Botnet command and control channels are known to use these kind of tunnels.
\end{list2}
\end{list1}

Kilde:\\
{\footnotesize\link{http://www.cisco.com/en/US/prod/collateral/iosswrel/ps6537/ps6553/white_paper_c11-629391.html}}


\slide{Collect information about IPv6}

\begin{list1}
\item \emph{Guidelines for the Secure Deployment of IPv6}, SP800-119, NIST\\
\link{http://csrc.nist.gov/publications/nistpubs/800-119/sp800-119.pdf}
\item \emph{The Second Internet: Reinventing Computer Networks with IPv6}, Lawrence E. Hughes, October 2010,\\ \link{http://www.secondinternet.org/}
\item \emph{IPv6 Network Administration}
af David Malone og Niall Richard Murphy
\item \link{http://www.ripe.net}
\item This presentation \smiley
\end{list1}

\slide{Allocating IPv6 addresses} 

\begin{list1}
\item You have plenty!
\item Providers and LIRs will typically get /32
\item Providers will typically give organisations /48 or /56
\item Your /48 can be used for:
\begin{list2}
\item 65536 subnets - all host subnets are /64
\item Each subnet has $2^{64}$ addresses
\end{list2}
\end{list1}

\slide{Preparing an IPv6
Addressing Plan}

\hlkimage{20cm}{ipv6-address-plan-ripe.png}

{\footnotesize \link{http://www.ripe.net/training/material/IPv6-for-LIRs-Training-Course/IPv6_addr_plan4.pdf}}

\slide{Example adress plan input}

\hlkimage{22cm}{ipv6-linked-to-ipv4.png}

\centerline{Easy and coupled with VLAN IDs it will work \smiley}

\slide{Run IPv6 in production}

\begin{list1}
\item Make sure you establish IPv6 in {\bf production}
\item Enabling service on IPv6 without production - bad experience for users
\item Start by enabling your DNS servers for IPv6 - and DNSSEC - and DNS over TCP\\
Remember that your firewall might have problems with large DNS packets
\item Add a production IPv6 router - hardware device or generic server
\item Tunnels are OK, and SixXS consider their service production
\end{list1}

\slide{F5 load balancer example}

\hlkimage{\linewidth-3cm}{f5-load-balancer.png}

\slide{World IPv6 Day}
\begin{quote}
{\bf About World IPv6 Day}

On 8 June, 2011, Google, Facebook, Yahoo!, Akamai and Limelight Networks will be amongst some of the major organisations that will offer their content over IPv6 for a 24-hour "test flight". The goal of the Test Flight Day is to motivate organizations across the industry - Internet service providers, hardware makers, operating system vendors and web companies - to prepare their services for IPv6 to ensure a successful transition as IPv4 addresses run out.

Please join us for this test drive and help accelerate the momentum of IPv6 deployment.
\end{quote}

\centerline{\link{http://isoc.org/wp/worldipv6day/} and \link{http://test-ipv6.com/}}


\slide{IPv6 business case}


\begin{list2}
\item An almost unlimited scalability with a very large IPv6 address space ($2^128$ addresses), enabling IP addresses to each and every device.

\item Address self-configuration mechanisms, easing the deployment.

\item Improved security and authentication features, such as mandatory IPSec capacities and the possibility to use of the address space to include encryption keys.

\item Peer-to-peer connectivity, solving the NAT barrier with specific and permanent IP addresses for any device and/or user of the Internet.

\item Mobility features, enabling a seamless connexion when moving from one access point to another access point on the Internet.

\item Multi cast and any cast functionalities.

\item IPv6 will provide an easier remote interaction with each and every device with a {\bfseries direct integration to the Internet.} In other words, IPv6 will make possible to move from a network of servers, to a network of things.

\end{list2}

\centerline{ Business case for IPv6 is {\bf continuity}}


{\footnotesize Partial quote from http://www.smartipv6building.org/index.php/en/ipv6-potential}



\slide{Conclusion}

\begin{center}
\vskip 5mm
{\color{titlecolor}\LARGE \bf IPv6 is here already - use it}
\vskip 5mm


\link{http://www.ipv6actnow.org/}

\link{http://digitaliser.dk/group/374895}

\link{http://www.ipv6tf.dk}

\end{center}


\slide{Up and running with IPv6}

\begin{list1}
%\item Join the fun - join the wireless network
\item Use ping/ping6 and traceroute to test connectivity
\item Try in your browser:
\begin{list2}
\item \link{http://www.kame.net} Dancing turtle
\item \link{http://www.ripe.net} RIPE, look for address up right corner 
\item \link{http://loopsofzen.co.uk/} Play a game
\item \link{https://www.sixxs.net/} Apply for IPv6 tunnel 
\end{list2}
\item Done \smiley
\end{list1}

\myquestionspage


\slide{More Information}

\hlkimage{18cm}{twitter-security-feed.png}

\begin{list1}
\item Twitter has become an important new ressource for lots of stuff
\item Twitter has replaced RSS for me 
\end{list1}

\slide{Ressources}

\begin{list1}
\item \emph{Guidelines for the Secure Deployment of IPv6}, SP800-119, NIST\\
\link{http://csrc.nist.gov/publications/nistpubs/800-119/sp800-119.pdf}
\item \emph{The Second Internet: Reinventing Computer Networks with IPv6}, Lawrence E. Hughes, October 2010,\\ \link{http://www.secondinternet.org/}
\item \emph{IPv6 Network Administration}
af David Malone og Niall Richard Murphy
 - god til real-life admins, typisk
O'Reilly bog
\item \emph{IPv6 Essentials} af Silvia Hagen, O'Reilly 2nd edition (May 17, 2006)
	god reference om emnet
\item \emph{IPv6 Core Protocols Implementation}
af Qing Li, Tatuya Jinmei og Keiichi Shima
\item \emph{IPv6 Advanced Protocols Implementation}
af Qing Li, Jinmei Tatuya og Keiichi Shima
\item - flere andre
\end{list1}



\slide{Danish resources - get involved}

\hlkimage{10cm}{taskforce-logo.jpg}

\centerline{ Danish IPv6 task force - unofficial
\link{http://www.ipv6tf.dk}}


\slide{VikingScan.org - free portscanning}

\hlkimage{18cm}{vikingscan.png}
%\vskip 1cm 
%\centerline{\link{http://www.vikingscan.org}}


\hlkprofiluk


\end{document}
