\documentclass[18pt,landscape,a4paper]{foils}
\usepackage{sec6slides}

\begin{document}

\mytitlepage{OpenBSD Packages\\\small - The Best Package Management Ever}
{EuroBSDCon2007 deathmatch}

\hlkimage{4cm}{openbsd-logo.png}

\slide{What is OpenBSD}

\hlkimage{15cm}{images/puffy41}

\begin{quote}
The OpenBSD project produces a {\bf FREE}, multi-platform 4.4BSD-based
UNIX-like operating system. Our efforts emphasize portability,
standardization, correctness, proactive security and integrated
cryptography. OpenBSD supports binary emulation of most programs from
SVR4 (Solaris), FreeBSD, Linux, BSD/OS, SunOS and HP-UX. 
\end{quote}

You can download for free but only get stickers when you buy, so go to
Wim and buy some :-)

\slide{OpenBSD packages and ports}

\begin{list1}
\item What is OpenBSD packages
\item OpenBSD packages are ready-to-install binary packages
\item These packages contain third party applications 
which can be installed, upgraded and uninstalled on OpenBSD
\item Source for most of this presentation is experience and:\\
15 - The OpenBSD packages and ports system\\
\link{http://www.openbsd.org/faq/faq15.html}
The OpenBSD ports tree is based on a concept originally borrowed from
FreeBSD - but improved of course :-)
\item A lot of packages are announced with other OpenBSD news at:\\
\link{http://undeadly.org}
\end{list1}

\slide{Binary Packages are the product}

\begin{list1}
\item Most packages are ready-to-install binary packages and this is
  the considered end-user product
\item The goal of the developers are not to make a great ports-tree,
  but to make great packages
\item The patches and packaging of packages are not meant for
  end-users, but some packages need special consideration
\item Don't waste your time building packages again and again, like Gentoo
\item Of course as a result the ports tree is great too :-)
\end{list1}

\slide{Building packages}

\begin{list1}
\item Sometime a few packages need to be built:
\item Java - due to license issues with sourcecode, requires download
  with click-through license  
\item Flavors, sometimes you want an application configured exactly
  like you like it. Not all flavors are built by the OpenBSD project
\end{list1}

\slide{Package administration}

\begin{alltt}
hlk@pumba:all$ pwd
/userdata/OpenBSD/packages/amd64/all
hlk@pumba:all$ sudo pkg_add nsping-0.8.tgz
nsping-0.8: complete               
\end{alltt}

\begin{list1}
\item Main commands are:
\begin{list2}
\item \verb+pkg_add+ install packages
\item \verb+pkg_delete+ remove installed packages
\item \verb+pkg_info+ display information about installed packages   
\end{list2}
\item Deleting all packages can be done with:\\ 
\verb+pkg_delete -q /var/db/pkg/*+
\end{list1}

\slide{Easy distribution}

\begin{list1}
\item Getting the packages installed requires access to the package
  file
\item \verb+pkg_add+ command understand URL notation, so you don't
  need to copy all package files to a server!
\item The command can be used with files or URL referring to other
  locations such as:
\begin{list2}
\item FTP good ole File Transfer Protocol
\item HTTP Hyper Text Transfer Protocol - easy distribution
\item HTTPS or Secure Copy Protocol (SCP) - secure distribution
\end{list2}
\item Valid commands:
\end{list1}
\begin{alltt}
\small
pkg_add -v ftp://ftp.openbsd.org/pub/OpenBSD/2.7/packages/i386/m4-1.4.tgz
pkg_add -v scp://login@host/usr/ports/packages/sparc/all/tcl-8.4.7.tgz  
\end{alltt}


\slide{Adding packages from URL}
\begin{alltt}
\small
hlk@pumba:hlk$ sudo pkg_add \verb+\+
ftp://ftp.se.openbsd.org/pub/OpenBSD/4.1/packages/amd64/zsh-4.2.6p2.tgz
Shell /usr/local/bin/zsh appended to /etc/shells
zsh-4.2.6p2|*****************************************       | 81%

\emph{then a few seconds more and:}
Shell /usr/local/bin/zsh appended to /etc/shells
zsh-4.2.6p2: complete  
\end{alltt}
%$

\slide{Package upgrading}

\begin{alltt}
\small
PKG_ADD(1)                 OpenBSD Reference Manual                 PKG_ADD(1)

NAME
     pkg_add - install software package distributions
...
 -i       Switch on interactive mode.  pkg_add may ask questions to the
          user if faced with difficult decisions.
...
 -u       Update the given pkgname(s), and anything it depends upon.  If
          no pkgname is given, pkg_add will update all installed packages.
          This relies on PKG_PATH to figure out the new package names.  
\end{alltt}

\begin{list1}
\item New option -u for package upgrades
\item Just go into a directory with new package files and upgrade
\end{list1}

\slide{}

\begin{alltt}
\small
hlk@laura:all$ pwd              
/userdata/OpenBSD/packages/amd64/all
hlk@laura:all$ sudo pkg_add -ui 

(go to movie)

\end{alltt}

\slide{Stable Packages}

\begin{list1}
\item OpenBSD has three flavors for the Base Operating System:
\begin{list2}
\item -release: The version of OpenBSD shipped every six months on CD.
\item -stable: Release, plus patches considered critical to security and reliability.
\item  -current: Where new development work is presently being done,
    and eventually, it will turn into the next release. 
\end{list2}
\item But OpenBSD also extends this to the packages, so OpenBSD has
\begin{list2}
\item -stable packages
\item -current packages
\end{list2}
\item So if you want to run a stable and secure server you can follow
  the stable branch for ports!
\item For the i386 and amd64 architectures you can also find updatede
  packages at:\\
\link{http://www.openbsd.org/pkg-stable.html}  
\end{list1}

\slide{Complete OpenBSD server upgrade}

\begin{list1}
\item No matter if you want to run -stable or snapshots of -current
  (or build OS and ports yourself) you can upgrade a server easily:
\item 1) Upload files to your server
\item 2) Replace kernel /bsd 
\item 3) Reboot
\item 4) Upgrade system
\item 5) Upgrade packages 
\item 6) Drink beer!
\end{list1}

If files are placed on a server the upgrade can be done in
about 1 hour - including packages!

\slide{OpenBSD Release Cycle}

\begin{list1}
\item A new OpenBSD version is released twice a year
\item Next one is The OpenBSD 4.2 Release\\
To be released Nov 1, 2007  
\item Last two releases are supported with security fixes, so you can
  decide to upgrade once a year or twice a year
\item I recommend upgrading with each new release, since they always
  include a lot of great new stuff!
\item See more on: \link{http://www.openbsd.org/42.html}
\end{list1}
\slide{Summary}

\hlkimage{10cm}{images/unix_plate-med.jpg}

\begin{list1}
\item Remember:\\
\item OpenBSD provides a secure foundation for your applications
\item OpenBSD packages makes it easy to install applications
\item OpenBSD has a predictable release cycle
\item OpenBSD has the easiest upgrade for both base and packages
\item Theo is always right and OpenBSD is free :-)
\end{list1}


\myquestionspage


\end{document}
