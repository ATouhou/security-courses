\slide{Kryptografi}

\hlkimage{18cm}{images/crypto-rot13.pdf}

\begin{list1}
\item Kryptografi er l�ren om, hvordan man kan kryptere data
\item Kryptografi benytter algoritmer som sammen med n�gler giver en
  ciffertekst - der kun kan l�ses ved hj�lp af den tilh�rende n�gle
\end{list1}

\slide{Public key kryptografi - 1}

\hlkimage{18cm}{images/crypto-public-key.pdf}

\begin{list1}
\item privat-n�gle kryptografi (eksempelvis AES) benyttes den samme
  n�gle til kryptering og dekryptering 
\item offentlig-n�gle kryptografi (eksempelvis RSA) benytter to
  separate n�gler til kryptering og dekryptering
\end{list1}

\slide{Public key kryptografi - 2}

\hlkimage{18cm}{images/crypto-public-key-2.pdf}

\begin{list1}

\item offentlig-n�gle kryptografi (eksempelvis RSA) bruger den private
  n�gle til at dekryptere
\item man kan ligeledes bruge offentlig-n�gle kryptografi til at
  signere dokumenter - som s� verificeres med den offentlige n�gle
\end{list1}


\slide{Kryptografiske principper}

\begin{list1}
\item Algoritmerne er kendte
\item N�glerne er hemmelige
\item N�gler har en vis levetid - de skal skiftes ofte
\item Et successfuldt angreb p� en krypto-algoritme er enhver genvej
  som kr�ver mindre arbejde end en gennemgang af alle n�glerne 
\item Nye algoritmer, programmer, protokoller m.v. skal gennemg�s n�je!
\item Se evt. Snake Oil Warning Signs:
Encryption Software to Avoid 
\link{http://www.interhack.net/people/cmcurtin/snake-oil-faq.html}
\end{list1}

\slide{DES, Triple DES og AES}

\hlkimage{15cm}{images/AES_head.png}

\begin{list1}
\item DES kryptering baseret p� den IBM udviklede Lucifer algoritme
  har v�ret benyttet gennem mange �r. 
\item Der er vedtaget en ny standard algoritme Advanced Encryption
  Standard (AES) som afl�ser Data Encryption Standard (DES)
\item Algoritmen hedder Rijndael og er udviklet
af Joan Daemen og Vincent Rijmen.
%\item \emph{Rijndael is available for free. You can use it for
%whatever purposes  you want, irrespective of whether
%it is accepted as AES or not.}

\item Kilde:
\link{http://csrc.nist.gov/encryption/aes/}\\
\href{http://www.esat.kuleuven.ac.be/~rijmen/rijndael/}
{http://www.esat.kuleuven.ac.be/\~{}rijmen/rijndael/}
\end{list1}


\slide{Form�let med kryptering}

\vskip 3 cm
\centerline{\hlkbig kryptering er den eneste m�de at sikre:}
\vskip 3 cm
\centerline{\hlkbig fortrolighed}
\vskip 3 cm
\centerline{\hlkbig autenticitet / integritet}


\slide{e-mail og forbindelser}

\begin{list1}
\item Kryptering af e-mail
\begin{list2}
\item Pretty Good Privacy - Phil Zimmermann
\item PGP = mail sikkerhed
\end{list2}
\item Kryptering af sessioner SSL/TLS
\begin{list2}
\item Secure Sockets Layer SSL / Transport Layer Services TLS
\item krypterer data der sendes mellem webservere og klienter
\item SSL kan bruges generelt til mange typer sessioner, eksempelvis
  POP3S, IMAPS, SSH m.fl.
\end{list2}
\vskip 1 cm 
\item Sender I kreditkortnummeret til en webserver der k�rer uden https?
\end{list1}

\slide{MD5 message digest funktion}

\hlkimage{16cm}{images/message-digest-1.pdf}

\begin{list1}
\item HASH algoritmer giver en unik v�rdi baseret p� input
%\item output fra algoritmerne kaldes ogs� message digest
%\item MD5 er et eksempel p� en meget brugt algoritme
%\item MD5 algoritmen har f�lgende egenskaber:
%  \begin{list2}
%  \item output er 128-bit "fingerprint" uanset l�ngden af input
\item v�rdien �ndres radikalt selv ved sm� �ndringer i input
%  \end{list2}
\item MD5 er blandt andet beskrevet i RFC-1321: The MD5 Message-Digest
  Algorithm 
%\item Algoritmen MD5 er baseret p� MD4, begge udviklet af Ronald
%  L. Rivest kendt fra blandt andet RSA Data Security, Inc
\item B�de MD5 og SHA-1 unders�ges n�je og der er fundet kollisioner
  som kan p�virke vores brug i fremtiden - \emph{stay tuned}
\end{list1} 

%%% Local Variables: 
%%% mode: latex
%%% TeX-master: "tcpip-security"
%%% End: 
