\documentclass[20pt,landscape,a4paper,footrule]{foils}
\usepackage{solido-network-slides}
\usepackage{simpsons}

\begin{document}
\selectlanguage{danish}
\mytitlepage{Introduktion til IT-sikkerhed idag\\{\small December 2014}}


\vskip 2cm
\centerline{\footnotesize Slides are available as PDF, kramshoej@Github}

\slide{Goals of today}

\Bart

\begin{list1}
\item Update on trends in information security and internet security
\item Offer input to what things to look into
\vskip 2cm
\item I will try to limit myself to things from 2014
\item Hodge-podge of security related things - inspiration 
\item Please give feedback and join me in discussions, dialogue \smiley
\end{list1}

\slide{Plan for today}

\hlkimage{10cm}{Shaking-hands_web.jpg}

\begin{list1}
\item Kl 09:30-11:00
\item Less presentation, more talk 
\item Less me talking (only) and more 2.0 social media interaction
\end{list1}


\slide{Part I: Paranoia defined}

\hlkimage{15cm}{paranoia-definition.png}

Source: google paranoia definition

\slide{Face reality}

From the definition:
\begin{quote}
suspicion and mistrust of people or their actions without {\bf evidence or justification}
{\bf the global paranoia about hackers and viruses}
\end{quote}

\begin{list1}
\item It is not paranoia when:
\begin{list2}
\item Criminals sell your credit card information and identity theft
\item Trade infected computers like a commodity
\item Hackers break in and steal information
\item Governments write laws that allows them to introduce back-doors - and use these
\item Governments do blanket surveillance of their population
\item Governments implement censorship, threaten citizens and journalist
\end{list2}
\end{list1}

\vskip 1cm
\centerline{You are not paranoid when there are people actively attacking you!}

\slide{What is data?}
\hlkimage{10cm}{Linus3-04041999.jpg}

\begin{list1}
\item Personal data you dont want to loose:
\begin{list2}
\item Wedding pictures
\item Pictures of your children
\item Sextapes
\item Personal finances
\end{list2}
\end{list1}

Source: picture of my son less than 24 hours old - precious!


\slide{Hacker types anno 2008}
\hlkimage{10cm}{lisbeth-salander.jpeg}

\begin{list1}
\item Lisbeth Salander from the Stieg Larsson's award-winning Millennium series
does research about people using hacking as a method to gain access
\item How can you find information about people?
\end{list1}


\slide{Google for it}

\hlkimage{16cm}{images/googledorks-1.pdf}

\begin{list1}
\item Google as a hacker tools?
\item Concept named googledorks when google indexes information not supposed to be public
\link{http://www.hackersforcharity.org/ghdb/}
\end{list1}


\slide{Attack overview}

\hlkimage{22cm}{sicherheitstacho.png}

{\small\link{http://www.sicherheitstacho.eu/?lang=en}}


\slide{Sony Hack}

\begin{list1}
\item  nearly 40GB of data hacked and leaked from Sony Pictures Entertainment's (SPE) internal computer systems.
\item Passwords - complete infrastructure and all passwords must be reset
\item Payroll information, financial information about producs and earnings
\item Information about planned products and strategy is out
\item Bad passwords allow access to critical assets
\item Leaked unreleased titles Annie, Mr. Turner, Still Alice, and To Write Love On Her Arms, as well as World War II drama Fury.
\end{list1}


Source:\\
\link{http://techcrunch.com/2014/11/30/five-sony-pictures-movie-screeners-leaked-after-hacking/}
\link{http://mashable.com/2014/12/04/sony-hack-data-details/}

\slide{Movie:}

\hlkimage{16cm}{youtube-bic-lock.png}

\begin{list1}
\item Just search for: kryptonite lock bic pen
\item \link{https://www.youtube.com/watch?v=LahDQ2ZQ3e0}
\end{list1}


\slide{Heartbleed CVE-2014-0160}

\hlkimage{22cm}{heartbleed-com.png}

Source: \link{http://heartbleed.com/}


\slide{Heartbleed hacking}

\begin{alltt}\footnotesize
  06b0: 2D 63 61 63 68 65 0D 0A 43 61 63 68 65 2D 43 6F  -cache..Cache-Co
  06c0: 6E 74 72 6F 6C 3A 20 6E 6F 2D 63 61 63 68 65 0D  ntrol: no-cache.
  06d0: 0A 0D 0A 61 63 74 69 6F 6E 3D 67 63 5F 69 6E 73  ...action=gc_ins
  06e0: 65 72 74 5F 6F 72 64 65 72 26 62 69 6C 6C 6E 6F  ert_order&billno
  06f0: 3D 50 5A 4B 31 31 30 31 26 70 61 79 6D 65 6E 74  =PZK1101&payment
  0700: 5F 69 64 3D 31 26 63 61 72 64 5F 6E 75 6D 62 65  _id=1&{\bf card_numbe}
  0710: XX XX XX XX XX XX XX XX XX XX XX XX XX XX XX XX  {\bf r=4060xxxx413xxx}
  0720: 39 36 26 63 61 72 64 5F 65 78 70 5F 6D 6F 6E 74  {\bf 96&card_exp_mont}
  0730: 68 3D 30 32 26 63 61 72 64 5F 65 78 70 5F 79 65  {\bf h=02&card_exp_ye}
  0740: 61 72 3D 31 37 26 63 61 72 64 5F 63 76 6E 3D 31  {\bf ar=17&card_cvn=1}
  0750: 30 39 F8 6C 1B E5 72 CA 61 4D 06 4E B3 54 BC DA  {\bf 09}.l..r.aM.N.T..
\end{alltt}

\begin{list2}
\item Obtained using Heartbleed proof of concepts - Gave full credit card details
\item "can XXX be exploited" - yes, clearly! PoCs ARE needed\\
without PoCs even Akamai wouldn't have repaired completely!
\item The internet was ALMOST fooled into thinking getting private keys from Heartbleed was not possible - scary indeed.
\end{list2}

\slide{Malware charateristics}

\begin{list1}
\item Malware is advanced and sophisticated
\item Modular frameworks
\item Use strong cryptography to hide 
\item Use 0-day exploits - unknown to others
\item Use rootkits to stay under radar and avoid anti-virus
\item Mutate and change to avoid detection
\item In general less noisy
\end{list1}

\slide{Botnets and malware sold with support}

\hlkimage{21cm}{dagens-tilbud-trojanere.pdf}

\begin{list1}
\item Malware programmers act like software houses
\item "Buy this version with updates and support"
\item Rent a bot net with 100.000 computers
\end{list1}


\slide{Phishing - Receipt for Your Payment to mark561@bt....com}
\hlkimage{21cm}{paypal-phish.png}

\centerline{\bf\LARGE Kan du selv genkende Phishing}
\centerline{\bf\LARGE kan brugere}

\slide{Risk management defined}

\hlkimage{27cm}{shon-harris-risk-management.png}

Source: Shon Harris \emph{CISSP All-in-One Exam Guide}

\slide{TwoFactor authentication: Example Duosecurity}

\hlkimage{12cm}{duosecurity-overview.png}
Video
\link{https://www.duosecurity.com/duo-push}

\link{https://www.duosecurity.com/}

\slide{Good security}

\hlkimage{15cm}{god-sikkerhed.pdf}

\begin{list1}
\item You always have limited resources for protection - use them as best as possible
\end{list1}

\slide{First advice}

\begin{list1}
\item Use technology
\item Learn the technology - read the freaking manual
\item Think about the data you have, upload, facebook license?! WTF!
\item Think about the data you create - nude pictures taken, where will they show up?
\begin{list2}
\item Turn off features you don't use
\item Turn off network connections when not in use
\item Update software and applications
\item Turn on encryption: IMAP{\bf S}, POP3{\bf S},
  HTTP{\bf S} also for data at rest, full disk encryption, tablet encryption
\item Lock devices automatically when not used for 10 minutes
\item Dont trust fancy logins like fingerprint scanner or face recognition on cheap devices
\end{list2}
\end{list1}


\slide{Theft - kindergarten and airports}

\begin{list1}
\item Many parents are in a hurry when they are picking up their kids
\item Many people can easily be distracted around crowds
\item Many people let their laptops stay out in the open - even at conferences
\item ... making theft likely/easy
\vskip 1 cm
\item Stolen for the value of the hardware - or for the data?
\item Industrial espionage, economic espionage or corporate espionage is real
\end{list1}

\centerline{Security breaches happens any day of the week}

\slide{Offline Backup}

\vskip 3cm
\centerline{\LARGE \bf Kom igang!}

\begin{list2}
\item Write a backup to DVD - most laptops today can do that
\item Save stuff in the cloud, examples Dropbox, Google Drive
\item Save data to external harddrive, cheap today
\end{list2}

Sad story Mat Honan epic hacking :-(\\ {\small\link{http://www.wired.com/gadgetlab/2012/08/apple-amazon-mat-honan-hacking/all/}}


\slide{How can we protect? Generic advice}

Recommendations \hlkrightimage{8cm}{Encrypt_all_the_things.png}
\begin{list2}
\item Lock your devices, phones, tables and computers
\item Update software and apps
\item Do NOT use the same password everywhere
\item Watch out when using open wifi-networks
\item Multiple browsers: one for Facebook, and separate for home banking apps?
\item Multiple laptops? One for private data, one for work?
\item Think of the data you produce, why do people take naked pictures and SnapChat them?
\item Use pseudonyms and aliases, do not use your real name everywhere
\item Enable encryption: IMAP{\bf S}, POP3{\bf S},
  HTTP{\bf S} \\
\end{list2}


\slide{Multiple browsers}

\hlkimage{20cm}{multi-browser-strategy.png}

\begin{list2}
\item Strict Security settings in the general browser, Firefox or Chrome?
\item More lax security settings for "trusted sites" - like home banking
\item Security plugins like HTTPS Everywhere and NoScripts for generic browsing
\end{list2}

\slide{HTTPS Everywhere}

\hlkimage{5cm}{HTTPS_Everywhere_new_logo.jpg}
\begin{quote}
HTTPS Everywhere is a Firefox extension produced as a collaboration between The Tor Project and the Electronic Frontier Foundation. It encrypts your communications with a number of major websites.
\end{quote}

\centerline{\link{http://www.eff.org/https-everywhere}}



\slide{Secure your mobile}

\hlkimage{20cm}{the-guardian-project.pdf}

\centerline{Dont forget your mobile platforms \link{https://guardianproject.info/}}


\slide{Balanced security}

\hlkimage{21cm}{afbalanceret-sikkerhed.pdf}

\begin{list1}
\item Better to have the same level of security
\item If you have bad security in some part - guess where attackers will end up
\item Hackers are not required to take the hardest path into the network
\item Realize there is no such thing as 100\% security 
\end{list1}



\slide{How to become secure} 

\hlkimage{8cm}{MangaRamblo.jpg}

\begin{list1}
\item Dont use computers at all, data about you is still processed by computers :-(
\item Dont use a single device for all types of data
\item Dont use a single server for all types of data, mail server != web server
\item Configure systems to be secure by default, or change defaults
\item Use secure protocols and VPN solutions
\end{list1}



\slide{Focus for the near future}

\begin{list1}
\item Walk through your infrastructure\\
get a detailed view of data, flows, protocols, bandwidth, ports and services
\item Make sure your organization is also in control, know your vendors
\item Create a list of critical phone numbers and contacts, enter it in your phone
\item Get control of BYOD Bring Your Own Devices
\end{list1}

\slide{Surveillance Self-Defense EFF}

\hlkimage{10cm}{ssd-eff-logo.png}

\begin{quote}
\centerline{Tips, Tools and How-tos For Safer Online Communications}

Modern technology has given the powerful new abilities to eavesdrop and collect data on innocent people. Surveillance Self-Defense is EFF's guide to defending yourself and your friends from surveillance by using secure technology and developing careful practices.
\end{quote}

Source: \link{https://ssd.eff.org/}


\slide{Sources for information}

\hlkimage{8cm}{twitter-security-feed.png}

\begin{list1}
\item Twitter has replaced RSS for me
\item Email lists are still a good source of data
\item Favourite Security Diary from Internet Storm Center\\
 \link{http://isc.sans.edu/index.html}\\
\link{https://isc.sans.edu/diaryarchive.html?year=2013&month=4}
\end{list1}


\myquestionspage





\end{document}
