\documentclass[20pt,landscape,a4paper,footrule]{foils}
\usepackage{solido-network-slides}




% noter
% http://www.darkreading.com/perimeter/how-malware-bypasses-our-most-advanced-security-measures/a/d-id/1318974?_mc=RSS_DR_EDT



% more:
% Gemalto hack

% Gray Hat Hacking bogen?


\begin{document}

\selectlanguage{danish}
\mytitlepage{Tendenser i sikkerhed\\{\small March 2015}}


\vskip 2cm
\centerline{\tiny Slides are available as PDF, kramshoej@Github}


\slide{Goals of today}
\hlkimage{12cm}{government-doing-nothing-wrong.jpg}

\begin{list1}
\item Update on trends in information security and internet security
\item Offer input to what things to look into
%\vskip 2cm
\item I will try to limit myself to things from 2015
\item Hodge-podge of security related things - inspiration
\item Please give feedback and join me in discussions, dialogue \smiley
\end{list1}

\slide{Plan for today}

\hlkimage{10cm}{Shaking-hands_web.jpg}

\begin{list1}
\item Kl 17:00-20:30 and some breaks
\item Less presentation, more talk
\item Less me talking (only) and more 2.0 social media interaction
\end{list1}

\slide{Generic advice}

Recommendations \hlkrightimage{8cm}{Encrypt_all_the_things.png}
\begin{list2}
\item Lock your devices, phones, tables and computers
\item Update software and apps
\item Do NOT use the same password everywhere
\item Watch out when using open wifi-networks
\item Multiple browsers: one for Facebook, one for banking apps?
\item Multiple laptops? One for private data, one for work?
\item Think of the data you produce - where is it stored
\item Use pseudonyms and aliases, do not use your real name everywhere
\item Enable encryption: IMAP{\bf S}, POP3{\bf S},
  HTTP{\bf S} and full disk encryption
\item Use Tor \link{http://torproject.org/}
\end{list2}


\slide{Democracy now: Why do we bother?}

\hlkimage{22cm}{jacob-appelbaum.png}
{\large
In a democracy we need the citizens with freedom that can act without constant surveillance

Democracy requires that we can actively select which personal data to give up and to whom
}


\vskip 1 cm
\centerline{\large Cryptography is peaceful protest against blanket surveillance}

\vskip 2cm
PS Wrote this slide loooong before Copenhagen shooting in 2015, still stand by it!

\slide{Why think of security?}

\hlkimage{8cm}{1984-not-instruction-manual.jpg}


\begin{quote}
	Privacy is necessary for an open society in the electronic age. Privacy is not secrecy. A private matter is something one doesn't want the whole world to know, but a secret matter is something one doesn't want anybody to know. Privacy is the power to selectively reveal oneself to the world. ~A Cypherpunk's Manifesto by Eric Hughes, 1993
\end{quote}

Copied from \link{https://cryptoparty.org/wiki/CryptoParty}



\slide{Multiple browsers}

\hlkimage{20cm}{multi-browser-strategy.png}

\begin{list2}
\item Strict Security settings in the general browser, Firefox or Chrome?
\item More lax security settings for "trusted sites" - like home banking
\item Security plugins like HTTPS Everywhere and others for generic browsing
\end{list2}

\slide{HTTPS Everywhere}

\hlkimage{5cm}{HTTPS_Everywhere_new_logo.jpg}
\begin{quote}
HTTPS Everywhere is a Firefox extension produced as a collaboration between The Tor Project and the Electronic Frontier Foundation. It encrypts your communications with a number of major websites.
\end{quote}

\centerline{\link{http://www.eff.org/https-everywhere}}

\begin{list2}
\item Also check out their other projects
\item Privacy Badger \link{https://www.eff.org/privacybadger}
\item Surveillance Self-Defense is EFF's guide to defending yourself and your friends\\
\link{https://ssd.eff.org/}
\end{list2}

\slide{Add-ons Galore}

\hlkimage{20cm}{tykling-firefox-addons.png}

You can find lots of privacy add-ons, above is a collection by @tykling from Twitter
{\tiny
\link{https://addons.mozilla.org/en-US/firefox/collections/tykling/tykling-firefox-addons/}\\
\link{https://www.denfri.dk/2015/03/5-firefox-tilfoejelser-der-kan-redde-dit-privatliv/}
}

% Censurafdelingen
% Twitter blocking, og husk antal Tor brugere i Turkiet vokser voldsomt


\slide{Face reality}

\begin{list2}
\item Criminals sell your credit card information and identity theft
\item Trade infected computers like a commodity
\item Governments write laws that allows them to introduce back-doors - and use these
\item Governments do blanket surveillance of their population
\item Governments implement censorship, threaten citizens and journalist
\item Governments will introduce back-doors in products we use
\item Danish police and TAX authorities have the legals means, see \emph{Rockerloven}
\end{list2}

\vskip 1cm
\centerline{You are not paranoid when there are people actively attacking you!}


\slide{Overview of malware bypass today}

Quote:
The primary malware installation, sometimes referred as an infection, can be achieved using several attack vectors.
The goal is always to run malicious code. Some of the most common attack vectors are:

\begin{list2}
\item 1. Browser-based social engineering: where a user is tricked into clicking on a legitimate-looking URL which in turn triggers code execution using browser or browser-plugin vulnerabilities in Java and Flash. More advanced attacks can hide in legitimate traffic without requiring any user-interaction. These are commonly referred to as drive-by downloads.
\item 2. Email-based social engineering and spear phishing: where a user receives an email that contains a hidden or visible binary, which executes when the user clicks on it.
\item 3. Credential theft: when guessed or stolen credentials are used to access a remote machine and execute (malicious) code, such as installing a backdoor.
\end{list2}

{\small Source: Great summary article by Alon Nafta, senior security engineer at SentinelOne\\
How Malware Bypasses Our Most Advanced Security Measures, february 2015}\\
{\tiny\link{http://www.darkreading.com/perimeter/how-malware-bypasses-our-most-advanced-security-measures/a/d-id/1318974?_mc=RSS_DR_EDT}}



\slide{Overview of malware bypass today, continued}

Evasion techniques
To evade detection, during and after installation, malware uses five primary techniques.
\begin{list1}
\item 1. Wrapping. This process attaches the malicious payload (the installer or the malware itself) to a legitimate file.
... IceFog is a well-known malware commonly wrapped with a legitimate-looking CleanMyMac application and used to target OS X users. On the Windows platform, OnionDuke has been used with legitimate Adobe installers shared over Tor networks to infect machines.
\item 2. Obfuscation. This involves modifying high level or binary code it in a way that does not affect its functionality, but completely changes its binary signature. ... Malware authors have adopted the technique to bypass antivirus engines and impair manual security research. ...
\end{list1}

Source: How Malware Bypasses Our Most Advanced Security Measures\\
{\tiny\link{http://www.darkreading.com/perimeter/how-malware-bypasses-our-most-advanced-security-measures/a/d-id/1318974?_mc=RSS_DR_EDT}}

\slide{Overview of malware bypass today, continued}

\begin{list1}
\item 3. Packers. These software tools are used to compress and encode binary files, which is another form of obfuscation.... These techniques are extremely effective at circumventing static signature engines.
\item 4. Anti-debugging. Like obfuscation, anti-bugging was originally created by software developers to protect commercial code from reverse-engineering. Anti-debugging can prevent a binary from being analyzed in an emulated environments such as virtual machines, security sandbox, and others. ...
\item 5.  Targeting. This technique is implemented when malware is designed to attack a specific type of system (e.g. Windows XP SP 3), application (e.g. Internet Explorer 10) and/or configuration (e.g. detecting a machine not running VMWare tools, which is often a telltale sign for usage of virtualization). ...
\end{list1}

Source: How Malware Bypasses Our Most Advanced Security Measures\\
{\tiny\link{http://www.darkreading.com/perimeter/how-malware-bypasses-our-most-advanced-security-measures/a/d-id/1318974?_mc=RSS_DR_EDT}}



\slide{Most vulnerable operating systems in 2014}

\hlkimage{18cm}{GFI-vulns-2014-OS-chart.jpg}

\begin{quote}
An average of 19 vulnerabilities per day were reported in 2014, according to the data from the National Vulnerability Database (NVD).
\end{quote}

Source:\\
{\footnotesize
\link{http://www.gfi.com/blog/most-vulnerable-operating-systems-and-applications-in-2014/}}


\slide{Most vulnerable applications in 2014}

\hlkimage{16cm}{gfi-vulns-application-chart.jpg}

\begin{quote}\small
Not surprisingly at all, web browsers continue to have the most security vulnerabilities because they are a popular gateway to access a server and to spread malware on the clients.
\end{quote}

Source:\\
{\footnotesize
\link{http://www.gfi.com/blog/most-vulnerable-operating-systems-and-applications-in-2014/}}


\slide{Release of vuln information without patch}

\begin{list1}
\item Google project Zero
\item Follow a
"90-day disclosure deadline statement... which in this instance has passed."
\item Released Zero-day information about Microsoft and Apple OS X vulnerabilities
\item MS patched some in \emph{first Patch Tuesday of 2015, which came out on Jan. 13.}
\end{list1}

Sources:\\
{\tiny
\link{http://googleonlinesecurity.blogspot.fr/2014/07/announcing-project-zero.html}\\
\link{http://searchsecurity.techtarget.com/news/2240238448/Googles-Project-Zero-reveals-another-Windows-zero-day-vulnerability}\\
\link{http://www.engadget.com/2015/01/02/google-posts-unpatched-microsoft-bug/}\\
\link{http://www.eweek.com/security/google-project-zero-continues-its-microsoft-zero-day-assault.html}\\
\link{http://www.zdnet.com/article/googles-project-zero-reveals-three-apple-os-x-zero-day-vulnerabilities/}
}

Trend with more vulnerabilities per day, and\\
even big vendors cannot react quickly enough


\slide{Samba remote code execution}


\begin{alltt}\small
  ===========================================================
  == Subject:     Unexpected code execution in smbd.
  ==
  == CVE ID#:     CVE-2015-0240
  ==
  == Versions:    Samba 3.5.0 to 4.2.0rc4
  ==
  == Summary:     Unauthenticated code execution attack on
  ==		smbd file services.
  ==
  ===========================================================
\end{alltt}

\centerline{Great, even our old tools still has multiple bugs}

Source:\\
\link{https://www.samba.org/samba/security/CVE-2015-0240}


\slide{DNS attacks, February 2015 - ongoing for +10 years!}

\hlkimage{18cm}{krebs-lenovo-google-dns-hack.png}
\begin{list1}
%\item DNS is the Domain Name System, \link{https://en.wikipedia.org/wiki/Dns}
\item DNS insecurity has huge impact on your security!
\item Most are denial of service, by may create Mitm or confidentiality concerns
\item Select DNS providers with care
\end{list1}


Sources:\\
{\tiny
\link{http://krebsonsecurity.com/2015/02/webnic-registrar-blamed-for-hijack-of-lenovo-google-domains/}\\
\link{http://www.version2.dk/artikel/google-og-lenovo-defaced-som-foelge-af-overset-sikkerhedsproblemstilling-91295}}


\slide{DNSSEC trigger}

\hlkimage{13cm}{dnssec-trigger.png}

Lots of DNSSEC tools, I recommend DNSSEC-trigger a local name server for your laptop

\begin{list2}
\item DNSSEC Validator for firefox\\ \link{https://addons.mozilla.org/en-us/firefox/addon/dnssec-validator/}
\item OARC tools \link{https://www.dns-oarc.net/oarc/services/odvr}
\item \link{http://www.nlnetlabs.nl/projects/dnssec-trigger/}
\end{list2}

\slide{DNSSEC get started now}

\hlkimage{17cm}{cz-nic-dnssec-tlsa-validator.png}

\begin{quote}
"TLSA records store hashes of remote server TLS/SSL certificates. The authenticity of a TLS/SSL certificate for a domain name is verified by DANE protocol (RFC 6698). DNSSEC and TLSA validation results are displayer by using several icons."
\end{quote}


\slide{DNSSEC and DANE}

\begin{quote}
"Objective:

Specify mechanisms and techniques that allow Internet applications to
establish cryptographically secured communications by using information
distributed through DNSSEC for discovering and authenticating public
keys which are associated with a service located at a domain name."
\end{quote}

DNS-based Authentication of Named Entities (dane)

Old news - we have been talking about DANE for years!
\link{https://datatracker.ietf.org/wg/dane/charter/}\\
{\footnotesize \link{http://googleonlinesecurity.blogspot.dk/2011/04/improving-ssl-certificate-security.html}}

Trend, we have HUGE populations of older servers, systems, people,
that are just not taking to new technologies - even if it would solve a lot of problems

Case in point: IPv6

\slide{But DNSSEC is bad! DNS Amplification?!}

Yes, DNSSEC has larger responses, used for amplification DDoS attacks.

\begin{quote}
"This is the official homepage for PacketQ, a simple tool to make SQL-queries against PCAP-files, making packet analysis and building statistics simple and quick. PacketQ was previously known as DNS2db but was renamed in 2011 when it was rebuilt and could handle protocols other than DNS among other things.

Look how easy it's to count DNS-packets in a PCAP-file."
\end{quote}

\begin{alltt}
\small
# packetq -s "select count(*) as count_dns from dns" packets.pcap
[ \{ "table_name": "result",
      "head": [
      \{ "name": "count_dns","type": "int" \} ],   {\bf "data": [ [95501] ] \}} ]
\end{alltt}

\link{https://github.com/dotse/packetq/wiki}

Trend, any problem has a Github repo with parts of the solution \smiley

\slide{Example, Using tools similar to PacketQ}

\hlkimage{18cm}{using-packetq.png}

Are you using your brain and existing tools? Building own specialised tools?\\
Discussion: bridging the gaps between Devops and Security? Good thing, easy?

{\footnotesize
\link{http://securityblog.switch.ch/2013/01/22/using-packetq/}\\
\link{http://jpmens.net/2013/05/27/server-agnostic-logging-of-dns-queries-responses/}
}

\slide{Storing query logs, old school or needed?}

\hlkimage{7cm}{bro-sample-ssl-scripts.png}

Looking at DNS PacketQ it was an Older link, but thinking the time is now for doing:

\begin{list2}
\item Netflow session logging, full 1:1 - NFSen, Suricata Flow mode
\item DNS query logs, keep it for at least a week? - with DSC and PacketQ
\item SSL/TLS full logs over sessions, certs, keys - with Bro/Suricata\\
\link{https://www.bro.org/sphinx-git/script-reference/scripts.html}
\item Log and search with Elasticsearch?\\
\link{https://www.elastic.co/guide/en/elasticsearch/guide/current/index.html}
%\item Moloch \link{https://github.com/aol/moloch}
\end{list2}

%\centerline{Why go to this extreme, storing information about past sessions?}

\slide{February 2015: Finding infected sources}

\begin{qoute}
"We were contacted by a client to help with their incident response in tracking down an
infection on a clients machine with the new CTB-Locker ransomware (Curve-Tor-Bitcoin Locker)
aka Critroni which had no signatures available at the time of infection for this variant.

LANGuardian includes a file share activity monitoring module which provided a very
detailed forensic analysis of the ransomware and the paths it had taken in order to
encrypt the clients system and also the fileserver in which it was connected to, the
initial infection came from the opening of an attachment in an e-mail."
\end{quote}

Source:
{\tiny\link{http://www.netfort.com/support-team-stories-detecting-the-source-of-ransomware/}}

\slide{Why?, because things like Superfish February 2015}

\centerline{Yet another SSL/TLS related problem}

\hlkimage{15cm}{robert-graham-superfish-cert.png}

Lenovo laptops included Adware, which did SSL/TLS Man in the Middle on connections.
They had a root certificate installed on the Windows operating system, WTF!

{\footnotesize Sources:\\
\link{https://en.wikipedia.org/wiki/Superfish}\\
\link{http://blog.erratasec.com/2015/02/extracting-superfish-certificate.html}\\
\link{http://www.version2.dk/blog/kibana4-superfish-og-emergingthreats-81610}\\
}{\tiny\link{https://www.eff.org/deeplinks/2015/02/further-evidence-lenovo-breaking-https-security-its-laptops}
}


\slide{FREAK March 2015}


\begin{quote}
"A group of cryptographers at INRIA, Microsoft Research and IMDEA have discovered some serious vulnerabilities in OpenSSL (e.g., Android) clients and Apple TLS/SSL clients (e.g., Safari) that allow a 'man in the middle attacker' to downgrade connections from 'strong' RSA to 'export-grade' RSA. These attacks are real and exploitable against a shocking number of websites -- including government websites. Patch soon and be careful."
\end{quote}

Source: Matthew Green, cryptographer and research professor at Johns Hopkins Univ\\
{\footnotesize\link{http://blog.cryptographyengineering.com/2015/03/attack-of-week-freak-or-factoring-nsa.html}
\link{https://www.smacktls.com/} \link{https://freakattack.com/}
}


OpenSSL, LibreSSL, Apple SSL flaw exit exit exit!, Android SSL, certs certs!!!111, SSLv3, Heartbleed, MS TLS

\vskip 1cm
\centerline{F this I'm going out drinking beer *drops mic*}



\vskip 1cm
PS From now on its TLS! Not SSL anymore, any SSLv2, SSLv3 is old and vulnerable


\slide{Bettercrypto.org pretty good advise}

SSL settings for nginx
\hlkimage{15cm}{bettercrypto-nginx.png}

Overview
\begin{quote}
"This whitepaper arose out of the need for system administrators to have an updated,
solid, well researched and thought-through guide for configuring SSL, PGP, SSH and
other cryptographic tools in the post-Snowden age. ... This guide is specifically
written for these system administrators."
\end{quote}

\link{https://bettercrypto.org/}


\slide{New tools}

\begin{list1}
\item Hacking is fun - learn a lot
\item Kibana 4
\item Burp Suite Professional
\item Security Onion and Suricata updates
\item Tor Browser 4.0.4
\item Tails 1.3
%\item \link{http://www.kali.org/} Kali Linux Rebirth of BackTrack
%\item \link{http://www.arachni-scanner.com/}\\- been on my todolist for too long, try it maybe?
\item Wordlists \verb+->+ change your passwords frequently!
\item Decrypt SSL sessions while debugging
\end{list1}



% Suricata, Logstash, Elasticsearch, D3JShttp://d3js.org/
\slide{Kibana 4 february 2015}

\hlkimage{14cm}{kibanascreenshothomepagebannerbigger.jpg}

\centerline{Highly recommended for a lot of data visualisation}

Source:
\link{https://www.elastic.co/products/kibana}





\slide{Hacker tools Burp}


\hlkimage{20cm}{burp-prssi.png}

\begin{list1}

\item Do it in your own network - your systems, keep it legal
\item Burp is a highly recommended commercial Web proxy EUR 275/user/year ~2.000DKK
\item Pro version includes scanner and spidering functionality
\end{list1}


\slide{February and March 2015, Security Onion updates}

\begin{list1}
\item Security Onion 12.04.5.1 ISO image now available\\
We have a new Security Onion 12.04.5.1 ISO image now available that contains all the latest Ubuntu and Security Onion updates as of February 5, 2015!
\item Suricata IDS engine 2.0.7 updated packages for SO released
\end{list1}

\centerline{Learn NSM with Security Onion today - its free}

Source:\\
\link{http://blog.securityonion.net/2015/02/security-onion-120451-iso-image-now.html}\\
\link{http://blog.securityonion.net/2015/03/suricata-207.html}

\slide{February 2015: Tor Browser 4.0.4 Released}

\hlkimage{10cm}{reddit-tor-donate-2015.jpg}

\begin{quote}
"Tor - a privacy oriented encrypted anonymizing service, has announced the launch of its next version of Tor Browser Bundle, i.e. Tor version 4.0.4, mostly supposed to improve the built-in utilities, privacy and security of online users on the Internet."
	\end{quote}


Source:
\link{http://thehackernews.com/2015/02/tor-browser-download.html}\\
\link{https://www.torproject.org/}

also new Tails 1.3 was released with bitcoin wallet\\
\link{http://thehackernews.com/2015/02/tails-tor-privacy-tools.html}\\
\link{https://tails.boum.org/download/index.en.html}



%\slide{more tools here }
%more tools here

\slide{Feb 2015 Today I Am Releasing Ten Million Passwords}

\begin{quote}
"  Why the FBI Shouldn't Arrest Me\\
  Although researchers typically only release passwords, I am releasing usernames with the passwords. Analysis of usernames with passwords is an area that has been greatly neglected and can provide as much insight as studying passwords alone. Most researchers are afraid to publish usernames and passwords together because combined they become an authentication feature. If simply linking to already released authentication features in a private IRC channel was considered trafficking, surely the FBI would consider releasing the actual data to the public a crime."
\end{quote}

Source: Mark Burnett's Blog\\
\link{https://xato.net/passwords/ten-million-passwords/}


\slide{Feb 2015 Decrypting TLS Browser Traffic With Wireshark }

\hlkimage{20cm}{wireshark-decrypt-ssl.png}

\begin{list1}
\item  Firefox and Chrome both support logging the symmetric session key used to encrypt TLS traffic to a file
\item Wireshark can read this file - and decrypt sessions - Nifty trick
\end{list1}


Source: great blog article about the features used\\
{\tiny\link{https://jimshaver.net/2015/02/11/decrypting-tls-browser-traffic-with-wireshark-the-easy-way/}}


\slide{DDoS in 2014}

\hlkimage{20cm}{arbor-2014-ntp.png}


\slide{DDoS in 2015}

\hlkimage{20cm}{arbor-networks-10-year-ddos.png}

Source:\\
Arbor Networks: Worldwide Infrastructure Security Report, Volume X January 2015


\slide{Detecting DDoS}

\hlkimage{15cm}{nfsen-ddos-profile-1.png}

We created a DDoS profile with the common types.

We can ask RDDtools about max, average etc.
\begin{alltt}\small
rrdtool graph x -s -24h DEF:v=DDoS/mx-cph-01.rrd:packets:MAX VDEF:vm=v,MAXIMUM PRINT:vm:%.lf
\end{alltt}




\slide{DDoS traffic before filtering}
\hlkimage{26cm}{ddos-before-filtering}

\centerline{Only two links shown, at least 3Gbit incoming for this single IP}

\slide{DDoS traffic after filtering}
\hlkimage{18cm}{ddos-after-filtering}
\centerline{Link toward server (next level firewall actually) about ~350Mbit outgoing}

\begin{list1}
\item Problem: We receive unauthenticated chaotic traffic

\item Solution: Discard early, discard on edge, reduce noise

\item Only use CPU resources for potentially real traffic
\end{list1}

\slide{Defense in depth - multiple layers of security}

\hlkimage{23cm}{network-layers-1.pdf}

% Even mckinsey is talking about defense in depth
% http://www.mckinsey.com/insights/business_technology/protecting_the_enterprise_with_cybersecure_it_architecture?cid=other-eml-alt-mip-mck-oth-1503

\slide{Proxy servers - protection services}

\begin{list1}
\item Several big players you need to research before needing them!
\item Arbor Networks sells software solutions for carriers\\
http://www.arbornetworks.com/

\item Prolexic sells DDoS services, DNS and BGP based\\
http://www.prolexic.com/

\item CloudFlare proxy based\\
http://www.cloudflare.com/
\end{list1}

\vskip 2cm
\centerline{Multiple Major Danish ISPs have bought services from the above companies}



\slide{Focus for the near future}

\begin{list2}
\item Walk through your infrastructure\\
get a detailed view of data, flows, protocols, bandwidth, ports and services

\item Create a list of critical phone numbers and contacts, enter it in your phone
\item Automate updates for both clients and servers, goal update everything in hours
\item Learn to run Nmap and Metasploit scripts - identify vulnerable servers
\end{list2}

\vskip 2cm
\centerline{consider the fact we have multiple overlapping critical security incidents now!}

\vskip 2cm
How many incidents can your organisation handle in parallel?

\slide{Near future - spring and summer 2014}

\hlkimage{5cm}{crypto-party-logo.png}

\begin{list1}
\item Document your processes, systems, databases, backup and restore procedures\\
Finish before summer - so you can have vacation, will be needed!
\item Crypto Parties - get them started, keep them going!
\item Conferences: DKNOG, TheCamp this summer, RIPE in May, CCC Summercamp
\end{list1}

\slide{Chaos Communication Camp 2015}

\begin{quote}
{\bf Chaos Communication Camp 2015: Save the date!}
Not just because of tradition, but because we can: There will be a Camp in 2015 again!

After more than three thousand guests at the last Camp in 2011, we expect many outdoor enthusiasts, who like to hack during the day and marvel at the light displays at night and will celebrate the ultimate party in the middle of the festival season with us. Lets just bring all the great experiences from the Congress out on the lawn and do all the experiments too dangerous for the halls of the CCH!
\end{quote}

{\small\link{http://events.ccc.de/2015/02/10/chaos-communication-camp-2015-save-the-date/}}



\slide{Sources for information}

\hlkimage{8cm}{twitter-security-feed.png}

\begin{list1}
\item Twitter has replaced RSS for me
\item Email lists are still a good source of data
%\item Favourite Security Diary from Internet Storm Center\\
% \link{http://isc.sans.edu/index.html}\\
%\link{https://isc.sans.edu/diaryarchive.html?year=2013&month=4}
\end{list1}



\slide{Open Mike night ...}

\vskip 3 cm

\centerline{\Large what did I forget? tells us about your favourites \smiley}

Things I forgot, didn't include: february Gemalto hack, Citizenfour won an oscar in February!

December Thunderbolt hack - thunderstrike

March, Rowhammer, \link{http://www.wired.com/2015/03/google-hack-dram-memory-electric-leaks/}

Samsung TVs listening and watching

DNS censorship, NemID bashing, Apple malware, Android malware, iPhone malware?

Did you notice how a lot of the links in this presentation uses HTTPS - encrypted

\myquestionspage





\end{document}
