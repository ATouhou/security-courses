\documentclass[20pt,landscape,a4paper,footrule]{foils}
%\usepackage{solido-network-slides}
\usepackage{zencurity-slides}


%
% Arrangement:	Penetration testing I - basale pentest metoder og introduktion
% Mål:	Introduktion til penetrationstest.
% Forudsætninger:	Der forventes kendskab til TCP/IP på brugerniveau.
% Beskrivelse:	Denne foredragsrække består af tre uafhængige dele.

% Denne del introducerer emnet penetrationstest, hvad er det og hvad
% er værdien for dig. Emner der gennemgås er blandt andet:

% * Regler og etik for penetrationstest (ISC)² Code of Ethics
% * Informationsindsamling - aktiv og passiv
% * Portscan med nmap - TCP og UDP portscanning
% * Servicescanning - identifikation af porte og protokoller
% * Sårbarheder
% * Exploits og introduktion til buffer overflows
% * Bruteforcing online og offline værktøjer
% * Opsamling og præsentation af data

% Der vil være demonstrationer af sårbarheder på alle foredragene -
% typisk med open source programmer, således at deltagerne kan afprøve
% de selvsamme demoer hjemme.

% Note: der tages udgangspunkt i open source og Unix, men resultater og principper kan overføres til ASP og .NET teknologierne.

\begin{document}

%\slide{}

\mytitlepage
{Digitalt selvforsvar og\\
Masseovervågnings-paranoia! feat. Peter Kofod}

%\begin{alltt}
%\tiny
%\centerline{$Id: pentest-I-foredrag.tex,v 1.2 2007/10/04 12:17:42 hlk Exp $}
%\end{alltt}

\vskip 2cm
\centerline{\footnotesize Slides are available as PDF, kramshoej@Github}

\LogoOn

%\dagsplan
\slide{Planen idag}

\hlkimage{10cm}{Shaking-hands_web.jpg}

\begin{list1}
\item Kl 14-17 m pause
\item Mindre foredrag mere snak
\item Mindre enetale, mere foredrag 2.0 med socialt medie, informationsdeling og interaktion
\item Iaften Cryptoparty feat. Henrik Kramshøj, Peter Kofod m.fl.
\end{list1}

\slide{Formålet i dag}
\vskip 2 cm

%{\hlkbig En 3 dages workshop, hvor du lærer at angribe dit netværk!}
\hlkimage{3cm}{dont-panic.png}
\centerline{\color{titlecolor}\LARGE Don't Panic!}


\begin{list1}
\item Introducere eksempler på trusler
\item Demonstrere hvordan overvågning i praksis kan lade sig gøre med det trådløse netværk som eksempel
\end{list1}

\centerline{\color{red} Vi kommer til at sniffe på netværket}

Sluk din telefon og laptops trådløse kort hvis du er paranoid \smiley

\slide{Joanna Rutkowska has a new iPhone}

\hlkimage{13cm}{rootkovska-iphone.png}

\centerline{This is what people do with their new phones now}


\slide{Jake Appelbaum Motorola Moto E}

\begin{quote}
The Motorola Moto E (model: XT1021 and related devices) is an affordable modern Android cellphone. It may be purchased in cash at your local MediaMarkt for around 100 Euros. It is easy to modify for your everyday surveillance detection, counter-surveillance and anti-surveillance needs.
\end{quote}

\hlkimage{20cm}{ioerror-moto-e.png}
\link{https://people.torproject.org/~ioerror/skunkworks/moto_e/}

\slide{Killyourphone: stop signalerne når vi fester}

\hlkimage{20cm}{killyourphone.png}

\link{http://killyourphone.com/}

\slide{Opfordring: Køb dine egne gadgets!}

\begin{list1}
\item Hvis du køber dine egne enheder bestemmer du:
\item Modellen, fabrikat, features
\item Indstillingerne: sikkerhedsindstillingerne, fuld disk kryptering
\item Hvilke cloudservices du vil bruge
\item Hvilken backup service, hvor og hvordan
\end{list1}

\centerline{Bemærk mange dimser styrer vi ikke 100\% - iPhone eksempelvis}

Der er eksempler på at arbejdspladsen kan kræve du åbner en firma-laptop, for politiet

\slide{Gode råd til jer}

\begin{list1}

\item Brug teknologien
\item Lær teknologien at kende - læs manualen!
\item Tænk på følsomheden af data I gemmer og overfører
\begin{list2}
\item Slå ting fra som I ikke bruger
\item Slå bluetooth fra når I ikke bruger den
\item Opdater softwaren på enheden
\item Slå kryptering til hvor I kan: IMAP{\bf S}, POP3{\bf S},
  HTTP{\bf S} og over bluetooth
\item Brug låsekode fremfor tastaturlås uden kode på mobiltelefonen
\item Stol ikke for meget på fingertryksaflæsere
\end{list2}

\vskip 2cm
\centerline{\large Gælder alle enheder og steder I gemmer data}

\end{list1}


\slide{Hacking er magi}

\hlkimage{7cm}{wizard_in_blue_hat.png}

\vskip 1 cm

\centerline{Hacking ligner indimellem  magi}


\slide{Hacking er ikke magi}

\hlkimage{17cm}{ninjas.png}

\vskip 1 cm
\centerline{Hacking kræver blot lidt ninja-træning}


\slide{Movie:Kryptonite lock - old}

\hlkimage{16cm}{youtube-bic-lock.png}

\begin{list1}
\item Just search for: kryptonite lock bic pen
\item \link{https://www.youtube.com/watch?v=LahDQ2ZQ3e0}
\end{list1}




\slide{Hacking eksempel - det er ikke magi}

\hlkimage{20cm}{ethernet-frame-1.pdf}

\begin{list1}
\item Mange af vores protokoller er åbne og usikre
\item Mange af vores programmer, apps og websites er usikre
\end{list1}



\slide{Øvelse: Snif på netværket}

\hlkimage{6cm}{exercise}

\begin{list1}
\item Jeg bruger Kali Linux til at lytte på det trådløse netværk
\item Vi starter med at se
\begin{list2}
\item Hvor mange og hvilke systemer, fabrikat
\item Hvad sker der på netværket, broadcast traffik
\item passiv vs aktiv informationsindsamling
\item Nmap scans og webservere
\item og hvad der lige dukker op \smiley
\end{list2}
\end{list1}


\slide{Opsummering og client side anbefalinger}

\hlkimage{26cm}{drive-by-download-wikipedia.png}

\vskip 1cm
\centerline{Kan vi undvære Java, Flash og PDF?}

\begin{list1}
\item SKAL din laptop hedde dit navn på netværket?
\begin{list2}
\item Gå alle dine indstillinger igennem, tag dem med iaften, hjælp hinanden
\end{list2}
\end{list1}

Kilde: \link{https://en.wikipedia.org/wiki/Drive-by_download}



\slide{HLK del 2. mere teknik}


\slide{Normal WLAN brug}

\hlkimage{20cm}{images/wlan-airpwn-1.pdf}

\slide{Packet injection - airpwn}

\hlkimage{20cm}{images/wlan-airpwn-2.pdf}





\slide{Hackerværktøjer}

Vi kommer til at bruge hackerværktøjer
\begin{list1}
\item \emph{Improving the Security of Your Site by Breaking Into it} af
Dan Farmer og Wietse Venema i 1993
\item De udgav i 1995 så en softwarepakke med navnet SATAN
\emph{Security Administrator Tool for Analyzing Networks}
\item De forårsagede en del panik og furore, alle kan hacke, verden bryder sammen

\vskip 1cm
\begin{quote}
We realize that SATAN is a two-edged sword -- like
many tools, it can be used for good and for evil
purposes. We also realize that intruders (including
wannabees) have much more capable (read intrusive)
tools than offered with SATAN.
\end{quote}
\end{list1}

\vskip 1cm
Kilde:
\link{http://www.fish2.com/security/admin-guide-to-cracking.html}


\slide{Brug hackerværktøjer!}

\begin{list1}
\item Hackerværktøjer -- bruger I dem? -- efter dette kursus gør I
\item Portscannere kan afsløre huller i forsvaret
\item Webtestværktøjer som crawler igennem et website og finder alle
  forms kan hjælpe
\item I vil kunne finde mange potentielle problemer proaktivt ved
  regelmæssig brug af disse værktøjer -- også potentielle driftsproblemer
\item Husk dog penetrationstest er ikke en sølvkugle
\item Honeypots kan måske være med til at afsløre angreb og
  kompromitterede systemer hurtigere
\end{list1}


\slide{Aftale om test af netværk}

{\bfseries Straffelovens paragraf 263 Stk. 2. Med bøde eller fængsel indtil 1 år og 6 måneder straffes den, der uberettiget skaffer sig adgang til en andens oplysninger eller programmer, der er bestemt til at bruges i et informationssystem. }

Hacking kan betyde:
\begin{list2}
\item At man skal betale erstatning til personer eller virksomheder
\item At man får konfiskeret sit udstyr af politiet
\item At man, hvis man er over 15 år og bliver dømt for hacking, kan
  få en bøde -- eller fængselsstraf i alvorlige tilfælde
\item At man, hvis man er over 15 år og bliver dømt for hacking, får
en plettet straffeattest. Det kan give problemer, hvis man skal finde
et job eller hvis man skal rejse til visse lande, fx USA og
Australien
\item Frygten for terror har forstærket ovenstående -- så lad være!
\end{list2}


\slide{Internet idag}

\hlkimage{14cm}{images/server-client.pdf}

\begin{list1}
\item Klienter og servere
\item Rødder i akademiske miljøer
\item Protokoller der er op til 20 år gamle
\item Meget lidt kryptering, mest på http til brug ved e-handel
\end{list1}

\slide{OSI og Internet modellerne}

\hlkimage{14cm,angle=90}{images/compare-osi-ip.pdf}



\slide{Most vulnerable operating systems in 2014}

\hlkimage{18cm}{GFI-vulns-2014-OS-chart.jpg}

\begin{quote}
An average of 19 vulnerabilities per day were reported in 2014, according to the data from the National Vulnerability Database (NVD).
\end{quote}

Source:\\
{\footnotesize
\link{http://www.gfi.com/blog/most-vulnerable-operating-systems-and-applications-in-2014/}}


\slide{Most vulnerable applications in 2014}

\hlkimage{16cm}{gfi-vulns-application-chart.jpg}

\begin{quote}\small
Not surprisingly at all, web browsers continue to have the most security vulnerabilities because they are a popular gateway to access a server and to spread malware on the clients.
\end{quote}

Source:\\
{\footnotesize
\link{http://www.gfi.com/blog/most-vulnerable-operating-systems-and-applications-in-2014/}}


\slide{April 2014: Heartbleed hacking}

\begin{alltt}\footnotesize
  06b0: 2D 63 61 63 68 65 0D 0A 43 61 63 68 65 2D 43 6F  -cache..Cache-Co
  06c0: 6E 74 72 6F 6C 3A 20 6E 6F 2D 63 61 63 68 65 0D  ntrol: no-cache.
  06d0: 0A 0D 0A 61 63 74 69 6F 6E 3D 67 63 5F 69 6E 73  ...action=gc_ins
  06e0: 65 72 74 5F 6F 72 64 65 72 26 62 69 6C 6C 6E 6F  ert_order&billno
  06f0: 3D 50 5A 4B 31 31 30 31 26 70 61 79 6D 65 6E 74  =PZK1101&payment
  0700: 5F 69 64 3D 31 26 63 61 72 64 5F 6E 75 6D 62 65  _id=1&{\bf card_numbe}
  0710: XX XX XX XX XX XX XX XX XX XX XX XX XX XX XX XX  {\bf r=4060xxxx413xxx}
  0720: 39 36 26 63 61 72 64 5F 65 78 70 5F 6D 6F 6E 74  {\bf 96&card_exp_mont}
  0730: 68 3D 30 32 26 63 61 72 64 5F 65 78 70 5F 79 65  {\bf h=02&card_exp_ye}
  0740: 61 72 3D 31 37 26 63 61 72 64 5F 63 76 6E 3D 31  {\bf ar=17&card_cvn=1}
  0750: 30 39 F8 6C 1B E5 72 CA 61 4D 06 4E B3 54 BC DA  {\bf 09}.l..r.aM.N.T..
\end{alltt}

\begin{list2}
\item Obtained using Heartbleed proof of concepts - Gave full credit card details
\item "can XXX be exploited" - yes, clearly! PoCs ARE needed\\
without PoCs even Akamai wouldn't have repaired completely!
\item The internet was ALMOST fooled into thinking getting private keys from Heartbleed was not possible - scary indeed.
\end{list2}



\slide{September 2015: Heartbleed vulnerable servers}

\hlkimage{14cm}{heartbleed-vulnerab-2015-sept.png}

Source: Data from Shodan and Shodan Founder John Matherly


\slide{Getting to your data: Google for it}

\hlkimage{16cm}{images/googledorks-1.pdf}

\begin{list1}
\item Google as a hacker tool? oprindeligt beskrevet af Johnny Long
\item Concept named googledorks when google indexes information not supposed to be public
\link{http://www.exploit-db.com/google-dorks/}
\end{list1}


\slide{Kali Linux the new backtrack}

\hlkimage{20cm}{kali-linux.png}

\begin{list1}
%\item BackTrack -- \link{http://www.backtrack-linux.org}
\item Kali -- \link{https://www.kali.org/} version 2.0 netop udkommet!
\item Wireshark -- \link{https://www.wireshark.org} avanceret netværkssniffer
\end{list1}


\slide{Wireshark -- grafisk pakkesniffer}

\hlkimage{20cm}{images/wireshark-website.png}

\centerline{\link{https://www.wireshark.org}}
\centerline{Både til Windows og Unix}


\slide{Brug af Wireshark}

\hlkimage{18cm}{images/wireshark-http.png}

\centerline{Læg mærke til filtermulighederne}

\slide{Hackerlab opsætning}

\hlkimage{10cm}{hacklab-1.png}

\begin{list2}
\item Hardware: en moderne laptop med CPU der kan bruge virtualiseting\\
Husk at slå virtualisering til i BIOS
\item Software: dit favoritoperativsystem, Windows, Mac, Linux
\item Virtualiseringssoftware: VMware, Virtual box, vælg selv
\item Hackersoftware: Kali som Virtual Machine \link{https://www.kali.org/}
\item Soft targets: Metasploitable, Windows 2000, Windows XP, ...
\end{list2}


\slide{Konsulentens udstyr wireless}

\hlkimage{16cm}{TL-WN722N.png}

\begin{list1}
\item Laptop or Netbook, I typically use USB wireless cards\\
{\bf NB: de indbyggede er ofte ringe - så check før køb ;-)}
\item Access Points - get a small selection for testing
\item Books:
\begin{list2}
\item
Kali Linux Wireless Penetration Testing: Beginner's Guide
Beginner's Guide, Vivek Ramachandran, Cameron Buchanan, March 2015\\
Also checkout his home page \link{http://www.vivekramachandran.com/}
\end{list2}
\end{list1}


\slide{Kali Nethunter}

\hlkimage{18cm}{kali-nethunter.png}

Source: \link{https://www.kali.org/kali-linux-nethunter/}

\slide{Kursusudbud, eksempelvis Kryptografi på Stanford}

%Stoler vi på de andre autentificeringsmetoder?}
\hlkimage{20cm}{crypto-class.png}

Åbent kursus på Stanford\\
\link{http://crypto-class.org/}

\myquestionspage

\end{document}
