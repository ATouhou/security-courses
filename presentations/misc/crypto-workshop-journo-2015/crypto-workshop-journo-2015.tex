\documentclass[20pt,landscape,a4paper,footrule]{foils}
\usepackage{crypto-slides}
%\usepackage{solido-network-slides}
\usepackage{pdf14}
%\usepackage{ulem}

% Basic things that we need are below
\selectlanguage{danish}

%\externaldocument{unix-audit-security-oevelser}
\externaldocument{\jobname-exercises}

%\slide{Pause}
%Er det tid til en lille pause?
%\hlkimage{15cm}{300px-Fozziecurtain.JPG}

\begin{document}

\mytitlepage{Cryptoworkshop hvad er det?}{}

%Key words:
%\vskip 2cm
\centerline{\footnotesize PDF available kramshoej@Github: \jobname}

\slide{agenda}

%Planen for Cryptoworkshop 20:30 - 22:00
Planen for Cryptoworkshop 20:10 - 21:50
\begin{list2}
\item Introduktion Peter Kofoed - optakten
\item Kort introduktion cryptoworkshop - denne præsentation
\item Minicryptoworkshop 10min Textsecure, Red Phone og Whispersystems Signal
\item Anonymiseringsværktøjet Tor
\item Minicryptoworkshop 10min Torbrowser install
\item Sikker email med OpenPGP - PGP/GPG/GPGTools
\item Minicryptoworkshop 30min PGP install

%\item Cryptoworkshop opdeling i grupper, Tor, PGP, OTR
%\item OTR bliver ved projektor med Alexander Færøy, @ahfaeroey
\end{list2}

Tak for praktiske hjælp fra \link{https://bitbureauet.dk/} Twitter: @bitbureauet

\slide{Advarsel}


\hlkimage{18cm}{dragon-drawing-6.jpg}

\begin{center}
Denne aften er en smagsprøve - introduktion

Du skal ikke føle dig sikker før du forstår programmerne bedre.

Brug, læs og lær - det er vigtigt at lære det at kende før man skal bruge det.
\end{center}


\slide{Internet Internals: A short story}

\hlkimage{12cm}{1969_4-node_map}

\centerline{1969: ARPANET 4 nodes}


\begin{list1}
\item Present some internet internals, what is it?
\end{list1}

\slide{TCP/IP basics }

\hlkimage{24cm}{sample-ip-network.pdf}
\vskip 1cm
\centerline{Internet Protocols since approx 1983}

\slide{Routing}

\hlkimage{14cm}{routing-2.pdf}

\centerline{Soo simple \smiley}

Dont forget IP routing is based on longest match in the routing table

\slide{Internet core today AS numbers}

\hlkimage{25cm}{network-bgp-asn.pdf}

Also pretty pictures in \emph{Intercountry BGP As Topology} Martin J. Levy\\
\link{http://oldwww.dknog.dk/dknog3/agenda}


\slide{BGP changes}


\begin{quote}
 A good analogy for the development of the Internet is that of
 constantly renewing the individual streets and buildings of a city,
 rather than razing the city and rebuilding it. The architectural
 principles therefore aim to provide a framework for creating
 cooperation and standards, as a small "spanning set" of rules that
 generates a large, varied and evolving space of technology.
\end{quote}
Source: RFC-1958

\begin{list2}
\item Constant change
\item Now ~50.000 ASNs announced, and ~80.000 allocated
\item Now ~515.000 prefixes - reach 550.000 routes/prefixes in 2015?
\end{list2}

Movietime
\link{https://stat.ripe.net/94.126.176.0/21\#tabId=routing}

\centerline{Lots of data to be found at \link{http://bgp.potaroo.net/}}

\slide{Passende paranoia}

\hlkimage{18cm}{GOOGLE-CLOUD-EXPLOITATION1383148810.jpg}
\centerline{SSL (encryption) added and removed here}



\slide{Solidaritetskryptering}

Hvorfor skal vi kryptere?

\begin{alltt}
       Køn
                       Seksualitet

 Tro religion       hatecrimes

 Politisk overbevisning, eller blot aktiv

 Whistleblowers             soldater      diplomater

\end{alltt}

\centerline{Du bestemmer ikke hvem der diskrimineres eller trues i andre lande}

\vskip2cm

Når vi krypterer hjælper vi andre! {\bf Solidaritetskryptering}

\slide{Kryptografi}

\hlkimage{18cm}{images/crypto-rot13.pdf}

\begin{list1}
\item Kryptografi er læren om, hvordan man kan kryptere data
\item Kryptografi benytter algoritmer som sammen med nøgler giver en
  ciffertekst - der kun kan læses ved hjælp af den tilhørende nøgle
\end{list1}

\slide{Public key kryptografi - 1}

\hlkimage{18cm}{images/crypto-public-key.pdf}

\begin{list1}
\item privat-nøgle kryptografi (eksempelvis AES) benyttes den samme
  nøgle til kryptering og dekryptering
\item offentlig-nøgle kryptografi (eksempelvis RSA) benytter to
  separate nøgler til kryptering og dekryptering
\end{list1}

\slide{Public key kryptografi - 2}

\hlkimage{18cm}{images/crypto-public-key-2.pdf}

\begin{list1}
\item offentlig-nøgle kryptografi (eksempelvis RSA) bruger den private
  nøgle til at dekryptere
\item man kan ligeledes bruge offentlig-nøgle kryptografi til at
  signere dokumenter\\ - som så verificeres med den offentlige nøgle
\end{list1}

\slide{Kryptering: Cryptography Engineering}

\hlkimage{8cm}{book-ce-150w.jpg}

\emph{Cryptography Engineering} by
Niels Ferguson, Bruce Schneier, and Tadayoshi Kohno
\link{https://www.schneier.com/book-ce.html}

\centerline{Kryptering sikrer fortrolighed og integritet af beskederne}



\slide{Hvad er cryptoworkshop}
\hlkimage{6cm}{crypto-party-logo.png}

Udspringer af CryptoParty bevægelsen som afholder kryptofester\\
\link{https://en.wikipedia.org/wiki/CryptoParty}

% Husk tools fra Frejas dokument
Iaften vil vi fokusere på disse:
\begin{list2}
%\item OTR Off-the-record Adium eller Pidgin, brug server cloak.dk
\item Android: TextSecure og Redphone
\item iPhone IOS installer Signal
\item Torproject - Tor Browser Bundle
\item OpenPGP - PGP/GPG Thunderbird Enigmail eller  Mac OS X GPGtools
\end{list2}

Note: følg også \link{https://cryptoparty.dk/}

\slide{Andre kilder til tools}

\hlkimage{12cm}{ssd-eff-logo.png}
\begin{list2}
\item Surveillance Self-Defense EFF guide
\link{https://ssd.eff.org/}
\item The Guardian Project Mobile Apps and Code You Can Trust\\
\link{https://guardianproject.info/}
\item Se citizenfour filmen - fik velfortjent Oscar!\\ {\footnotesize\link{http://www.wired.com/2014/10/laura-poitras-crypto-tools-made-snowden-film-possible/}}
\item Information Security for Journalists\\
\link{http://www.tcij.org/resources/handbooks/infosec}
\end{list2}



\slide{10minute cryptoworkshop}

\hlkimage{24cm}{textsecure-redphone.pdf}

Forsøg, hvor mange kan kommunikere sikkert indenfor 10min?

\begin{list2}
\item Android installer TextSecure og Redphone
\item iPhone IOS installer Signal
\end{list2}

\vskip 1cm
\centerline{Send krypteret SMS til en anden herinde og sig YEAH!}

og brug så krypteret SMS fremover \smiley

\slide{Tor project anonym webbrowsing}

\hlkimage{21cm}{tor-project.png}

\centerline{\link{https://www.torproject.org/}}

\centerline{Der findes alternativer, men Tor er mest kendt}

\slide{Tor project - how it works 1}

\hlkimage{21cm}{how-tor-works-1.png}

\centerline{pictures from \link{https://www.torproject.org/about/overview.html.en}}

\slide{Tor project - how it works 2}

\hlkimage{21cm}{how-tor-works-2.png}

\centerline{pictures from \link{https://www.torproject.org/about/overview.html.en}}

\slide{Tor project - how it works 3}

\hlkimage{21cm}{how-tor-works-3.png}

\centerline{pictures from \link{https://www.torproject.org/about/overview.html.en}}

\slide{Tor project install}

\hlkimage{12cm}{tor-project.png}

Der findes diverse tools til Tor, Torbutton on/off knap til Firefox osv.

Det anbefales at bruge Torbrowser bundles fra \link{https://www.torproject.org/}

\slide{Torbrowser - outdated}

\hlkimage{20cm}{torbrowser-outdated.png}

\centerline{\color{red}Hov den mangler opdatering!}

\slide{Torbrowser - anonym browser}

\hlkimage{20cm}{torbrowser-main-window.png}

\centerline{\color{titlecolor} Mere anonym browser - Firefox i forklædning}


\slide{Torbrowser - hidden service web site}

\hlkimage{18cm}{sample-tor-site.png}

\centerline{\color{titlecolor} .onion er Tor adresser - hidden sites}

%\centerline{Den viste side er SecureDrop hos Radio24syv}\\
\link{http://www.radio24syv.dk/dig-og-radio24syv/securedrop/}

\slide{Whonix - Tor to the max!}

\hlkimage{17cm}{400px-Whonix.jpg}

\begin{quote}
Whonix is an operating system focused on anonymity, privacy and security. It's based on the Tor anonymity network[5], Debian GNU/Linux[6] and security by isolation. DNS leaks are impossible, and not even malware with root privileges can find out the user's real IP. \link{https://www.whonix.org/}

\end{quote}

\centerline{Torbrowser er godt, Whonix giver lidt ekstra sikkerhed}





\slide{Email er usikkert}

\hlkimage{20cm}{email-uden-kryptering.png}

\centerline{Email uden kryptering - er som et postkort}



\slide{Email med kryptering - afsendelse}

\hlkimage{18cm}{email-med-kryptering.png}


\centerline{En sikker krypteret email er ikke sværere at sende}

\slide{Krypteret Email under transporten}

\hlkimage{11cm}{modtaget-email-med-kryptering.png}

\centerline{En sikker krypteret email er beskyttet undervejs}

\slide{Thunderbird Enigmail}

\hlkimage{22cm}{enigmail-homepage.png}

Enigmail er en udvidelse til Thunderbird email programmet\\ \link{https://www.enigmail.net}

\slide{Thunderbird Enigmail: Key management}

\hlkimage{15cm}{enigmail-keyman1.png}

\begin{list2}
\item Indbygget i Enigmail er funktionalitet til at generere nøgler
\item følg den \emph{wizard} som kommer frem.
\item {\bf Lav din første nøgle med 1 års levetid} - du ved mere om et år \smiley
\end{list2}


\slide{Thunderbird Enigmail: Compose email}

\hlkimage{18cm}{enigmail-compose.png}

Når du sender vælger du om der skal krypteres og signeres

\slide{Bonus: Full Disk Encryption Mac OS X}

\hlkimage{15cm}{apple-filevault-enabled.png}

\centerline{Indbygget, gratis, stærk - slå det til når I kommer hjem}

\slide{Bonus: DNS censur i Danmark}

\hlkimage{4cm}{Censored_rubber_stamp.png}

Hvis du er træt af den danske censur på DNS, så kan du skifte til at bruge:\\
Censurfridns.dk UncensoredDNS

\begin{list2}
\item anycast.censurfridns.dk / 91.239.100.100 / 2001:67c:28a4::
\item ns1.censurfridns.dk / 89.233.43.71 / 2002:d596:2a92:1:71:53::
\end{list2}
Se også \link{http://www.censurfridns.dk} og\\
 \link{blog.censurfridns.dk} for mere info.

\vskip 2cm

\centerline{\Large Det er uacceptabelt at pille ved DNS - punktum!}


% The end
\myquestionspage

\end{document}
\input{references.tex}
