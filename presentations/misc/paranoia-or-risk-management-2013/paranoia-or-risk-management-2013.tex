\documentclass[20pt,landscape,a4paper,footrule]{foils}
\usepackage{solido-network-slides}
\usepackage{pdf14}
%\usepackage{ulem}

% Basic things that we need are below
\selectlanguage{danish}

%\externaldocument{unix-audit-security-oevelser}
\externaldocument{\jobname-exercises}

\begin{document}

\mytitlepage
{Paranoia or RISK management}{2013}




\slide{Agenda}
\vskip 2 cm

\hlkimage{5cm}{dont-panic.png}
\centerline{\color{titlecolor}\LARGE Don't Panic!}

\begin{list1}
\item Kl 16:00-18:30 presentation
\item Less presentation, more social interaction, sharing information
\end{list1}
\centerline{You are welcome to email questions later}
%PS er her hele weekenden

\slide{Goals: Increase Security Awareness}

\hlkimage{10cm}{homer-end-is-near.jpg}
\begin{list1}
\item Fact of life: Software has errors, hardware fails
\item Sometimes software can be made to fail in interesting ways
\item Humans can be social engineered
\item We are being attacked by criminals - including paranoid governments
\end{list1}

\slide{Detailed agenda}

\begin{list1}
\item Part I: Paranoia defined
\item Part II: What are the vulnerabilities and threats
\item Part III: Reduce risk and mitigate impact
\end{list1}

\slide{Security is not magic}

%\vskip 2cm

\hlkimage{4cm}{wizard_in_blue_hat.png}

\hlkrightimage{5cm}{003scawebgoshindomanicon.png}
.
%{\large Superheltegerninger}

\begin{list1}
\item Think security, it may seem like magic - but it is not
\item Follow news about security
\item Support communities, join and learn
\end{list1}


\slide{Part I: Paranoia defined}

\hlkimage{15cm}{paranoia-definition.png}

Source: google paranoia definition

\slide{Face reality}

From the definition:
\begin{quote}
suspicion and mistrust of people or their actions {\bf without evidence or justification}.
{\bf "the global paranoia about hackers and viruses"}
\end{quote}

\begin{list1}
\item It is not paranoia when:
\begin{list2}
\item Criminals sell your credit card information and identity theft
\item Trade infected computers like a commodity
\item Governments write laws that allows them to introduce back-doors - and use these
\item Governments do blanket surveillance of their population
\item Governments implement censorship, threaten citizens and journalist
\end{list2}
\end{list1}

\vskip 1cm
\centerline{You are not paranoid when there are people actively attacking you!}


\slide{Credit card fraud and identity theft statistics}

\hlkimage{19cm}{credit-card-fraud.png}

{\small Source: 
\link{http://www.statisticbrain.com/credit-card-fraud-statistics/}}


\slide{Identity theft statistics}

\hlkimage{19cm}{identity-theft-stat.png}

{\small Source: 
\link{http://www.statisticbrain.com/identity-theft-fraud-statistics/}}




\slide{Trading in infected computers}

Botnets and malware today sold as SaaS with support contracts and updates

\hlkimage{21cm}{dagens-tilbud-trojanere.pdf}

\begin{list1}
\item Malware programmers do better support than regular software companies
\item "Buy this version and get a year of updates free"
\item Rent our botnet with 100,000 by the hour
\end{list1}



\slide{Government back-doors}

What if I told you:

{\Large \bf Governments will introduce back-doors}


\begin{list1}
\item Intercepting encrypted communications with fake certificates - check
\item May 5, 2011 A Syrian Man-In-The-Middle Attack against Facebook\\
"Yesterday we learned of reports that the Syrian Telecom Ministry had launched a man-in-the-middle attack against the HTTPS version of the Facebook site."\\
{\small Source:\\
\link{https://www.eff.org/deeplinks/2011/05/syrian-man-middle-against-facebook}}
\item Mapping out social media and finding connections - check
\end{list1}



\slide{Infecting activist machines}
\begin{list1}
\item Infecting activist machines - check
\item Tibet activists are repeatedly being targeted with virus and malware, such as malicious apps for Android like KakaoTalk
% http://www.scmagazine.com/android-malware-targeting-tibetans-has-state-sponsored-fingerprints/article/287219/#
% http://www.scmagazine.com//trojan-targets-tibetan-activist-groups-that-use-macs/article/234132/
\item TOR-users infected with malicious code to reveal their real IPs\\
{\footnotesize\link{https://blog.torproject.org/blog/hidden-services-current-events-and-freedom-hosting}}

\end{list1}

\slide{UK: Seize smart phones and download data}

\begin{quote}
Officers use counter-terrorism laws to remove a mobile phone from any passenger they wish coming through UK air, sea and international rail ports and then scour their data.

The blanket power is so broad they do not even have to show reasonable suspicion for seizing the device and can retain the information for "as long as is necessary".

Data can include call history, contact books, photos and who the person is texting or emailing, although not the contents of messages.
\end{quote}


{\small Source:
http://www.telegraph.co.uk/technology/10177765/Travellers-mobile-phone-data-seized-by-police-at-border.html}


\slide{UK wouldn't seize data like that, you are lying}

\begin{quote}
(Reuters) - British authorities came under pressure on Monday to explain why anti-terrorism powers were used to detain for nine hours the partner of a journalist who has written articles about {\bf U.S. and British surveillance programs} based on {\bf leaks from Edward Snowden}.

Brazilian David Miranda, the partner of American journalist Glenn Greenwald, was detained on Sunday at London's Heathrow Airport where he was in transit on his way from Berlin to Rio de Janeiro. {\bf He was released without charge}.
\end{quote}

{\small Source:\\
{http://www.reuters.com/article/2013/08/19/us-britain-snowden-detention-idUSBRE97I0J520130819}}


\slide{Skype is insecure}

\begin{quote}
August 7, 2013
Restoring Trust in Government and the Internet
In July 2012, responding to allegations that the video-chat service Skype -- owned by Microsoft -- was changing its protocols to make it possible for the government to eavesdrop on users, Corporate Vice President Mark Gillett took to the company's blog to deny it.

Turns out that wasn't quite true.
\end{quote}

\centerline{\bf So Skype owned by Microsoft is not trustworthy - stop the presses!}

 Source:\\ 
{\small\link{http://www.schneier.com/blog/archives/2013/08/restoring_trust.html}}



\slide{Government backdoors is not news}

\hlkimage{12cm}{nation-of-sheep.jpg}

\begin{list1}
\item Nothing new really, see for example D.I.R.T and Magic Lantern
\item D.I.R.T - Data Interception by Remote Transmission since the late 1990s\\
\link{http://cryptome.org/fbi-dirt.htm}\\
\link{http://cryptome.org/dirty-secrets2.htm}

\item They will always use \emph{Le mal du jour} to increase monitoring
\end{list1}

\slide{Government monitoring is not news}

\begin{list1}
\item FBI Carnivore\\
"... that was designed to monitor email and electronic communications. It used a customizable packet sniffer that can monitor all of a target user's Internet traffic."
\link{http://en.wikipedia.org/wiki/Carnivore_(software)}

\item NarusInsight
"Narus provided Egypt Telecom with Deep Packet Inspection equipment, a content-filtering technology that allows network managers to inspect, track and target content from users of the Internet and mobile phones, as it passes through routers on the information superhighway. Other Narus global customers include the national telecommunications authorities in Pakistan and Saudi Arabia, ..."\\
\link{http://en.wikipedia.org/wiki/NarusInsight}

\end{list1}

\slide{Denmark}

\begin{list1}
\item Even Denmark which is considered a peaceful democracy has allowed this to go TO FAR
\item Danish police and TAX authorities have the legals means, even for small tax-avoidance cases, see \emph{Rockerloven}
\item Danish TAX authorities have legal means to go into your property to catch builders working for cash and not reporting tax income
\item In both criminal and piracy cases we see a lot of extraneous equipment seized 
\end{list1}

\slide{Governments blanket surveillance}

\hlkimage{10cm}{prism_logo.jpg}

\begin{list1}
\item NSA - need we say more?\\
\link{http://en.wikipedia.org/wiki/PRISM_(surveillance_program)}
\item Governments also implementing censorship
\item Outlaw and/or discredit crypto
\item Go after TOR exit nodes
\end{list1}


\slide{Use protection - always}

\hlkimage{14cm}{protect-from-governments.jpg}
%{\LARGE Protecting yourself against criminals or the government is the same thing!}

\slide{A vulnerability can and will be abused}

What if I told you:

{\Large \bf Criminals will be happy to leverage backdoors created by government}

It does not matter if the crypto product has a weakness to allow investigations or the software has a backdoor to help law enforcement. Data and vulnerabilities WILL be abused and exploited.




\slide{Part II: What are the vulnerabilities and threats}

Hackers do not discriminate

\begin{list1}
\item We have seen lots of hacker stories, and we learn:
\item We are all targets of hacking
\item Social Engineering rockz! Phishing works.
\item Anyone can be hacked - resources used to protect vs attackers resources
\end{list1}

\vskip 2cm
\centerline{\LARGE \bf Hacking is not cool}


\slide{Good security}

\hlkimage{15cm}{god-sikkerhed.pdf}

\begin{list1}
\item You always have limited resources for protection - use them as best as possible
\end{list1}


\slide{First advice}

\begin{list1}
\item Use technology
\item Learn the technology - read the freaking manual
\item Think about the data you have, upload, facebook license?! WTF!
\item Think about the data you create - nude pictures taken, where will they show up?
\begin{list2}
\item Turn off features you don't use
\item Turn off network connections when not in use
\item Update software and applications
\item Turn on encryption: IMAP{\bf S}, POP3{\bf S},
  HTTP{\bf S} also for data at rest, full disk encryption, tablet encryption
\item Lock devices automatically when not used for 10 minutes
\item Dont trust fancy logins like fingerprint scanner or face recognition on cheap devices
\end{list2}
\end{list1}

\slide{Why think of security?}

\hlkimage{8cm}{1984-not-instruction-manual.jpg}


\begin{quote}
	Privacy is necessary for an open society in the electronic age. Privacy is not secrecy. A private matter is something one doesn't want the whole world to know, but a secret matter is something one doesn't want anybody to know. Privacy is the power to selectively reveal oneself to the world. ~A Cypherpunk's Manifesto by Eric Hughes, 1993
\end{quote}

Copied from \link{https://cryptoparty.org/wiki/CryptoParty}



\slide{Evernote password reset}

What happens when security breaks?
\hlkimage{21cm}{evernote-password-reset.png}

Sources:\\
\link{http://evernote.com/corp/news/password_reset.php}

\slide{Twitter password reset}

\hlkimage{18cm}{twitter-250k-users.png}

Sources:\\
\link{http://blog.twitter.com/2013/02/keeping-our-users-secure.html}

\slide{January 2013: Github Public passwords?}


\hlkimage{20cm}{github-credentials.png}

 Sources:\\
{\footnotesize\link{https://twitter.com/brianaker/status/294228373377515522}\\
\link{http://www.webmonkey.com/2013/01/users-scramble-as-github-search-exposes-passwords-security-details/}\\
\link{http://www.leakedin.com/}\\
\link{http://www.offensive-security.com/community-projects/google-hacking-database/}
}

\vskip 5mm
\centerline{Use different passwords for different sites, yes - every site!}

%Sidetrack: Mat Honan epic hacking :-( using password reset\\
%\link{http://www.wired.com/gadgetlab/2012/08/apple-amazon-mat-honan-hacking/all/}

\slide{Opbevaring af passwords}

\hlkimage{10cm}{password-window.png}

\vskip 5mm
\centerline{Use some kind of Password Safe program which encrypts your password database}



\slide{Hacker types anno 2008}
\hlkimage{10cm}{lisbeth-salander.jpeg}

\begin{list1}
\item Lisbeth Salander from the Stieg Larsson's award-winning Millennium series
does research about people using hacking as a method to gain access
\item How can you find information about people?
\end{list1}

\slide{From search patterns to persons}

\begin{list1}
\item First identify some basic information
\item Use search patterns like from email to full name
\item Some will give direct information about target
\item Others will point to intermediary information, domain names
\item Pivot and redo searching when new information bits are found
\item What information is public? (googledorks!)
\end{list1}

\slide{Example patterns - for a Dane}

\begin{list1}
\item Name, fullname, aliases
\item IDs and membership information, CPR (kind a like social security number)
\item Computerrelated information: IP, Whois, Handles, IRC nicks
\item Nick names
\item Writing style, specific use of words, common speling mistakes
\item Be creative
\end{list1}


\slide{Google for it}

\hlkimage{16cm}{images/googledorks-1.pdf}

\begin{list1}
\item Google as a hacker tools?
\item Concept named googledorks when google indexes information not supposed to be public
\link{http://johnny.ihackstuff.com/}
\end{list1}


\slide{Listbeth in a box?}

\hlkimage{17cm}{maltego-1.png}

%\hlkimage{17cm}{Paterva_network_small.png}
Maltego can automate the mining and gathering of information\\
uses the concept of transformations\\ \link{http://www.paterva.com/maltego/}

\slide{Phishing - Receipt for Your Payment to mark561@bt....com}
\hlkimage{21cm}{paypal-phish.png}

\pause
\centerline{\bf\LARGE\color{titlecolor}Kan du selv genkende Phishing}


\slide{Zip files?}

\hlkimage{18cm}{emailvirus-zipfile.pdf}

\slide{Money!}

\hlkimage{17cm}{emailscam-banking.pdf}


\slide{Spearphishing - targetted attacks}


Spearphishing - targetted attacks directed at specific individuals or companies 

\begin{list2}
\item Use 0-day vulnerabilities only in a few places
\item Create backdoors and mangle them until not recognized by Anti-virus software
\item Research and send to those most likely to activate program, open file, visit page
\item Stuxnet is an example of a targeted attack using multiple 0-day vulns
\end{list2}


\slide{Mobile devices}
\begin{list1}
\item What characterizes mobile devices
\begin{list2}
\item Small - sometimes can fit in a pocket
\item Less resources, smaller CPU and memory than PC
\item Limited functionality, limited control - can be rooted (sometimes)
\item Synchronize with data from multiple sources
\item Storage has increased to +32Gb
\item Has \emph{viewer programmer} for Word, Excel, PDF m.fl.
\item Has browser built-in - often not changed
\item Always on - mobile 3G/4G and Wifi - connected most of the day
\end{list2}
\end{list1}

\slide{Bluetooth security}

\hlkimage{16cm}{images/kramse-apple-bt.png}

\begin{list2}
\item All small devices also have bluetooth
\item Bluetooth - turn it off when not in use
\item In your car - built-in bluetooth, GPS has bluetooth?
\item Turn on security features for bluetooth allow access on to \emph{paired} devices
\end{list2}


\slide{Car Whisperer using bluetooth}

\hlkimage{10cm}{images/car_whisperer.jpg}

\begin{list1}  
\item Bluetooth kits for cars use passkey like '0000' or '1234'
\end{list1}

Sources:\\
\link{http://trifinite.org/blog/archives/2005/07/introducing_the.html}\\
\link{http://trifinite.org/trifinite_stuff_carwhisperer.html}


\slide{Problems with mobile devices}


\begin{list1}
\item Can store a lot of data - sensitive data can be lost
\item Has microphone, camera, GPS location, tracking/stalking
\begin{list2}
\item Calendar 
\item Contacts and email, 
\item Tasks - To Do listen
\end{list2}
\item Easy access to data - easy to get the data
\begin{list2}
\item Vendors make it easy to switch device, move data to new device
\item Phones can often move data without inserting {\color{red} SIM card}
\item Backup of data to memory cards - copy all data in minutes
\end{list2}
\item Access to the company network using VPN or wireless?
\begin{list2}
\item Reuse some login data from mobile devices and connect laptop to the network
\end{list2}
\end{list1}



\slide{Theft - kindergarten and airports}

\begin{list1}
\item Many parents are in a hurry when they are picking up their kids
\item Many people can easily be distracted around crowds
\item Many people let their laptops stay out in the open - even at conferences
\item ... making theft likely/easy
\vskip 1 cm
\item Stolen for the value of the hardware - or for the data?
\item Industrial espionage, economic espionage or corporate espionage is real
\end{list1}


\slide{Are your data secure}

\hlkimage{15cm}{images/data-integrity-1.pdf}

\begin{list1}
\item Stolen laptop, tablet, phone - can anybody read your data?
\item Do you trust "remote wipe"
\item How do you in fact wipe data securely off devices, and SSDs?
\item Encrypt disk and storage devices before using them in the first place!
\end{list1}


\slide{Circumvent security - single user mode boot}
\begin{list1}
\item Unix systemer often allows boot into singleuser mode\\
press command-s when booting Mac OS X 
\item Laptops can often be booted using PXE network or CD boot
\item Mac computers can become a Firewire disk\\
hold t when booting - firewire target mode
\item Unrestricted access to un-encrypted data
\item Moving hard drive to another computer is also easy
\end{list1}
\pause
\centerline{Physical access is often - {\bf game over}}


\slide{Encrypting hard disk}

\hlkimage{17cm}{images/apple-filevault.png}

\begin{list1}
\item Becoming available in the most popular client operating systems
\begin{list2}
\item Microsoft Windows Bitlocker - requires Ultimate or Enterprise
\item Apple Mac OS X - FileVault og FileVault2
\item FreeBSD GEOM og GBDE - encryption framework
\item Linux distributions like Ubuntu ask to encrypt home dir during installation
\item PGP disk - Pretty Good Privacy - makes a virtuel krypteret disk
\item TrueCrypt - similar to PGP disk, a virtual drive with data, cross platform
\item Some vendors have BIOS passwords, or disk passwords
\end{list2}
\end{list1}

\slide{Firewire attacks}

\begin{list1}
\item Firewire, DMA \& Windows, Winlockpwn via FireWire\\
Hit by a Bus: Physical Access Attacks with Firewire Ruxcon 2006
\vskip 5mm
\item Removing memory from live system - data is not immediately lost, and can be read under some circumstances\\
Lest We Remember: Cold Boot Attacks on Encryption Keys\\
\link{http://citp.princeton.edu/memory/}
\vskip 1cm 

\item This is very CSI or Hollywoord like - but a real threat 
\end{list1}

\centerline{So perhaps use both hard drive encryption AND turn off computer after use?}

\slide{... and deleting data}

\hlkimage{14cm}{dban-screenshot.png}

\begin{list1}
\item Getting rid of data from old devices is a pain
\item Some tools will not overwrite data, leaving it vulnerable to recovery
\item Even secure erase programs might not work on SSD - due to reallocation of blocks
\item I have used Darik's Boot and Nuke ("DBAN") \link{http://www.dban.org/}
\end{list1}



\slide{Client side: Flash, PDF, Facebook}

\hlkimage{26cm}{drive-by-download-wikipedia.png}

\vskip 1cm
\centerline{Can we avoid using Flash and PDF?}

Source: \link{http://en.wikipedia.org/wiki/Drive-by_download}

\slide{Flashback}

\hlkimage{18cm}{flash-security-2011}
Source: \link{http://www.locklizard.com/adobe-flash-security.htm}


\slide{Flash blockers}

\hlkimage{6cm}{clicktoflash.png}

\begin{list1}
\item Safari \link{http://clicktoflash.com/}
\item Firefox Extension Flashblock and NoScript
\item Chrome extension called FlashBlock and built-in configurable setting Click to play 
\item Internet Explorer: IE has the Flash block functionality built-in so you don't need to install any additional plugins to be able to block flash on IE 8.
\item FlashBlockere til iPad? iPhone? Android?
\item why aren't Flash blockers on by default?
\item Note it is easy to inject malicious javascript and flash when sharing a wireless network
\end{list1}



\slide{Secure protocols}

\begin{list1}
\item Securing e-mail
\begin{list2}
\item Pretty Good Privacy - Phil Zimmermann
\item OpenPGP = e-mail security
\end{list2}
\item Network sessions use SSL/TLS
\begin{list2}
\item Secure Sockets Layer SSL / Transport Layer Services TLS
\item Encrypting data sent and received
\item SSL/TLS already used for many protocols as a wrapper: POP3S, IMAPS, SSH, SMTP+TLS m.fl.
\end{list2}
\item Encrypting traffic at the network layer - Virtual Private Networks VPN
\begin{list2}
\item {\color{green}IPsec IP Security Framework, se ogs� L2TP}
\item {\color{red} PPTP Point to Point Tunneling Protocol - d�rlig og usikker, brug den ikke mere!}
\item OpenVPN uses SSL/TLS across TCP or UDP
\end{list2}
\end{list1}

\centerline{Note: SSL/TLS is not trivial to implement, key management!}




\slide{Basic tools - PGP}

\hlkimage{5cm}{images/pgp.png}

\begin{list2}
\item Pretty Good Privacy - PGP 
\item Originally developed by Phil Zimmermann 
\item Now a commecrial entity \link{http://www.pgp.com}
\item Source exported from USA on paper and scanned outside - which was legal 
\item \link{http://www.pgpi.org}
\end{list2}


\slide{Basic tools - GPG}

\hlkimage{5cm}{images/logo-gnupg.png}
\begin{list1}
\item Gnu Privacy Guard, GnuPG or GPG
\item Web site:
  \link{http://www.gnupg.org/}
\item Open Source - GPL license 
\item Available for most popular operating systems
\item Highly recommended
\end{list1}


\slide{Enigmail - GPG plugin til Mail}

\hlkimage{20cm}{images/enigmail-compose}
\begin{list2}
\item Enigmail is a plugin for the Thunderbird mail client
\item Screenshot from \link{http://enigmail.mozdev.org}
\end{list2}


\slide{Enigmail - OpenGPG Key Manager}

\hlkimage{18cm}{images/enigmail-keyman1}

\vskip 1cm
\centerline{Key Manager built-int Enigmail is recommended}  

\slide{GPGMail plugin for Mac OS X Mail.app}

\hlkimage{20cm}{images/gpgmail-new-message.png}      

\begin{list2}
\item Uses GPG and is part of the GPGTools
\item \link{https://gpgtools.org/}
\end{list2}


\slide{Filetransfer programs FileZilla - SFTP}

\hlkimage{19cm}{filezilla.png}

\begin{list1}
\item Stop using FTP! Dammit!
\item Lots of programs support SFTP and SCP for secure copying of data
\item \link{http://filezilla-project.org/}
\end{list1}

\slide{HTTPS Everywhere}

\hlkimage{5cm}{HTTPS_Everywhere_new_logo.jpg}
\begin{quote}
HTTPS Everywhere is a Firefox extension produced as a collaboration between The Tor Project and the Electronic Frontier Foundation. It encrypts your communications with a number of major websites.
\end{quote}

\centerline{\link{http://www.eff.org/https-everywhere}}


\slide{Convergence - who do you trust}

\hlkimage{21cm}{convergence.png}


\centerline{\link{http://convergence.io/}}
\centerline{Warning: radical change to how certificates work}


\slide{VPN}

\hlkimage{12cm}{openvpn-gui-systray.png}

\begin{list1}
\item Virtual Private Networks are useful - or even required when travelling
\item VPN \link{http://en.wikipedia.org/wiki/Virtual_private_network}
\item SSL/TLS VPN - Multiple incompatible vendors: OpenVPN, Cisco, Juniper, F5 Big IP
\item IETF IPsec does work cross-vendors - sometimes, and is also increasingly becoming blocked or unusable due to NAT :-(
\end{list1}



\slide{Tor project}

\hlkimage{21cm}{tor-project.png}

\centerline{\link{https://www.torproject.org/}}

\slide{Tor project - how it works 1}

\hlkimage{21cm}{how-tor-works-1.png}

\centerline{pictures from \link{https://www.torproject.org/about/overview.html.en}}

\slide{Tor project - how it works 2}

\hlkimage{21cm}{how-tor-works-2.png}

\centerline{pictures from \link{https://www.torproject.org/about/overview.html.en}}

\slide{Tor project - how it works 3}

\hlkimage{21cm}{how-tor-works-3.png}

\centerline{pictures from \link{https://www.torproject.org/about/overview.html.en}}


\slide{Hackers and ressources}
% m�ske til reference afsnit?
\hlkimage{3cm}{hackers_JOLIE+1995.jpg}

\begin{list1}
\item Hackers work all the time to break stuff
\item Use hackertools:
\begin{list2}
\item Nmap, Nping - test network ports \link{http://nmap.org}
\item Wireshark advanced network analyzer - \link{http://http://www.wireshark.org/} 
\item Metasploit Framework exploit development and delivery \link{http://www.metasploit.com/}
%\item Paros proxy \link{http://www.parosproxy.org}
\item Burpsuite web scanner and proxy \link{http://portswigger.net/burp/}
\item Skipfish web scanner \link{http://code.google.com/p/skipfish/}
\item Kali Linux pentesting operating system \link{http://www.kali.org} 
\item Most used hacker tools \link{http://sectools.org/}
\end{list2}
\end{list1}

Picture: Angelina Jolie as \emph{Kate Libby/Acid Burn} Hackers 1995



\slide{Part III: Reduce risk and mitigate impact}

\slide{Risk management defined}

\hlkimage{24cm}{shon-harris-risk-management.png}

Source: Shon Harris \emph{CISSP All-in-One Exam Guide}


\slide{First advice use the modern operating systems}

\begin{list1}
\item Newer versions of Microsoft Windows, Mac OS X and Linux
\begin{list2}
\item Buffer overflow protection
\item Stack protection, non-executable stack
\item Heap protection, non-executable heap
\item \emph{Randomization of parameters} stack gap m.v.
\end{list2}
\item Note: these still have errors and bugs, but are better than older versions
\item OpenBSD has shown the way in many cases\\ \link{http://www.openbsd.org/papers/}
\end{list1}

\vskip 1cm

\centerline{Always try to make life worse and more costly for attackers}

\slide{Defense in depth - flere lag af sikkerhed}

\hlkimage{8cm}{security-layers-1-uk.pdf}

\centerline{\hlkbig\color{solido-orange} Defense using multiple layers is stronger!}

\slide{Are passwords dead?}

google: passwords are dead\\
About 6,580,000 results (0.22 seconds) 


Can we stop using passwords?

Muffett on Passwords has a long list of password related information, from the author of crack \link{http://en.wikipedia.org/wiki/Crack_(password_software)}

\link{http://dropsafe.crypticide.com/muffett-passwords}

\slide{Storing passwords}

\hlkimage{20cm}{images/passwordsafe-yubico.png}


\begin{list1}
\item PasswordSafe \link{http://passwordsafe.sourceforge.net/}
\item Apple Keychain provides an encrypted storage
\item Browsere, Firefox Master Password, Chrome passwords, ... who do YOU trust
\end{list1}


\slide{Google looks to ditch passwords for good}

\hlkimage{10cm}{yubico-neo-v1-454x284.jpg}

"Google is currently running a pilot that uses a YubiKey cryptographic card developed by Yubico

The YubiKey NEO can be tapped on an NFC-enabled smartphone, which reads an encrypted one-time password emitted from the key fob."

{\footnotesize Source:
\link{http://www.zdnet.com/google-looks-to-ditch-passwords-for-good-with-nfc-based-replacement-7000010073/}
}

\slide{Yubico Yubikey}

\hlkimage{20cm}{yubico-overview.png}
\begin{quote}
A Yubico OTP is unique sequence of characters generated every time the YubiKey button is touched. The Yubico OTP is comprised of a sequence of 32 Modhex characters representing information encrypted with a 128 bit AES-128 key
\end{quote}

\link{http://www.yubico.com/products/yubikey-hardware/}

\slide{Duosecurity}

\hlkimage{12cm}{duosecurity-overview.png}
Video
\link{https://www.duosecurity.com/duo-push}

\link{https://www.duosecurity.com/}

\slide{Low tech 2-step verification }

\centerline{\Large Printing code on paper, low level pragmatic }

\hlkimage{9cm}{google-backup-codes.png}

\begin{list1}
\item Login from new devices today often requires two-factor - email sent to user
\item Google 2-factor auth. SMS with backup codes
\item Also read about S/KEY developed at Bellcore {\bf in the late 1980s}\\ \link{http://en.wikipedia.org/wiki/S/KEY}
\end{list1}

\centerline{Conclusion passwords: integrate with authentication, not reinvent}

\slide{Integrate or develop?}

From previous slide:\\
\centerline{Conclusion passwords: integrate with authentication, not reinvent}


\begin{list1}
\item Dont:
\begin{list2}
\item Reinvent the wheel - too many times, unless you can maintain it afterwards
\item Never invent cryptography yourself
\item No copy paste of functionality, harder to maintain in the future
\end{list2}
\item Do:
\begin{list2}
\item Integrate with existing solutions
\item Use existing well-tested code: cryptography, authentication, hashing
\item Centralize security in your code
\item Fine to hide which authentication framework is being used, easy to replace later
\end{list2}
\end{list1}

\slide{Balanced security}

\hlkimage{21cm}{afbalanceret-sikkerhed.pdf}

\begin{list1}
\item Better to have the same level of security
\item If you have bad security in some part - guess where attackers will end up
\item Hackers are not required to take the hardest path into the network
\item Realize there is no such thing as 100\% security 
\end{list1}



\slide{Work together}

\hlkimage{18cm}{Shaking-hands_web.jpg}

\begin{list1}
\item Team up!
\item We need to share security information freely
\item We often face the same threats, so we can work on solving these together
\end{list1}


\slide{Cisco IOS password}

\begin{quote}
Title: Cisco's new password hashing scheme easily cracked\\

Description: In an astonishing decision that has left crytographic
experts scratching their heads, engineer's for Cisco's IOS operating
system chose to switch to a {\bf one-time SHA256 encoding - without salt} -
for storing passwords on the device. This decision leaves password
hashes vulnerable to high-speed cracking - modern graphics cards can
compute over {\bf 2 billion SHA256 hashes in a second - and is actually
considerably less secure than Cisco's previous implementation.} As users
cannot downgrade their version of IOS without a complete reinstall, and
no fix is yet available, security experts are urging users to avoid
upgrades to IOS version 15 at this time.
\end{quote}

Reference: via SANS @RISK newsletter\\
\link{http://arstechnica.com/security/2013/03/cisco-switches-to-weaker-hashing-scheme-passwords-cracked-wide-open/}






\slide{buffer overflows is a C problem}

\begin{list1}
\item {\bfseries Et buffer overflow}
er det der sker n�r man skriver flere data end der er afsat plads til
i en buffer, et dataomr�de. Typisk vil programmet g� ned, men i visse
tilf�lde kan en angriber overskrive returadresser for funktionskald og
overtage kontrollen. 
\item {\bfseries Stack protection} 
er et udtryk for de systemer der ved hj�lp af operativsystemer,
programbiblioteker og lign. beskytter stakken med returadresser og
andre variable mod overskrivning gennem buffer overflows. StackGuard
og Propolice er nogle af de mest kendte.
\end{list1}


\slide{Buffer og stacks}

\hlkimage{20cm}{buffer-overflow-1.pdf}

\begin{alltt}
main(int argc, char **argv)
\{      char buf[200];
        strcpy(buf, argv[1]); 
        printf("%s\textbackslash{}n",buf);
\}
\end{alltt}


\slide{Overflow - segmentation fault }

\hlkimage{20cm}{buffer-overflow-2.pdf}


\begin{list1}
\item Bad function overwrites return value!
\item Control return address
\item Run shellcode from buffer, or from other place
\end{list1}


\slide{Exploits - udnyttelse af s�rbarheder}

\begin{list1}
\item exploit/exploitprogram er
\begin{list2}
\item udnytter eller demonstrerer en s�rbarhed
\item rettet mod et specifikt system.
\item kan v�re 5 linier eller flere sider
\item Meget ofte Perl eller et C program 
\end{list2}
\end{list1}


\slide{Exploits}

\vskip 1 cm

\begin{alltt}
$buffer = ""; 
$null = "\textbackslash{}x00"; \pause
$nop = "\textbackslash{}x90"; 
$nopsize = 1; \pause
$len = 201; // what is needed to overflow, maybe 201, maybe more!
$the_shell_pointer = 0xdeadbeef; // address where shellcode is 
# Fill buffer
for ($i = 1; $i < $len;$i += $nopsize) \{
    $buffer .= $nop;
\}\pause
$address = pack('l', $the_shell_pointer);
$buffer .= $address;\pause
exec "$program", "$buffer";
\end{alltt}
\vskip 1 cm
\centerline{Demo exploit in Perl}
%Eksempel p� webserver buffer overflow, nosejob?

\slide{Privilegier least privilege}

\begin{list1}
\item Hvorfor afvikle applikationer med administrationsrettigheder -
  hvis der kun skal l�ses fra eksempelvis en database?
\item {\bfseries least privilege} 
betyder at man afvikler kode med det mest
restriktive s�t af privileger - kun lige nok til at
opgaven kan udf�res
\item Dette praktiseres ikke i webl�sninger i Danmark - eller meget f� steder
\end{list1}

\slide{Privilegier privilege escalation}
\begin{list1}
\item {\bfseries privilege escalation} er n�r man p� en eller anden vis
opn�r h�jere privileger p� et system, eksempelvis som
f�lge af fejl i programmer der afvikles med h�jere
privilegier. Derfor HTTPD servere p� UNIX afvikles som
nobody - ingen specielle rettigheder.
\item En angriber der kan afvikle vilk�rlige kommandoer kan ofte finde
  en s�rbarhed som kan udnyttes lokalt - f� rettigheder = lille skade
\end{list1}


\slide{local vs. remote exploits}

\begin{list1} 
\item {\bfseries local vs. remote}
angiver om et exploit er rettet mod
en s�rbarhed lokalt p� maskinen, eksempelvis
opn� h�jere privilegier, eller beregnet
til at udnytter s�rbarheder over netv�rk
\item {\bfseries remote root exploit} 
- den type man frygter mest, idet
det er et exploit program der n�r det afvikles giver
angriberen fuld kontrol, root user er administrator
p� UNIX, over netv�rket. 
\item {\bfseries zero-day exploits} dem som ikke offentligg�res - dem
  som hackere holder for sig selv. Dag 0 henviser til at ingen kender
  til dem f�r de offentligg�res og ofte er der umiddelbart ingen
  rettelser til de s�rbarheder
\end{list1}




\slide{Code quality}

\begin{list1}
\item Why are programs still insecure?
\item {\bf Programs are complex!}
\item Try implementing tools to improve quality
\item Hudson Extensible continuous integration server \link{http://hudson-ci.org/}
\item Sonar \link{http://www.sonarsource.org/}
\item Yasca can scan source code written in Java, C/C++, HTML, JavaScript, ASP, ColdFusion, PHP, COBOL, .NET, and other languages. Yasca can integrate easily with other tools\\
\link{http://www.scovetta.com/yasca.html}

\item Software analysis can help\\
\link{http://samate.nist.gov/index.php/Source_Code_Security_Analyzers.html}
\end{list1}

NB: you still have to think \smiley




\slide{OWASP top ten}

\hlkimage{16cm}{owasp.jpg}

\begin{quote}
The OWASP Top Ten provides a minimum standard for web application
security. The OWASP Top Ten represents a broad consensus about what
the most critical web application security flaws are.  
\end{quote}

\begin{list1}
\item The Open Web Application Security Project (OWASP) 
\item OWASP har gennem flere �r udgivet en liste over de 10 vigtigste
  sikkerhedsproblemer for webapplikationer  
\item \link{http://www.owasp.org}
\end{list1}


\slide{Change management}

\begin{list1}
\item Do people have focus on software in production
\item Can you re-install a server quickly, easily
\item Making changes to production systems
\item Fall back plan when updating that production database live
\item Good system administrators are hard to come by
\end{list1}

\slide{CWE Common Weakness Enumeration}

\hlkimage{22cm}{cwe-mitre-org.png}
\link{http://cwe.mitre.org/}

\slide{CWE/SANS Monster mitigations}

\hlkimage{18cm}{cwe-monster-mitigations.png}


Source:
\link{http://cwe.mitre.org/top25/index.html}

\slide{Deadly sins bogen}

\hlkimage{8cm}{24-deadly.jpg}

\begin{list1}
\item \emph{24 Deadly Sins of Software Security} 
Michael Howard, David LeBlanc, John Viega 2. udgave, f�rste hed 19 Deadly Sins
\end{list1}

\slide{Deadly sins bogen - close up}

\hlkimage{16cm}{24-deadly-sins-close-up.png}

\slide{Deadly Sins 1/2}

\begin{list1}
\item Part I Web Application Sins 1-4
\begin{list2}
\item 1) SQL Injection
\item 2) Web Server-Related Vulnerabilities
\item 3) Web Client-Related Vulnerabilities (XSS)
\item 4) Use of Magic URLs, Predictable Cookies, and Hidden Form Fields
\end{list2}
\item Part II Implementation Sins 5-18\\
5) Buffer Overruns, 6) Format String, 7) Integer Overflows, 8) C++ Catastrophes, 9) Catching Exceptions, 10) Command Injection
11) Failure to Handle Errors Correctly
12) Information Leakage
13) Race Conditions
14) Poor Usability
15) Not Updating Easily
16) Executing Code with Too Much Privilege
17) Failure to Protect Stored Data
18) The Sins of Mobile Code
\end{list1}

\vskip 2cm
\centerline{\Large Still want to program in C?}

\slide{Deadly Sins 2/2}

\begin{list1}
\item Part III Cryptographic Sins 19-21
\begin{list2}
\item 19) Use of Weak Password-Based System
\item 20) Weak Random Numbers
\item 21) Using Cryptography Incorrectly
\end{list2}
\item Part IV Networking Sins 22-24
\begin{list2}
\item 22) Failing to Protect Network Traffic, 
\item 23) Improper use of PKI, Especially SSL, 
\item 24) Trusting Network Name Resolution
\end{list2}
\end{list1}



\slide{Create your own exploits and spearphishing?}


\begin{list1}
\item Metasploit 
 Still rocking the internet\\
\link{http://www.metasploit.com/}
\item Armitage GUI fast and easy hacking for Metasploit\\
\link{http://www.fastandeasyhacking.com/}
\item Metasploit Unleashed\\
\link{http://www.offensive-security.com/metasploit-unleashed/Main_Page}
\item Social-Engineer Toolkit\\
\link{https://www.trustedsec.com/downloads/social-engineer-toolkit/}
\vskip 1cm
\item You can get these easily on \link{http://www.kali.org}
\end{list1}

Kilde:\\
{\small \link{http://www.metasploit.com/redmine/projects/framework/wiki/Release_Notes_360}}


\slide{Kali Linux the new backtrack}

\hlkimage{\linewidth-2cm}{kali-linux.png}

\begin{list1}
\item BackTrack \link{http://www.backtrack-linux.org}
\item  Kali \link{http://www.kali.org/}
\end{list1}




\slide{The Exploit Database - dagens buffer overflow}

\hlkimage{20cm}{exploit-db.png}

\centerline{\link{http://www.exploit-db.com/}}



\slide{We must allow open hacker tools}

\begin{list1}
\item I 1993 skrev Dan Farmer og Wietse Venema artiklen\\
\emph{Improving the Security of Your Site by Breaking Into it}
\item I 1995 udgav de softwarepakken SATAN\\
\emph{Security Administrator Tool for Analyzing Networks}
\begin{quote}\large
We realize that SATAN is a two-edged sword - like many tools,\\
it can be used for good and for evil purposes. We also \\
realize that intruders (including wannabees) have much \\
more capable (read intrusive) tools than offered with SATAN. 
\end{quote}
\item Se \link{http://sectools.org} og \link{http://www.packetstormsecurity.org/}

\end{list1}
Kilde:
\link{http://www.fish2.com/security/admin-guide-to-cracking.html}




\slide{Twitter news}

%\hlkimage{18cm}{twitter-security-feed.png}
\hlkimage{10cm}{twitter-safety.png}

\vskip 1cm
\centerline{Twitter is one of the fastest newsfeeds in the world}


\slide{How to become secure} 

\begin{list1}
\item Dont use computers at all, data about you is still processed by computers :-(
\item Dont use a single device for all types of data
\item Dont use a single server for all types of data, mail server != web server
\item Configure systems to be secure by default, or change defaults
\item Use secure protocols and VPN solutions
\item Some advice can be found in these places
\begin{list2}
\item \link{http://csrc.nist.gov/publications/PubsSPs.html}
\item \link{http://www.nsa.gov/research/publications/index.shtml}
\item \link{http://www.nsa.gov/ia/guidance/security_configuration_guides/index.shtml}
\end{list2}
\end{list1}


\slide{Checklisten}

\begin{list2}
\item BIOS kodeord, lock-codes for mobile devices
\item Firewall - specifically for laptops
\item Two browser strategy, one with paranoid settings
\item Use OpenPGP for email
\item Use a password safe for storing passwords
\item Use hard drive encryption
\item Keep systems updated
\item Backup your data
\item Dispose of data securely
\end{list2}

\slide{Be careful - questions?}

\hlkimage{5cm}{michael-conrad.jpg}
\centerline{\Large Hey, Lets be careful out there!}
\vskip 2 cm

\begin{center}
\myname

%\myweb
\end{center}

\vskip 2cm
Source: Michael Conrad \link{http://www.hillstreetblues.tv/}



\end{document}
\input{references.tex}

