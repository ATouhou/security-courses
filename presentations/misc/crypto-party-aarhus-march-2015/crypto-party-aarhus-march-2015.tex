\documentclass[20pt,landscape,a4paper,footrule]{foils}
\usepackage{crypto-slides}
%\usepackage{solido-network-slides}
\usepackage{pdf14}
%\usepackage{ulem}

% Basic things that we need are below
\selectlanguage{danish}

%\externaldocument{unix-audit-security-oevelser}
\externaldocument{\jobname-exercises}

%\slide{Pause}
%Er det tid til en lille pause?
%\hlkimage{15cm}{300px-Fozziecurtain.JPG}

\begin{document}

\mytitlepage{Cryptoparty hvad er det?}{}

%Key words:

\slide{agenda}

Planen for Cryptoparty 20:30 - 22:00
\begin{list2}
\item Introduktion Peter Kofoed
\item Kort introduktion cryptoparty - denne præsentation
\item Minicryptoparty 10min
\item Cryptoparty opdeling i grupper, Tor, PGP, OTR
\item OTR bliver ved projektor med Alexander Færøy, @ahfaeroey
\end{list2}

\slide{Passende paranoia}

\hlkimage{18cm}{GOOGLE-CLOUD-EXPLOITATION1383148810.jpg}
\centerline{SSL (encryption) added and removed here}



\slide{Solidaritetskryptering}

Hvorfor skal vi kryptere?

\begin{alltt}
       Køn
                       Seksualitet

 Tro religion       hatecrimes

 Politisk overbevisning, eller blot aktiv

 Whistleblowers             soldater      diplomater

\end{alltt}

\centerline{Du bestemmer ikke hvem der diskrimineres eller trues i andre lande}

\vskip2cm

Når vi krypterer hjælper vi andre! {\bf Solidaritetskryptering}


\slide{Kryptografi}

\hlkimage{18cm}{images/crypto-rot13.pdf}

\begin{list1}
\item Kryptografi er læren om, hvordan man kan kryptere data
\item Kryptografi benytter algoritmer som sammen med nøgler giver en
  ciffertekst - der kun kan læses ved hjælp af den tilhørende nøgle
\end{list1}

\slide{Public key kryptografi - 1}

\hlkimage{18cm}{images/crypto-public-key.pdf}

\begin{list1}
\item privat-nøgle kryptografi (eksempelvis AES) benyttes den samme
  nøgle til kryptering og dekryptering
\item offentlig-nøgle kryptografi (eksempelvis RSA) benytter to
  separate nøgler til kryptering og dekryptering
\end{list1}

\slide{Public key kryptografi - 2}

\hlkimage{18cm}{images/crypto-public-key-2.pdf}

\begin{list1}
\item offentlig-nøgle kryptografi (eksempelvis RSA) bruger den private
  nøgle til at dekryptere
\item man kan ligeledes bruge offentlig-nøgle kryptografi til at
  signere dokumenter\\ - som så verificeres med den offentlige nøgle
\end{list1}


\slide{Email er usikkert}

\hlkimage{20cm}{email-uden-krypterin.png}

\centerline{Email uden kryptering - er som et postkort}



\slide{Email med kryptering - afsendelse}

\hlkimage{18cm}{email-med-kryptering.png}


\centerline{En sikker krypteret email er ikke sværere at sende}

\slide{Krypteret Email under transporten}

\hlkimage{11cm}{modtaget-email-med-kryptering.png}

\centerline{En sikker krypteret email er beskyttet undervejs}

\slide{Kryptering: Cryptography Engineering}

\hlkimage{8cm}{book-ce-150w.jpg}

\emph{Cryptography Engineering} by
Niels Ferguson, Bruce Schneier, and Tadayoshi Kohno
\link{https://www.schneier.com/book-ce.html}

\centerline{Kryptering sikrer fortrolighed og integritet af beskederne}

\slide{Tor project anonym webbrowsing}

\hlkimage{21cm}{tor-project.png}

\centerline{\link{https://www.torproject.org/}}

\centerline{Der findes alternativer, men Tor er mest kendt}

\slide{Tor project - how it works 1}

\hlkimage{21cm}{how-tor-works-1.png}

\centerline{pictures from \link{https://www.torproject.org/about/overview.html.en}}

\slide{Tor project - how it works 2}

\hlkimage{21cm}{how-tor-works-2.png}

\centerline{pictures from \link{https://www.torproject.org/about/overview.html.en}}

\slide{Tor project - how it works 3}

\hlkimage{21cm}{how-tor-works-3.png}

\centerline{pictures from \link{https://www.torproject.org/about/overview.html.en}}

\slide{Tor project install}

\hlkimage{12cm}{tor-project.png}

Der findes diverse tools til Tor, Torbutton on/off knap til Firefox osv.

Det anbefales at bruge Torbrowser bundles fra \link{https://www.torproject.org/}

\slide{Torbrowser - outdated}

\hlkimage{20cm}{torbrowser-outdated.png}

\centerline{\color{red}Hov den mangler opdatering!}

\slide{Torbrowser - anonym browser}

\hlkimage{20cm}{torbrowser-main-window.png}

\centerline{\color{titlecolor} Mere anonym browser - Firefox i forklædning}


\slide{Torbrowser - sample site}

\hlkimage{18cm}{sample-tor-site.png}

\centerline{\color{titlecolor} .onion er Tor adresser - hidden sites}

\footnotesize{Den viste side er SecureDrop hos Radio24syv \link{http://www.radio24syv.dk/dig-og-radio24syv/securedrop/}}



\slide{Whonix - Tor to the max!}

\hlkimage{17cm}{400px-Whonix.jpg}

\begin{quote}
Whonix is an operating system focused on anonymity, privacy and security. It's based on the Tor anonymity network[5], Debian GNU/Linux[6] and security by isolation. DNS leaks are impossible, and not even malware with root privileges can find out the user's real IP. \link{https://www.whonix.org/}

\end{quote}

\centerline{Torbrowser er godt, Whonix giver lidt ekstra sikkerhed}



\slide{Hvad er et cryptoparty}
\hlkimage{6cm}{crypto-party-logo.png}


% Husk tools fra Frejas dokument
Iaften vil vi fokusere på disse:
\begin{list2}
\item OTR Off-the-record Adium eller Pidgin, brug server cloak.dk
\item Torproject - Tor Browser Bundle
\item OpenPGP - PGP/GPG Thunderbird Enigmail eller GPGmail Mac
\end{list2}

Andre kilder til tools:
\begin{list2}
\item Surveillance Self-Defense EFF guide
\link{https://ssd.eff.org/}
\item Se citizenfour filmen - fik velfortjent Oscar!\\ {\footnotesize\link{http://www.wired.com/2014/10/laura-poitras-crypto-tools-made-snowden-film-possible/}}
\item Information Security for Journalists\\
\link{http://www.tcij.org/resources/handbooks/infosec}
\end{list2}

\slide{10minute cryptoparty}

\hlkimage{24cm}{textsecure-redphone.pdf}

Forsøg, hvor mange kan kommunikere sikkert indenfor 10min?

\begin{list2}
\item Android installer TextSecure og Redphone
\item iPhone IOS installer Signal
\end{list2}

\vskip 1cm
\centerline{Send krypteret SMS til en anden herinde}

og brug så krypteret SMS fremover \smiley

\slide{Full Disk Encryption Mac OS X}

\hlkimage{15cm}{apple-filevault-enabled.png}

\centerline{Indbygget, gratis, stærk - når I kommer hjem}

\slide{DNS censur i Danmark}

\hlkimage{6cm}{Censored_rubber_stamp.png}

Hvis du er træt af den danske censur på DNS, så kan du skifte til at bruge:
Censurfridns.dk UncensoredDNS

Du udskifter blot dine DNS indstillinger på din PC til:
\begin{list2}
\item anycast.censurfridns.dk / 91.239.100.100 / 2001:67c:28a4::
\item ns1.censurfridns.dk / 89.233.43.71 / 2002:d596:2a92:1:71:53::
\end{list2}
Se også \link{http://www.censurfridns.dk} og \link{blog.censurfridns.dk} for mere info.

\vskip 2cm

\centerline{\Large Det er uacceptabelt at pille ved DNS - punktum!}


% The end
\myquestionspage

\end{document}
\input{references.tex}
