\documentclass[20pt,landscape,a4paper,footrule]{foils}
\usepackage{solido-network-slides}
\begin{document}

\selectlanguage{danish}

\mytitlepage
{IT-sikkerhed 2015}{PROSA Superhelteseminar}

% Spor 2: I dybden med tekniske udfordringer og it-sikkerhed, oplæg v. Henrik Kramshøj

% De to slots bliver forbundet med emnerne, så der er kobling mellem good/bad, sikkerhed/hack, skidt/godt
%
% Indenfor internetteknologier, internet, internetsikkerhed, hacking, tools,
% sniffing, osv. - der arbejdes stadig på indholdet mere specifikt


\vskip 2cm
\centerline{\tiny Slides are available as PDF, kramshoej@Github}


\slide{Goals of today}
\hlkimage{12cm}{government-doing-nothing-wrong.jpg}

\begin{list1}
\item How to become a super hero at work
\item Offer input to what things to look into
\item Hodge-podge of security related things - inspiration
\item Please give feedback and join me in discussions, dialogue \smiley
\end{list1}

\slide{Plan for today}

\hlkimage{10cm}{Shaking-hands_web.jpg}

\begin{list1}
\item Kl 17:30-19:30 with a break
\item Less presentation, more talk
\item Less me talking (only) and more 2.0 social media interaction
\end{list1}

\centerline{Trying to fit in demo and workshop-like stuff}

\slide{Generic advice}

Recommendations \hlkrightimage{8cm}{Encrypt_all_the_things.png}
\begin{list2}
\item Lock your devices, phones, tables and computers
\item Update software and apps
\item Do NOT use the same password everywhere
\item Watch out when using open wifi-networks
\item Multiple browsers: one for Facebook, one for banking apps?
\item Multiple laptops? One for private data, one for work?
\item Think of the data you produce - where is it stored
\item Use pseudonyms and aliases, do not use your real name everywhere
\item Enable encryption: IMAP{\bf S}, POP3{\bf S},
  HTTP{\bf S} and full disk encryption
\item Use Tor \link{http://torproject.org/}
\end{list2}


\slide{The current situation}

\hlkimage{10cm}{homer-end-is-near.jpg}
\begin{list1}
\item Internet security sucks
\item Personal computers like laptops suck at security
\item Mobile devices suck even more at security - less CPU/MEM/storage
\item We depend on cloud services and underfunded infrastructure - OpenSSL
\item We depend on others and the whole internet - DDoS
\end{list1}


\slide{Goals: Internet Ninjas}

\hlkimage{10cm}{ninjas.png}
\begin{list1}
\item Real super heroes are just ninjas
\item By knowing the internet, technologies and possibilities
\item Using technology and knowledge make it seem magical
\item In reality preparedness and defense in depth go a looooong way
\item Common sense is not magic, structured methods are king
\end{list1}

\slide{Challenges}

\begin{list1}
\item Less resources available for IT and infosec
\item Lots of new malware, virus, vulnerabilities and hacking
\item Dataloss ransomware, theft
\item Loss of confidentiality, 2014: 700 million lost accounts
\item Infosec charlatans, hype and lies
\end{list1}

\vskip 2cm
\centerline{Your boss wants: No cost, and please show us great results}

\slide{Solutions}

\begin{list1}
\item Automate your job, Ansible is our poison demo
\item Backup your life, help others backup, Duplicity is my choice
\item \link{http://ssd.eff.org} Learn self-defense for yourself, practice infosec war
\item Use hackertools to detect and identify
\item Categories, sort, prioritize, group problems - solve more
\item Measure, collect and present - make it pretty
\item Learn from devops, Elasticsearch Logstash Kibana D3.js
\end{list1}


\centerline{Use your brain}

A lot will seem easy and basic from the outside, but when you are knee-deep\\
in something you loose focus. Take a step back once in a while.

\slide{Case: Aalborg Farve og Lak.}

\begin{quote}
"Vi skulle alligevel have nyt Navision-system i maj, så vi måtte fremrykke den investering. På den måde kunne vi få tastet alt ind i det nye system. I hele sagen har vi dog tabt omkring en million kroner med de mistede ordrer, ny software og revisionsbistand,"
\end{quote}
Medejer og salgs- og personaleansvarlig hos Aalborg Farve- og Lak, Pernille Skall

\begin{list1}
\item Break-in through Windows Xp
\item Ransomware infection - across multiple systems
\item Latest backup from November (currently we are in April!)
\item Great that they share
\item Todays break-ins use yesterdays vulns, repeated and documented multiple times
\end{list1}


{\small\link{http://www.computerworld.dk/art/233684/hacker-kom-ind-via-labelprinter-tog-dansk-firmas-it-systemer-som-gidsel}}

\slide{Hackertools are for everyone!}

\hlkimage{2cm}{hackers_JOLIE+1995.jpg}


\begin{list2}
\item Hackers work all the time to break stuff, Use hackertools:
\item Nmap, Nping \link{http://nmap.org}
\item Wireshark - \link{http://www.wireshark.org/}
\item Aircrack-ng \link{http://www.aircrack-ng.org/}
\item Metasploit Framework \link{http://www.metasploit.com/}
\item Burpsuite \link{http://portswigger.net/burp/}
\item Skipfish \link{http://code.google.com/p/skipfish/}
\item Kali Linux \link{http://www.kali.org}
\end{list2}

\vskip 5mm
\centerline{Most popular hacker tools \link{http://sectools.org/}}

\slide{Kali Linux the pentest toolbox}

\hlkimage{\linewidth-8cm}{kali-linux.png}

\begin{list1}
\item  Kali \link{http://www.kali.org/}
\item 100.000s of videos on youtube
\item Also versions for Raspberry Pi, mobile and other small computers
\end{list1}


\slide{Metasploit and Armitage Still rocking the internet}


%\hlkimage{20cm}{metasploit-about.png}
\hlkimage{10cm}{armitage-overview.png}

\begin{list1}
\item \link{http://www.metasploit.com/}
\item Armitage GUI fast and easy hacking for Metasploit\\
\link{http://www.fastandeasyhacking.com/}
\item Recommened training Metasploit Unleashed\\
\link{http://www.offensive-security.com/metasploit-unleashed/Main_Page}
%\item Bog: Metasploit: The Penetration Tester's Guide, No Starch Press\\
%ISBN-10: 159327288X
\end{list1}



\slide{Big data tools - examples}

\hlkimage{18cm}{kibana-solido.png}
\begin{list1}
\item Net: Bro \link{http://www.bro-ids.org} Suricata \link{http://suricata-ids.org}
\item DNS: DSC and PacketQ \link{https://github.com/dotse/packetq/wiki}
\item Syslog: Elasticsearch, Logstash, and Kibana
\end{list1}
\centerline{Collect and present data more easily - non-programmers}


\slide{Security devops}

\begin{list1}
\item We need devops skillz in security - automate, security is also big data
\item integrate tools, transfer, sort, search, pattern matching, statistics, ...
\item tools, languages, databases, protocols, data formats
\item Example introductions:
\begin{list2}
\item Seven languages/database/web frameworks in Seven Weeks
\item Elasticsearch the definitive guide\\
\link{http://www.elastic.co/guide/en/elasticsearch/guide/current/index.html}
\item \link{https://www.elastic.co/products/kibana}
\item \link{https://www.elastic.co/products/logstash}
\end{list2}
\end{list1}

\centerline{We are all Devops now, even security people!}

Do you even Github? \smiley \link{https://github.com/stars}

\slide{Network Security Through Data Analysis}

\hlkimage{8cm}{network-security-through-data-analysis.png}

Low page count, but high value! Recommended.

Network Security Through Data Analysis: Building Situational Awareness\\
By Michael Collins\\
Publisher: O'Reilly Media
Released: February 2014 Pages: 348



\slide{BRO IDS}

\hlkimage{14cm}{bro-ids.png}

\begin{quote}
While focusing on network security monitoring, Bro provides a comprehensive platform for more general network traffic analysis as well. Well grounded in more than 15 years of research, Bro has successfully bridged the traditional gap between academia and operations since its inception.
\end{quote}

\link{https://www.bro.org/}

\slide{BRO more than an IDS}

\begin{quote}
	The key point that helped me understand was the explanation that Bro is a
               domain-specific language for networking applications and that Bro-IDS
               (http://bro-ids.org/) is an application written with Bro.
\end{quote}

Why I think you should try Bro\\
\link{https://isc.sans.edu/diary.html?storyid=15259}\\

\slide{Bro scripts}

\begin{alltt}\small
global dns_A_reply_count=0;
global dns_AAAA_reply_count=0;
...
event dns_A_reply(c: connection, msg: dns_msg, ans: dns_answer, a: addr)
	\{
	++dns_A_reply_count;
	\}

event dns_AAAA_reply(c: connection, msg: dns_msg, ans: dns_answer, a: addr)
	\{
	++dns_AAAA_reply_count;
	\}
\end{alltt}

Source: dns-fire-count.bro from\\
{\small \link{https://github.com/LiamRandall/bro-scripts/tree/master/fire-scripts}}



\slide{Security Onion}

\hlkimage{10cm}{security-onion.png}
\begin{list2}
\item Security Onion is a Linux distro for IDS, NSM, and log management
\item \link{http://securityonion.blogspot.dk}
\item \link{http://blog.securityonion.net/p/securityonion.html}
\end{list2}

\centerline{Nice starting point for researching dashboards/network packets}

\slide{February and March 2015, Security Onion updates}

\begin{list2}
\item Security Onion 12.04.5.1 ISO image now available
\item Suricata IDS engine 2.0.7 updated packages for SO released
\item Learn NSM with Security Onion today - its free
\end{list2}
{\small\link{http://blog.securityonion.net/2015/03/suricata-207.html}\\
\link{http://blog.securityonion.net/2015/02/security-onion-120451-iso-image-now.html}}




\slide{Lets get to work!}

\begin{list2}
\item Get Kibana working
\item Get access to Kibana
\item Produce some data
\item Create dashboards
\end{list2}

While demoing Ansible, and vagrant

Lots of examples\\
\link{https://github.com/geerlingguy/ansible-vagrant-examples/}


\myquestionspage



%\slide{Books}

%D3.js, Interactive Data Visualization for the Web, og Network Security Through Data Analysis: Building Situational Awareness - den slags

%Stay safe


\end{document}
