\documentclass[20pt,landscape,a4paper,footrule]{foils}
\usepackage{solido-network-slides}
%\usepackage{ulem}

% Basic things that we need are below
\selectlanguage{danish}

%\externaldocument{unix-audit-security-oevelser}
\externaldocument{\jobname-exercises}

\begin{document}

% 11.15-11.45 V�rkt�jer til beskyttelse af internettrafik
% - Spor du s�tter p� internettet
% - Profilering
% - Brug af TOR
% v. Direkt�r Henrik Kramsh�j, Solido Networks ApS

\mytitlepage
{V�rkt�jer til beskyttelse af internettrafik}{Beskyt dig selv}

\slide{Democracy now: Why do we bother?}

\hlkimage{22cm}{jacob-appelbaum.png}
{\large
In a democracy we need the citizens with freedom that can act without constant surveillance

Democracy requires that we can actively select which personal data to give up and to whom
}


\vskip 2 cm
\centerline{\large Cryptography is peaceful protest against blanket surveillance}

\slide{Why think of security?}

\hlkimage{8cm}{1984-not-instruction-manual.jpg}


\begin{quote}
	Privacy is necessary for an open society in the electronic age. Privacy is not secrecy. A private matter is something one doesn't want the whole world to know, but a secret matter is something one doesn't want anybody to know. Privacy is the power to selectively reveal oneself to the world. ~A Cypherpunk's Manifesto by Eric Hughes, 1993
\end{quote}

Copied from \link{https://cryptoparty.org/wiki/CryptoParty}


\slide{Spor du s�tter p� internettet}

\hlkimage{14cm}{panopticlick.png}
\begin{itemize}
\item Vi bruger allesammen browsere, email klienter og andre programmer
\item Mange programmer er nemme at genkende og mange sender i klartekst - uden kryptering
\end{itemize}

Source: \link{https://panopticlick.eff.org/}

\slide{Chrome}

\hlkimage{16cm}{panopticlick-chrome.png}

\centerline{Panopticlick Chrome plugins}

\slide{Safari}

\hlkimage{16cm}{panopticlick-safari.png}

\centerline{Panopticlick Safari plugins - {\bf Garmin plugins!}}

\slide{Profiling}

\begin{itemize}
\item Derudover cookies
\item IP-adressen du kommer fra og andre parametre fra TCP/IP - kan ofte validere/bestemme OS
\item Har du IPv6
\item Tidspunkterne for login
\item Skrivestil
\end{itemize}

\slide{Dive into the bitstream}

\hlkimage{19cm}{wireshark-http.png}

\centerline{Anyone in the network path can sniff this!}

\slide{Anonymous access to internet - Tor project}

\hlkimage{21cm}{tor-project.png}

\centerline{\link{https://www.torproject.org/}}

\vskip 2cm
\centerline{Der findes alternativer, men Tor er mest kendt}

\slide{Tor project - how it works 1}

\hlkimage{21cm}{how-tor-works-1.png}

\centerline{pictures from \link{https://www.torproject.org/about/overview.html.en}}

\slide{Tor project - how it works 2}

\hlkimage{21cm}{how-tor-works-2.png}

\centerline{pictures from \link{https://www.torproject.org/about/overview.html.en}}

\slide{Tor project - how it works 3}

\hlkimage{21cm}{how-tor-works-3.png}

\centerline{pictures from \link{https://www.torproject.org/about/overview.html.en}}


\slide{Brug af TOR - demo}

\hlkimage{22cm}{torbrowser-main-window.png}


\slide{Whonix - Tor to the max!}

\hlkimage{17cm}{400px-Whonix.jpg}

\begin{quote}
Whonix is an operating system focused on anonymity, privacy and security. It's based on the Tor anonymity network[5], Debian GNU/Linux[6] and security by isolation. DNS leaks are impossible, and not even malware with root privileges can find out the user's real IP. \link{https://www.whonix.org/}

\end{quote}

\centerline{Torbrowser er godt, Whonix giver lidt ekstra sikkerhed}


\slide{Secure your mobile}



\hlkimage{20cm}{the-guardian-project.pdf}

\centerline{Dont forget your mobile platforms \link{https://guardianproject.info/}}


\slide{Sp�rgsm�l?}


\vskip 4cm

\begin{center}
\hlkbig 

\myname

\myweb
\vskip 2 cm

I er altid velkomne til at sende sp�rgsm�l p� e-mail
\end{center}


\slide{VikingScan.org - free portscanning}

\hlkimage{18cm}{vikingscan.png}
%\vskip 1cm 
%\centerline{\link{http://www.vikingscan.org}}


\end{document}
\input{references.tex}

