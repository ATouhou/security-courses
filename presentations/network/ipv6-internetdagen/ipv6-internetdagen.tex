\documentclass[20pt,landscape,a4paper,footrule]{foils}
\usepackage{solido-network-slides}

\begin{document}
\selectlanguage{english}
\mytitlepage{IPv6 is here your fridge is on the network}


\vskip 2cm
\centerline{\footnotesize Slides are available as PDF}

\slide{Goal}
\hlkimage{5cm}{kame-noanime-small.png}

\begin{list1}
\item Introduce IPv6 
\item IPv6 addressing 
\item IPv4 vs IPv6 - Differences and similarities
\item The future is here
\item Denmark is falling behind on IPv6 
\item Ressources
\end{list1}

\centerline{Expect you to be administrators of IP networks}

\slide{Internet idag}

\hlkimage{14cm}{images/server-client.pdf}

\begin{list1}
\item Clients and servers
\item Rooted in academic networks
\item Protocols which are more than 20 years old
\item Very little encryption and security built into the network
\end{list1}

\slide{Internetworking: history}

\begin{list2}  
\item[1960s]  L. Kleinrock, MIT packet-switching theory,
 J. C. R. Licklider,MIT - notes , \\
Paul Baran: On Distributed Communications
\item[1969]  ARPANET 4 nodes
\item[1971]  14 nodes
\item[1973]  Design of Internet Protocols started
\item[1973]  Email is about 75\% of all ARPANET traffic
\item[1974]  TCP/IP: Cerf/Kahn: A protocol for Packet
        Network Interconnection
\item[1983]  EUUG $\rightarrow$ DKUUG/DIKU forbindelse
\item[1988]  About 60.000 systems on the internet - 
        The Morris Worm hits about 10\%
\item[2002]  Ialt ca. 130 millioner p� Internet
\item[2010] 1,966,514,816 users \link{http://www.internetworldstats.com/stats.htm}
\item[2010] {\bfseries IANA reserved blocks 8\% (March 2010) - \link{http://www.potaroo.net/tools/ipv4/}}
\end{list2}

\slide{Why IPv6}

\hlkimage{18cm}{ipv4-report-march.png}

\centerline{March 2010}
\centerline{\link{http://www.potaroo.net/tools/ipv4/}}

\slide{Why IPv6}

\hlkimage{18cm}{ipv4-report-sept.png}

\begin{center}
Updated September 2010\\
\link{http://www.potaroo.net/tools/ipv4/}

No more talk, we need IPv6, get to work - end of discussion
\end{center}
	
\slide{OSI \& Internet Protocols}

\hlkimage{14cm,angle=90}{images/compare-osi-ip.pdf}

\slide{IPv6: Internet redesigned? - no!}
 
\begin{list1}
\item Preserve the good stuff
\item back to basics, internet as it used to be!
\item fate sharing - connection rely on end points, not intermediary NAT boxes
\item end-to-end transparency - you have an address and I have an address
\item Wants: bandwidth +10G, low latency/predictable latency, Quality of Service, Security
\end{list1}

\vskip 5mm
\centerline{\color{titlecolor}\LARGE \bf IPv6 is evolution, not revolution}
\vskip 5mm

Note: IPv6 was not designed to solve all problems, so don't expect it to!




\slide{How to use IPv6}

\begin{center}
\vskip 3 cm
\hlkbig
www.solidonetworks.com

hlk@solidonetworks.com
\end{center}

\slide{Really how to use IPv6?}

\begin{list1}
\item Get IPv6 address and routing
\item Add AAAA (quad A) records to your DNS
\item Done
\end{list1}

\begin{alltt}
\LARGE
www     IN	A       91.102.95.20
        IN	AAAA    2a02:9d0:10::9
\end{alltt}

\slide{IPv4 header - RFC-791 September 1981}

\begin{alltt}
\footnotesize
      0                   1                   2                   3   
      0 1 2 3 4 5 6 7 8 9 0 1 2 3 4 5 6 7 8 9 0 1 2 3 4 5 6 7 8 9 0 1 
     +-+-+-+-+-+-+-+-+-+-+-+-+-+-+-+-+-+-+-+-+-+-+-+-+-+-+-+-+-+-+-+-+
     |Version|  IHL  |Type of Service|          Total Length         |
     +-+-+-+-+-+-+-+-+-+-+-+-+-+-+-+-+-+-+-+-+-+-+-+-+-+-+-+-+-+-+-+-+
     |         Identification        |Flags|      Fragment Offset    |
     +-+-+-+-+-+-+-+-+-+-+-+-+-+-+-+-+-+-+-+-+-+-+-+-+-+-+-+-+-+-+-+-+
     |  Time to Live |    Protocol   |         Header Checksum       |
     +-+-+-+-+-+-+-+-+-+-+-+-+-+-+-+-+-+-+-+-+-+-+-+-+-+-+-+-+-+-+-+-+
     |                       Source Address                          |
     +-+-+-+-+-+-+-+-+-+-+-+-+-+-+-+-+-+-+-+-+-+-+-+-+-+-+-+-+-+-+-+-+
     |                    Destination Address                        |
     +-+-+-+-+-+-+-+-+-+-+-+-+-+-+-+-+-+-+-+-+-+-+-+-+-+-+-+-+-+-+-+-+
     |                    Options                    |    Padding    |
     +-+-+-+-+-+-+-+-+-+-+-+-+-+-+-+-+-+-+-+-+-+-+-+-+-+-+-+-+-+-+-+-+

                     Example Internet Datagram Header
\end{alltt}


\slide{IPv6 header - RFC-2460 December 1998}

\begin{alltt}
\footnotesize
      0                   1                   2                   3   
      0 1 2 3 4 5 6 7 8 9 0 1 2 3 4 5 6 7 8 9 0 1 2 3 4 5 6 7 8 9 0 1 
     +-+-+-+-+-+-+-+-+-+-+-+-+-+-+-+-+-+-+-+-+-+-+-+-+-+-+-+-+-+-+-+-+
     |Version| Traffic Class |           Flow Label                  |
     +-+-+-+-+-+-+-+-+-+-+-+-+-+-+-+-+-+-+-+-+-+-+-+-+-+-+-+-+-+-+-+-+
     |         Payload Length        |  Next Header  |   Hop Limit   |
     +-+-+-+-+-+-+-+-+-+-+-+-+-+-+-+-+-+-+-+-+-+-+-+-+-+-+-+-+-+-+-+-+
     |                                                               |
     +                                                               +
     |                                                               |
     +                         Source Address                        +
     |                                                               |
     +                                                               +
     |                                                               |
     +-+-+-+-+-+-+-+-+-+-+-+-+-+-+-+-+-+-+-+-+-+-+-+-+-+-+-+-+-+-+-+-+
     |                                                               |
     +                                                               +
     |                                                               |
     +                      Destination Address                      +
     |                                                               |
     +                                                               +
     |                                                               |
     +-+-+-+-+-+-+-+-+-+-+-+-+-+-+-+-+-+-+-+-+-+-+-+-+-+-+-+-+-+-+-+-+
\end{alltt}


%%%%%%%%%%%%%%%%%%%%%%%%%%%%%%%%%%%%%%%%%%%%%%%%%%%%%%%%%%%%%%%%%%%%%%%
\slide{IPv6 - extension headers RFC-2460}

\begin{list2}
\item Hop-by-Hop Options
\item Routing (Type 0)
\item Fragment - fragmentation only at end-points!
\item Destination Options
\item Authentication
\item Encapsulating Security Payload
\end{list2}


\slide{IPv6 addressing RFC-4291}

\begin{list1}
\item Addresses are always 128-bit identifiers for interfaces and sets of
   interfaces 
\item Unicast:   An identifier for a {\bf single interface}.\\
A packet sent to a
               unicast address is delivered to the interface identified
               by that address.
\item Anycast:   An identifier for a {\bf set of interfaces} (typically
               belonging to different nodes).\\  A packet sent to an
               anycast address is {\bf delivered to one} of the interfaces
               identified by that address (the "nearest" one, according
               to the routing protocols' measure of distance).

\item Multicast: An identifier for a {\bf set of interfaces} (typically
               belonging to different nodes). \\ A packet sent to a
               multicast address is {\bf delivered to all interfaces
               identified by that address}.
\end{list1}

\slide{IPv6 addressing RFC-4291, cont.}

\hlkimage{22cm}{ipv6-address-1.pdf}

\begin{list1}
\item 8 times 4 hex-digits seperated by colon x:x:x:x:x:x:x:x
\item Written as ipv6-address/prefix-length CIDR notation
\item Leading zeros can be removed
\item One or more groups of 16 bits of zeros can be replaced by ::
\end{list1}

Note: \link{http://en.wikipedia.org/wiki/Classless_Inter-Domain_Routing}

\slide{Examples:}
\begin{list2}
\item ABCD:EF01:2345:6789:ABCD:EF01:2345:6789

\item Adddress 2001:DB8:0:0:8:800:200C:417A
\item Address of loopback ::1
\item IPv6 prefix 2a02:09d0:95::1/64, subnet 2a02:09d0:0095:0000::/64
\item Address 2a02:09d0:95::1 or 2a02:09d0:0095:0000:0000:0000:0000:0001
\vskip 1 cm
\item Hint: use programming libraries to parse them :-) 
\end{list2}

\slide{Danish sites}
\begin{list1}
\item Name servers for .dk\\
p.nic.dk has IPv6 address 2001:500:14:6036:ad::1\\
s.nic.dk has IPv6 address 2a01:3f0:0:303::53\\
b.nic.dk has IPv6 address 2a01:630:0:80::53
\item ns1.gratisdns.dk has IPv6 address 2a02:9d0:3002:1::2
\item ns1.censurfridns.dk has IPv6 address 2002:d596:2a92:1:71:53::
\item www.solidonetworks.com has IPv6 address 2a02:9d0:10::9
\end{list1}


\slide{IPv6 in practice ipconfig/ifconfig and ping}

\begin{alltt}\small
$ ifconfig en0
en0: flags=8863<UP,BROADCAST,SMART,RUNNING,SIMPLEX,MULTICAST> mtu 1500
	inet6 {\bf fe80::216:cbff:feac:1d9f%en0} prefixlen 64 scopeid 0x4 
	inet 10.0.42.15 netmask 0xffffff00 broadcast 10.0.42.255
	inet6 {\bf 2001:16d8:dd0f:cf0f:216:cbff:feac:1d9f} prefixlen 64 autoconf 
	ether 00:16:cb:ac:1d:9f 
	media: autoselect (1000baseT <full-duplex>) status: active

$ ping6 ::1
PING6(56=40+8+8 bytes) ::1 --> ::1
16 bytes from ::1, icmp_seq=0 hlim=64 time=0.089 ms
16 bytes from ::1, icmp_seq=1 hlim=64 time=0.155 ms

$ traceroute6 2001:16d8:dd0f:cf0f::1
traceroute6 to 2001:16d8:dd0f:cf0f::1 (2001:16d8:dd0f:cf0f::1) 
from 2001:16d8:dd0f:cf0f:216:cbff:feac:1d9f, 64 hops max, 12 byte packets
 1  2001:16d8:dd0f:cf0f::1  0.399 ms  0.371 ms  0.294 ms
\end{alltt}
	


\slide{IPv6 autoconfiguration}

\hlkimage{24cm}{modified-eui64.pdf}

\begin{list1}
\item DHCPv6 is available, but {\bfseries stateless autoconfiguration} is king
\item Routers announce subnet prefix via {\bfseries router advertisements}
\item Individual nodes then combine this with their EUI64 identifier
\end{list1}

%\link{http://www.cisco.com/web/about/ac123/ac147/archived_issues/ipj_7-2/ipv6_autoconfig.html}

\slide{Router advertisement daemon}

\hlkimage{14cm}{ipv6-router-advertisement.png}



\slide{Getting connected}

\begin{list1}
\item Native IPv6 - available at some places in DK\\
Ask your provider - prepare to switch provider if no plan
\item Automatic tunnels 6to4, Teredo etc.
\begin{list2}
\item 6to4 benytter IPv4 infrastrukturen
\item Teredo sender IPv6 gennem IPv4/UDP pakker
\end{list2}
\item Configured tunnels and tunnelbrokers
\begin{list2}
\item \link{http://sixxs.net} IPv6 Deployment \& Tunnel Broker
\item \link{http://he.net} hurricane electric internet services
\end{list2}

\end{list1}

\vskip 2cm
\centerline{Notice: you probably already have IPv6 traffic in your network!}

\slide{Allocating IPv6 addresses} 

\begin{list1}
\item You have plenty!
\item Providers will typically get /32
\item Providers will typically give you /48 or /56
\item Your /48 can be used for:
\begin{list2}
\item 65536 subnets
\item Each subnet has $2^{64}$ addresses
\end{list2}
\end{list1}

\slide{The future is here}

What can we use IPv6 for?

\hlkimage{17cm}{df20030604.jpg}

\centerline{\small Source: Dr Fun 2003/06/04
The brave new world of IPv6 }

%\link{http://www.ibiblio.org/Dave/Dr-Fun/df200306/df20030604.jpg}}


\slide{Think!} 

You have a gazillion IPs what now?

\vskip 3 cm
\centerline{\Large Be creative. No limits!}

\pause
About 4 billion mobiles and 1 billion PCs\\ source Vincent Cert \link{http://www.youtube.com/watch?v=t9M0RPNr9qg}

\slide{Home automation} 

\begin{list1}
\item Putting your fridge on the internet, need more milk!
\item Report back to manufacturer, each different part has address, easier
\item Ping light2324.kitchen - still working?
\end{list1}

\slide{Internet sharing and always on}

\begin{list1}
\item Internet tethering to your friends, at home, at the bus, train\\
Each will get their own address - enables direct two-way communication
\item Mobile IPv6 - better than IPv4 and will be useful
\end{list1}


\slide{Sensors}


\begin{list1}
\item Sensors\\
Does your lawn need water and where?\\
Throw a bucket of sensor and let them figure it out
\item Pressure sensors\\
Measure the load on ships, containers, people, real life traffic 
\item Tracking devices\\
Busses, taxis, deliveries
\item Snow on a mountain\\
Spread sensors across a mountain and mesh network them, no problem
\item Ad-Hoc networks\\
6LoWPAN IPv6 over Low power Wireless Personal Area Networks
\item Intelligent Clothing - Wearable Electronics, Smart Clothes
\end{list1}

\slide{Sample idea, Biodevices Vital Jacket}

\hlkimage{12cm}{biodevices-vital-jacket.jpg}

\begin{quote}
	Biodevices brings us the Vital Jacket. This garment is used to monitor ECG waves and Heart rate levels. This can be used for sports, fitness, and medical purposes.
\end{quote}

\link{http://www.crunchwear.com/biodevices-vital-jacket/}


\slide{Smart IPv6 building}

Building automation
\begin{list2}
\item To reduce energy consumption by at least 25\%.
\item To ease the deployment and integration of building automation systems.
\item To manage access control and to improve security.
\item To provide innovative tools for meeting and conference rooms.
\item To develop innovative interfaces within the building (virtual assistant, etc.).
\item To enable individual environment customization by the users (temperature, light, music, etc.).
\item and more
\end{list2}

\link{http://www.smartipv6building.org/}

\slide{New applications}

\begin{list1}
\item Who would have guessed the applications?
\item World Wide Web 
\item World Wide chatting - MSN, IRC, Jabber etc.
\item Distribution of software - peer to peer
\item Facebook
\item Twittter
\item Foursquare
\item Whats next?
\item Smart internet devices + GPS + video + users = fun and business!
\item Sometimes named the Internet of Things
\end{list1}

\slide{IPv6 business case}


\begin{list2}
\item An almost unlimited scalability with a very large IPv6 address space ($2^128$ addresses), enabling IP addresses to each and every device.

\item Address self-configuration mechanisms, easing the deployment.

\item Improved security and authentication features, such as mandatory IPSec capacities and the possibility to use of the address space to include encryption keys.

\item Peer-to-peer connectivity, solving the NAT barrier with specific and permanent IP addresses for any device and/or user of the Internet.

\item Mobility features, enabling a seamless connexion when moving from one access point to another access point on the Internet.

\item Multi cast and any cast functionalities.

\item IPv6 will provide an easier remote interaction with each and every device with a {\bfseries direct integration to the Internet.} In other words, IPv6 will make possible to move from a network of servers, to a network of things.

\end{list2}

\centerline{ Business case for IPv6 is {\bf continuity}}


{\footnotesize Partial quote from http://www.smartipv6building.org/index.php/en/ipv6-potential}




\slide{IPv6 ripeness}

\hlkimage{28cm}{userfiles-v6-ripeness.png}

\centerline{IPv6 ripeness from \link{http://labs.ripe.net/}}

\slide{Curent status Denmark}

\begin{list1}
\item Too little interest - less than 100 people thinking about IPv6?
\item Some providers have some IPv6 connectivity
\item NO ISPs have IPv6 to consumers
\item NO ISPs market IPv6 as a product, except me perhaps :-)
\item Perceived NO NEEED
\vskip 2 cm
\pause
\item Free, a major French ISP rolled-out IPv6 at end of year 2007
\item XS4All As of August 2010 native IPv6 DSL connections became available to almost all their customers.
\end{list1}

Source: \link{http://en.wikipedia.org/wiki/IPv6_deployment}


\slide{Danish resources - get involved}

\hlkimage{10cm}{taskforce-logo.jpg}

\begin{center}
Danish IPv6 task force - unofficial\\
\link{http://www.ipv6tf.dk}
\end{center}

\slide{Conclusion}

\begin{center}
\vskip 5mm
{\color{titlecolor}\LARGE \bf IPv6 is here already - use it}
\vskip 5mm


\link{http://www.ipv6actnow.org/}

\link{http://digitaliser.dk/group/374895}

\link{http://www.ipv6tf.dk}
\end{center}
\vskip 1cm 

\myquestionspage

\slide{VikingScan.org - free portscanning}

\hlkimage{18cm}{vikingscan.png}
%\vskip 1cm 
%\centerline{\link{http://www.vikingscan.org}}

\slide{Referencer: netv�rksb�ger}

\begin{list2}
\item Stevens, Comer, 
\item Network Warrior
\item TCP/IP bogen p� dansk
\item KAME b�gerne
\item O'Reilly generelt IPv6 Essentials og IPv6 Network Administration  
\item O'Reilly cookbooks: Cisco, BIND og Apache HTTPD
\item Cisco Press og website
\item Firewall b�ger, Radia Perlman: IPsec, 
\end{list2}

\slide{B�ger om IPv6}

\begin{list1}
\item \emph{IPv6 Network Administration}
af David Malone og Niall Richard Murphy
 - god til real-life admins, typisk
O'Reilly bog
\item \emph{IPv6 Essentials} af Silvia Hagen, O'Reilly 2nd edition (May 17, 2006)
	god reference om emnet
\item \emph{IPv6 Core Protocols Implementation}
af Qing Li, Tatuya Jinmei og Keiichi Shima
\item \emph{IPv6 Advanced Protocols Implementation}
af Qing Li, Jinmei Tatuya og Keiichi Shima
\item - flere andre
\end{list1}

\hlkprofiluk

\end{document}
