\documentclass[20pt,landscape,a4paper,footrule]{foils}
\usepackage{solido-network-slides}

% 

\begin{document}
\selectlanguage{english}
\mytitlepage{Hands on p� IPv6}


\vskip 2cm
\centerline{\footnotesize Slides are available as PDF}

\slide{Goal}
\hlkimage{6cm}{kame-noanime-small.png}

\begin{list1}
\item Introduce IPv6 - facts and features
\item IPv6 addressing 
\item Neighbor Discovery Protocol
\item IPv4 vs IPv6 - Differences and similarities
\item Practical information about implementing IPv6 networks
\item Join a test network with your laptop
\end{list1}

%\centerline{Participate and demonstrate IPv6 working - make the turtle dance!}

\slide{Internet idag}

\hlkimage{14cm}{images/server-client.pdf}

\begin{list1}
\item Clients and servers
\item Rooted in academic networks
\item Protocols which are more than 20 years old, moved to TCP/IP in 1981 
\end{list1}


%%%%%%%%%%%%%%%%%%%%%%%%%%%%%%%%%%%%%%%%%%%%%%%%%%%%%%%%%%%%%%%%%%%%%%%
%% Node 1: UCLA (30 August, hooked up 2 September) 
%% Node 2: Stanford Research Institute (SRI) (1 October) 
%% Node 3: University of California Santa Barbara (UCSB) (1 November) 
%% Node 4: University of Utah (December) 
%% RFC 4: Network Timetable
%% http://www.zakon.org/robert/internet/timeline/

\slide{Internetworking: history}

\begin{list2}  
\item[1961]  L. Kleinrock, MIT packet-switching theory
\item[1962]  J. C. R. Licklider, MIT - notes 
\item[1964]  Paul Baran: On Distributed Communications
\item[1969]  ARPANET 4 nodes
\item[1971]  14 nodes
\item[1973]  Design of Internet Protocols started
\item[1973]  Email is about 75\% of all ARPANET traffic
\item[1974]  TCP/IP: Cerf/Kahn: A protocol for Packet
        Network Interconnection
\item[1983]  EUUG $\rightarrow$ DKUUG/DIKU forbindelse
\item[1988]  About 60.000 systems on the internet - 
        The Morris Worm hits about 10\%
\item[2002]  Ialt ca. 130 millioner p� Internet
\item[2010] IANA reserved blocks 7\% (Maj 2010) - \link{http://www.potaroo.net/tools/ipv4/}
\end{list2}

\slide{Fremtiden - 2010 og fremefter}

\begin{quote}
{\bf The Mobile Network in 2010 and 2011}\\
Global mobile data traffic grew 2.6-fold in 2010, nearly tripling for the third year in a row. The 2010 mobile data traffic growth rate was higher than anticipated. Last year's forecast projected that the growth rate would be 149 percent. This year's estimate is that global mobile data traffic grew 159 percent in 2010.\\
...\\
Last year's mobile data traffic was three times the size of the entire global Internet in 2000. Global mobile data traffic in 2010 (237 petabytes per month) was over three times greater than the total global Internet traffic in 2000 (75 petabytes per month).\\
...\\
There will be 788 million mobile-only Internet users by 2015. The mobile-only Internet population will grow 56-fold from 14 million at the end of 2010 to 788 million by the end of 2015.

\end{quote}

Kilde:
\emph{Cisco Visual Networking Index: Global Mobile Data Traffic Forecast Update, 2010 - 2015}
%\link{http://www.cisco.com/en/US/solutions/collateral/ns341/ns525/ns537/ns705/ns827/white_paper_c11-520862.html}

\slide{OSI \& Internet Protocols}

\hlkimage{14cm,angle=90}{images/compare-osi-ip.pdf}

\slide{IPv6: Internet redesigned? - no!}
 
\begin{list1}
\item Preserve the good stuff
\item back to basics, internet as it used to be!
\item fate sharing - connection rely on end points, not intermediary NAT boxes
\item end-to-end transparency - you have an address and I have an address
\item Wants: bandwidth +10G, low latency/predictable latency, Quality of Service, Security
\end{list1}

\vskip 5mm
\centerline{\color{titlecolor}\LARGE \bf IPv6 is evolution, not revolution}
\vskip 5mm

Note: IPv6 was not designed to solve all problems, so don't expect it to!


\slide{The Internet has done this before!}


\begin{quote}
   Because all hosts can not be converted to TCP simultaneously, and
   some will implement only IP/TCP, it will be necessary to provide
   temporarily for communication between NCP-only hosts and TCP-only
   hosts.  To do this certain hosts which implement both NCP and IP/TCP
   will be designated as relay hosts.  These relay hosts will support
   Telnet, FTP, and Mail services on both NCP and TCP.  These relay
   services will be provided  beginning in November 1981, and will be
   fully in place in January 1982.

  
   Initially there will be many NCP-only hosts and a few TCP-only hosts,
   and the load on the relay hosts will be relatively light.  As time
   goes by, and the conversion progresses, there will be more TCP
   capable hosts, and fewer NCP-only hosts, plus new TCP-only hosts.
   But, presumably most hosts that are now NCP-only will implement
   IP/TCP in addition to their NCP and become "dual protocol" hosts.
   So, while the load on the relay hosts will rise, it will not be a
   substantial portion of the total traffic.
\end{quote}

\centerline{NCP/TCP Transition Plan November 1981 RFC-801}

%\slide{KAME - IPv6 reference implementation}

%\hlkimage{6cm}{kame-noanime-small.png}

%\begin{list1}
%\item The KAME project was a joint effort of six companies in Japan 
%to provide a free stack of IPv6, IPsec, and Mobile IPv6 for BSD variants.
%\item The code is integrated and available in:
%\begin{list2}
%\item FreeBSD 4.0 and beyond
%\item OpenBSD 2.7 and beyond
%\item NetBSD 1.5 and beyond
%\item BSD/OS 4.2 and beyond
%\item and parts reused elsewhere
%\end{list2}
%\end{list1}


%center
%\centerline{\link{http://www.kame.net} and  \link{http://www.wide.ad.jp/}}


\slide{How to use IPv6}

\begin{center}
\vskip 3 cm
\hlkbig
www.solidonetworks.com

hlk@solidonetworks.com
\end{center}

\slide{Really how to use IPv6?}

\begin{list1}
\item Get IPv6 address and routing
\item Add AAAA (quad A) records to your DNS
\item Done
\end{list1}
\vskip 1cm
\centerline{\Large www.solidonetworks.com}

\begin{alltt}
\LARGE
www     IN	A       91.102.95.20
        IN	AAAA    2a02:9d0:10::9
\end{alltt}

\slide{IPv4 header - RFC-791}

\begin{alltt}
\small  
    0                   1                   2                   3   
    0 1 2 3 4 5 6 7 8 9 0 1 2 3 4 5 6 7 8 9 0 1 2 3 4 5 6 7 8 9 0 1 
   +-+-+-+-+-+-+-+-+-+-+-+-+-+-+-+-+-+-+-+-+-+-+-+-+-+-+-+-+-+-+-+-+
   |Version|  IHL  |Type of Service|          Total Length         |
   +-+-+-+-+-+-+-+-+-+-+-+-+-+-+-+-+-+-+-+-+-+-+-+-+-+-+-+-+-+-+-+-+
   |         Identification        |Flags|      Fragment Offset    |
   +-+-+-+-+-+-+-+-+-+-+-+-+-+-+-+-+-+-+-+-+-+-+-+-+-+-+-+-+-+-+-+-+
   |  Time to Live |    Protocol   |         Header Checksum       |
   +-+-+-+-+-+-+-+-+-+-+-+-+-+-+-+-+-+-+-+-+-+-+-+-+-+-+-+-+-+-+-+-+
   |                       Source Address                          |
   +-+-+-+-+-+-+-+-+-+-+-+-+-+-+-+-+-+-+-+-+-+-+-+-+-+-+-+-+-+-+-+-+
   |                    Destination Address                        |
   +-+-+-+-+-+-+-+-+-+-+-+-+-+-+-+-+-+-+-+-+-+-+-+-+-+-+-+-+-+-+-+-+
   |                    Options                    |    Padding    |
   +-+-+-+-+-+-+-+-+-+-+-+-+-+-+-+-+-+-+-+-+-+-+-+-+-+-+-+-+-+-+-+-+

                    Example Internet Datagram Header
\end{alltt}


\slide{IPv6 header - RFC-2460}


\begin{alltt}
\footnotesize

   +-+-+-+-+-+-+-+-+-+-+-+-+-+-+-+-+-+-+-+-+-+-+-+-+-+-+-+-+-+-+-+-+
   |Version| Traffic Class |           Flow Label                  |
   +-+-+-+-+-+-+-+-+-+-+-+-+-+-+-+-+-+-+-+-+-+-+-+-+-+-+-+-+-+-+-+-+
   |         Payload Length        |  Next Header  |   Hop Limit   |
   +-+-+-+-+-+-+-+-+-+-+-+-+-+-+-+-+-+-+-+-+-+-+-+-+-+-+-+-+-+-+-+-+
   |                                                               |
   +                                                               +
   |                                                               |
   +                         Source Address                        +
   |                                                               |
   +                                                               +
   |                                                               |
   +-+-+-+-+-+-+-+-+-+-+-+-+-+-+-+-+-+-+-+-+-+-+-+-+-+-+-+-+-+-+-+-+
   |                                                               |
   +                                                               +
   |                                                               |
   +                      Destination Address                      +
   |                                                               |
   +                                                               +
   |                                                               |
   +-+-+-+-+-+-+-+-+-+-+-+-+-+-+-+-+-+-+-+-+-+-+-+-+-+-+-+-+-+-+-+-+
\end{alltt}



%%%%%%%%%%%%%%%%%%%%%%%%%%%%%%%%%%%%%%%%%%%%%%%%%%%%%%%%%%%%%%%%%%%%%%%
\slide{IPv6 - new and improved}

\begin{list1}
\item IPv6 - extension headers RFC-2460
\begin{list2}
\item Hop-by-Hop Options
\item Routing (Type 0)
\item Fragment - fragmentation only at end-points!
\item Destination Options
\item Authentication
\item Encapsulating Security Payload
\end{list2}
\item Note: IPsec (AH and ESP) are mandatory for IPv6 hosts
\item Path MTU, PMTU implemented larger default MTU, at least 1280 bytes
\item Fragmentation only at the source host, no router fragmentation
\end{list1}

\slide{IPv6 addressing RFC-4291}

\begin{list1}
\item Addresses are always 128-bit identifiers for interfaces and sets of
   interfaces 
\item Unicast:   An identifier for a {\bf single interface}.\\
A packet sent to a
               unicast address is delivered to the interface identified
               by that address.
\item Anycast:   An identifier for a {\bf set of interfaces} (typically
               belonging to different nodes).\\  A packet sent to an
               anycast address is {\bf delivered to one} of the interfaces
               identified by that address (the "nearest" one, according
               to the routing protocols' measure of distance).

\item Multicast: An identifier for a {\bf set of interfaces} (typically
               belonging to different nodes). \\ A packet sent to a
               multicast address is {\bf delivered to all interfaces
               identified by that address}.
\end{list1}

\slide{IPv6 addressing RFC-4291, cont.}

\hlkimage{22cm}{ipv6-address-1.pdf}

\begin{list1}
\item 8 times 4 hex-digits seperated by colon x:x:x:x:x:x:x:x
\item Written as ipv6-address/prefix-length CIDR notation
\item Leading zeros can be removed
\item One or more groups of 16 bits of zeros can be replaced by ::
\end{list1}


\slide{Examples:}
\begin{list2}
\item ABCD:EF01:2345:6789:ABCD:EF01:2345:6789

\item Adddress 2001:DB8:0:0:8:800:200C:417A
\item Address of loopback ::1
\item IPv6 prefix 2a02:09d0:95::1/64, subnet 2a02:09d0:0095:0000::/64
\item Address 2a02:09d0:95::1 or 2a02:09d0:0095:0000:0000:0000:0000:0001
\end{list2}

\pause
\begin{list2}
\item Danish sites
\item Name servers for .dk\\
p.nic.dk has IPv6 address 2001:500:14:6036:ad::1\\
s.nic.dk has IPv6 address 2a01:3f0:0:303::53\\
b.nic.dk has IPv6 address 2a01:630:0:80::53
\item ns1.gratisdns.dk has IPv6 address 2a02:9d0:3002:1::2
\item www.solidonetworks.com has IPv6 address 2a02:9d0:10::9
\end{list2}


\slide{IPv6 address - special prefixes }

\begin{list2}
\item link-local unicast addresses\\
fe80::/10 generated from the interface MAC address EUI-64
\item FEC0::/10 site-local - deprecated in RFC-3879

\item 2001:0DB8::/32 NON-ROUTABLE range to be used for documentation purpose RFC-3849.

\item FC00::/7 Unique Local IPv6 Unicast Addresses RFC-4193\\
\link{http://www.simpledns.com/private-ipv6.aspx}\\
If you do not like to put public addresses on internal network - use this instead
\end{list2}


\slide{Windows - ipconfig}

\hlkimage{17cm}{win-ipconfig-ipv6.png}

\slide{Windows - control panel}
\hlkimage{17cm}{win-control-panel-ipv6.png}

\slide{Unix - practical examples ifconfig and ping}

\begin{alltt}\small
$ ifconfig en0
en0: flags=8863<UP,BROADCAST,SMART,RUNNING,SIMPLEX,MULTICAST> mtu 1500
	inet6 {\bf fe80::216:cbff:feac:1d9f%en0} prefixlen 64 scopeid 0x4 
	inet 10.0.42.15 netmask 0xffffff00 broadcast 10.0.42.255
	inet6 {\bf 2001:16d8:dd0f:cf0f:216:cbff:feac:1d9f} prefixlen 64 autoconf 
	ether 00:16:cb:ac:1d:9f 
	media: autoselect (1000baseT <full-duplex>) status: active

$ {\bf ping6 ::1}
PING6(56=40+8+8 bytes) ::1 --> ::1
16 bytes from ::1, icmp_seq=0 hlim=64 time=0.089 ms
16 bytes from ::1, icmp_seq=1 hlim=64 time=0.155 ms

$ {\bf traceroute6 2001:16d8:dd0f:cf0f::1}
traceroute6 to 2001:16d8:dd0f:cf0f::1 (2001:16d8:dd0f:cf0f::1) 
from 2001:16d8:dd0f:cf0f:216:cbff:feac:1d9f, 64 hops max, 12 byte packets
 1  2001:16d8:dd0f:cf0f::1  0.399 ms  0.371 ms  0.294 ms
\end{alltt}
	
\slide{ping6 global unicast address}

\begin{alltt}
\footnotesize
$ {\bf ping6 2001:1448:81:beef:20a:95ff:fef5:34df}
PING6(56=40+8+8 bytes) 2001:1448:81:beef::1 --> 2001:1448:81:beef:20a:95ff:fef5:34df
16 bytes from 2001:1448:81:beef:20a:95ff:fef5:34df, icmp_seq=0 hlim=64 time=10.639 ms
16 bytes from 2001:1448:81:beef:20a:95ff:fef5:34df, icmp_seq=1 hlim=64 time=1.615 ms
16 bytes from 2001:1448:81:beef:20a:95ff:fef5:34df, icmp_seq=2 hlim=64 time=2.074 ms
^C
--- 2001:1448:81:beef:20a:95ff:fef5:34df ping6 statistics ---
3 packets transmitted, 3 packets received, 0% packet loss
round-trip min/avg/max = 1.615/4.776/10.639 ms
$ {\bf ping6 www.kame.net}
PING6(56=40+8+8 bytes) 2001:16d8:dd00:75::2 --> 2001:200:dff:fff1:216:3eff:feb1:44d7
16 bytes from 2001:200:dff:fff1:216:3eff:feb1:44d7, icmp_seq=0 hlim=50 time=340.213 ms
16 bytes from 2001:200:dff:fff1:216:3eff:feb1:44d7, icmp_seq=1 hlim=50 time=371.965 ms
16 bytes from 2001:200:dff:fff1:216:3eff:feb1:44d7, icmp_seq=2 hlim=50 time=393.242 ms
16 bytes from 2001:200:dff:fff1:216:3eff:feb1:44d7, icmp_seq=3 hlim=50 time=418.496 ms
^C
--- orange.kame.net ping6 statistics ---
4 packets transmitted, 4 packets received, 0.0% packet loss
round-trip min/avg/max/std-dev = 340.213/380.979/418.496/28.727 ms

\end{alltt}

\slide{ping6 link-local address}

\begin{alltt}
\footnotesize
$ {\bf ping6 -I en1 fe80::20d:93ff:fe4d:55fe}
PING6(56=40+8+8 bytes) fe80::223:6cff:fe9a:f52c%en1 --> fe80::20d:93ff:fe4d:55fe
16 bytes from fe80::20d:93ff:fe4d:55fe%en1, icmp_seq=0 hlim=64 time=1.557 ms
16 bytes from fe80::20d:93ff:fe4d:55fe%en1, icmp_seq=1 hlim=64 time=1.725 ms
^C
--- fe80::20d:93ff:fe4d:55fe ping6 statistics ---
2 packets transmitted, 2 packets received, 0.0% packet loss
round-trip min/avg/max/std-dev = 1.557/1.641/1.725/0.084 ms
\end{alltt}

Note: -I en1 specifies that this interface is being used.

\slide{ ping6 til specielle adresser}


\begin{alltt}
\small
$ {\bf ping6 -I en1 ff02::1}

PING6(56=40+8+8 bytes) fe80::230:65ff:fe17:94d1 --> ff02::1
16 bytes from fe80::230:65ff:fe17:94d1, icmp_seq=0 hlim=64 time=0.483 ms
16 bytes from fe80::20a:95ff:fef5:34df, icmp_seq=0 hlim=64 time=982.932 ms
16 bytes from fe80::230:65ff:fe17:94d1, icmp_seq=1 hlim=64 time=0.582 ms
16 bytes from fe80::20a:95ff:fef5:34df, icmp_seq=1 hlim=64 time=9.6 ms
16 bytes from fe80::230:65ff:fe17:94d1, icmp_seq=2 hlim=64 time=0.489 ms
16 bytes from fe80::20a:95ff:fef5:34df, icmp_seq=2 hlim=64 time=7.636 ms
^C
--- ff02::1 ping6 statistics ---
4 packets transmitted, 4 packets received, +4 duplicates, 0% packet loss
round-trip min/avg/max = 0.483/126.236/982.932 ms

ff02::1 multicast address of all-hosts on the local link 
ff02::2 multicast address of all-routers on the local link 
\end{alltt}

\slide{Nping testing IPv6 TCP connections}


\begin{alltt}
\small
$ {\bf nping -6 www.solidonetworks.com}

Starting Nping 0.5.35DC1 ( http://nmap.org/nping ) at 2011-04-06 06:25 CEST
SENT (0.0000s) Starting TCP Handshake > 2a02:9d0:10::9:80
RECV (0.7110s) Handshake with 2a02:9d0:10::9:80 completed
SENT (1.0010s) Starting TCP Handshake > 2a02:9d0:10::9:80
RECV (1.0960s) Handshake with 2a02:9d0:10::9:80 completed
SENT (2.0030s) Starting TCP Handshake > 2a02:9d0:10::9:80
RECV (2.0860s) Handshake with 2a02:9d0:10::9:80 completed
SENT (3.0050s) Starting TCP Handshake > 2a02:9d0:10::9:80
RECV (3.0940s) Handshake with 2a02:9d0:10::9:80 completed
SENT (4.0060s) Starting TCP Handshake > 2a02:9d0:10::9:80
RECV (4.1240s) Handshake with 2a02:9d0:10::9:80 completed
 
Max rtt: 711.307ms | Min rtt: 82.840ms | Avg rtt: 219.132ms
TCP connection attempts: 5 | Successful connections: 5 | Failed: 0 (0.00%)
Tx time: 4.00702s | Tx bytes/s: 99.82 | Tx pkts/s: 1.25
Rx time: 4.12474s | Rx bytes/s: 48.49 | Rx pkts/s: 1.21
Nping done: 1 IP address pinged in 4.12 seconds
\end{alltt}



\slide{Hello neighbors}

\begin{alltt}\small
$ {\bf ping6 -w -I en1 ff02::1}
PING6(72=40+8+24 bytes) fe80::223:6cff:fe9a:f52c%en1 --> ff02::1
30 bytes from fe80::223:6cff:fe9a:f52c%en1: bigfoot
36 bytes from fe80::216:cbff:feac:1d9f%en1: mike.kramse.dk.
38 bytes from fe80::200:aaff:feab:9f06%en1: xrx0000aaab9f06
34 bytes from fe80::20d:93ff:fe4d:55fe%en1: harry.local
36 bytes from fe80::200:24ff:fec8:b24c%en1: kris.kramse.dk.
31 bytes from fe80::21b:63ff:fef5:38df%en1: airport5
32 bytes from fe80::216:cbff:fec4:403a%en1: main-base
44 bytes from fe80::217:f2ff:fee4:2156%en1: Base Station Koekken 
35 bytes from fe80::21e:c2ff:feac:cd17%en1: arnold.local
\end{alltt}


%\slide{TCP/IP basic configuration}

%\begin{list1}
%\item Unix systems network configuration is mostly using:
%\item \verb+ifconfig+, \verb+route+ and \verb+netstat+  
%\item Diagnosing and testing is often done using \verb+ping+ and \verb+traceroute+
%\item Most platforms added \verb+ping6+ and \verb+traceroute6+, but some uses 
%ping/traceroute for both IPv4 and IPv6
%\end{list1}

%\begin{alltt}
%ifconfig en0 10.0.42.1 netmask 255.255.255.0
%route add default gw 10.0.42.1 
%\end{alltt}
%Linux wants 'gw' but BSD doesn't:
%\vskip 5 mm
%\begin{alltt}
%route add default 10.0.42.1
%\end{alltt}



\slide{CentOS}

Only Two places need updating the file /etc/sysconfig/network:
\begin{alltt}\small
NETWORKING=yes
NETWORKING_IPV6=yes
HOSTNAME=host1.armadahosting.com
GATEWAY=10.234.123.254
\end{alltt}

From the file: /etc/sysconfig/network-scripts/ifcfg-eth0:
\begin{alltt}\small
DEVICE=eth0
BOOTPROTO=none
ONBOOT=yes
BROADCAST=10.234.123.255
NETWORK=10.234.123.0
NETMASK=255.255.255.0
IPADDR=10.234.123.90
USERCTL=no
IPV6INIT=yes
IPV6ADDR=2a02:9d0:10::10:234:123:90
IPV6_DEFAULTGW=2a02:9d0:10::1
\end{alltt}

\slide{IPv6 autoconfiguration}

\hlkimage{24cm}{modified-eui64.pdf}

\begin{list1}
\item DHCPv6 is available, but {\bfseries stateless autoconfiguration} is king
\item Routers announce subnet prefix via {\bfseries router advertisements}
\item Individual nodes then combine this with their EUI64 identifier
\end{list1}

%\link{http://www.cisco.com/web/about/ac123/ac147/archived_issues/ipj_7-2/ipv6_autoconfig.html}

\slide{Router Advertisement ICMPv6}

\hlkimage{16cm}{ipv6-router-advertisement.png}




\slide{IPv6 sockets}

\begin{alltt}
\small
root# netstat -an | grep -i listen

tcp46  0  0  *.80             *.*    LISTEN
tcp4   0  0  *.6000           *.*    LISTEN
tcp4   0  0  127.0.0.1.631    *.*    LISTEN
tcp4   0  0  *.25             *.*    LISTEN
tcp4   0  0  *.20123          *.*    LISTEN
tcp46  0  0  *.20123          *.*    LISTEN
tcp4   0  0  127.0.0.1.1033   *.*    LISTEN
\end{alltt}

Note: some platforms show tcp/tcp6 for IPv4/IPv6 and some show tcp4/tcp6

\slide{IPv6 - inet6 family}


\begin{alltt}
\small
root# netstat -an -f inet6

Active Internet connections (including servers)
Proto Recv Send  Local  Foreign   (state)
tcp46  0   0     *.80     *.*     LISTEN
tcp46  0   0     *.22780  *.*     LISTEN
udp6   0   0     *.5353   *.*                    
udp6   0   0     *.5353   *.*                    
udp6   0   0     *.514    *.*                    
icm6   0   0     *.*      *.*                    
icm6   0   0     *.*      *.*                    
icm6   0   0     *.*      *.* 
\end{alltt}

Note: this is from a Mac OS X and edited a little

\slide{IPv6 is default for a lot of services}

\begin{alltt}
\small
root# telnet localhost 80

Trying ::1...
Connected to localhost.
Escape character is '^]'.
GET / HTTP/1.0

HTTP/1.1 200 OK
Date: Thu, 19 Feb 2004 09:22:34 GMT
Server: Apache/2.0.43 (Unix)
Content-Location: index.html.en
Vary: negotiate,accept-language,accept-charset
...
\end{alltt}


\slide{Apache HTTPD server}


\begin{alltt}
Listen 0.0.0.0:80
Listen [::]:80
...
Allow from 127.0.0.1
Allow from 2001:1448:81:0f:2d:9ff:f86:3f 
Allow from 217.157.20.133
\end{alltt}

\begin{list1}
\item IPv6 goes into the same places as IPv4
\item IPv4 and port numbers are fine
\item IPv6 and port numbers - use \verb+[adress]:port+
\item Example from Apache HTTPD web server \link{http://httpd.apache.org}
\end{list1}


\slide{Web server access log}

\begin{alltt}
\footnotesize
::1 - - [19/Feb/2004:09:05:33 +0100] "GET /images/IPv6ready.png HTTP/1.1" 304 0
::1 - - [19/Feb/2004:09:05:33 +0100] "GET /images/valid-html401.png HTTP/1.1" 304 0
::1 - - [19/Feb/2004:09:05:33 +0100] "GET /images/snowflake1.png  HTTP/1.1" 304 0
::1 - - [19/Feb/2004:09:05:33 +0100] "GET /~hlk/security6.net/images/logo-1.png
HTTP/1.1" 304 0
2001:1448:81:beef:20a:95ff:fef5:34df - - [19/Feb/2004:09:57:35 +0100] 
"GET / HTTP/1.1" 200 1456
2001:1448:81:beef:20a:95ff:fef5:34df - - [19/Feb/2004:09:57:35 +0100] 
"GET /apache_pb.gif HTTP/1.1" 200 2326
2001:1448:81:beef:20a:95ff:fef5:34df - - [19/Feb/2004:09:57:36 +0100]
"GET /favicon.ico HTTP/1.1" 404 209
2001:1448:81:beef:20a:95ff:fef5:34df - - [19/Feb/2004:09:57:36 +0100] 
"GET /favicon.ico HTTP/1.1" 404 209
\end{alltt}

\vskip 8mm
\centerline{Can you process logs with IPv6 addresses}


\slide{Routing - IPv4}

\begin{alltt}
\small
$ netstat -rn
Routing tables

Internet:
Destination    Gateway         Flags  Refs      Use  Netif 
default        10.0.0.1        UGSc    23        7    en0
10/24          link#4          UCS      1        0    en0
10.0.0.1       0:0:24:c1:58:ac UHLW    24       18    en0  
10.0.0.33      127.0.0.1       UHS      0        1    lo0
10.0.0.63      127.0.0.1       UHS      0        0    lo0
127            127.0.0.1       UCS      0        0    lo0
127.0.0.1      127.0.0.1       UH       4     7581    lo0
169.254        link#4          UCS      0        0    en0  
\end{alltt}

\slide{Routing - IPv6}
\begin{alltt}
\small
$ netstat -f inet6 -rn 
Routing tables

Internet6:
Destination                 Gateway           Flags      Netif 
default             fe80::200:24ff:fec1:58ac  UGc         en0
::1                         ::1               UH          lo0
2001:1448:81:cf0f::/64      link#4            UC          en0
2001:1448:81:cf0f::1        0:0:24:c1:58:ac   UHLW        en0
fe80::/64                   fe80::1           Uc          lo0
fe80::1                     link#1            UHL         lo0
fe80::/64                   link#4            UC          en0
fe80::20d:93ff:fe28:2812    0:d:93:28:28:12   UHL         lo0
fe80::/64                   link#5            UC          en1
fe80::20d:93ff:fe86:7c3f    0:d:93:86:7c:3f   UHL         lo0
ff01::/32                   ::1               U           lo0
ff02::/32                   ::1               UC          lo0
ff02::/32                   link#4            UC          en0
ff02::/32                   link#5            UC          en1
\end{alltt}


\slide{ARP in IPv4}

\hlkimage{22cm}{images/arp-basic.pdf} 

%server 00:30:65:22:94:a1\\
%client 00:40:70:12:95:1c\\
%hacker 00:02:03:04:05:06\\

\slide{ARP request and reply}
\begin{list1}
\item {\bfseries ping 10.0.0.2} from server
\item ARP Address Resolution Protocol request/reply:
  \begin{list2}
  \item ARP request broadcasted on layer 2 - Who has 10.0.0.2 Tell 10.0.0.1
  \item ARP reply (from 10.0.0.2) 10.0.0.2 is at 00:40:70:12:95:1c
  \end{list2}
\item IP ICMP request/reply:
  \begin{list2}
    \item Echo (ping) request from 10.0.0.1 to 10.0.0.2
\item Echo (ping) reply from 10.0.0.2 to 10.0.0.1
\item ...
  \end{list2}
\item ARP is performed on Ethernet before IP can be transmitted
\end{list1}


\slide{ IPv6 neighbor discovery protocol (NDP)}

\hlkimage{20cm}{ipv6-arp-ndp.pdf}

\begin{list1}
\item ARP er v�k
\item NDP erstatter og udvider ARP, Sammenlign \verb+arp -an+ med \verb+ndp -an+
\item Til dels erstatter ICMPv6 s�ledes DHCP i IPv6, DHCPv6 findes dog
\item {\bf NB: bem�rk at dette har stor betydning for firewallregler!}
\item RFC4861 Neighbor Discovery for IP version 6 (IPv6)
\end{list1}

\slide{ARP vs NDP}

\begin{alltt}
\small
$ {\bf arp -an}
? (10.0.42.1) at{\bf 0:0:24:c8:b2:4c} on en1 [ethernet]
? (10.0.42.2) at 0:c0:b7:6c:19:b on en1 [ethernet]

$ {\bf ndp -an}
Neighbor                      Linklayer Address  Netif Expire    St Flgs Prbs
::1                           (incomplete)         lo0 permanent R      
2001:16d8:ffd2:cf0f:21c:b3ff:fec4:e1b6 0:1c:b3:c4:e1:b6 en1 permanent R      
fe80::1%lo0                   (incomplete)         lo0 permanent R      
fe80::200:24ff:fec8:b24c%en1 {\bf 0:0:24:c8:b2:4c}      en1 8h54m51s  S  R   
fe80::21c:b3ff:fec4:e1b6%en1  0:1c:b3:c4:e1:b6     en1 permanent R      
\end{alltt}

\slide{ICMPv6 has more}

\begin{list1}
\item Autoconfiguration - what is the network prefix
\item Duplicate Address Detection - can I use this address
\item Neighbor Discovery - which neighbors exist
\item Link layer addresses - "ARP" for IPv6
\item Neighbor Unreachability Detection, or NUD) - neighbors still alive
\end{list1}

\slide{IPv6 firewalls - you MUST allow SOME ICMPv6}

\begin{alltt}\small
# Simple stateful network firewall rules for IPv6 
# using IPv4 file for input and inspiration from
# http://www.ipv6style.jp/en/building/20040526/2.shtml
# input from 
        $fwcmd6 -f flush
        $fwcmd6 add allow all from any to any via lo0
# Allow ICMPv6 destination unreach
        $fwcmd6 add pass ipv6-icmp from any to any icmptypes 1
# Allow NS/NA/toobig (don't filter it out)
        $fwcmd6 add pass ipv6-icmp from any to any icmptypes 2  
# Allow timex Time exceeded 
        $fwcmd6 add pass ipv6-icmp from any to any icmptypes 3  
# Allow parameter problem
        $fwcmd6 add pass ipv6-icmp from any to any icmptypes 4  
# IPv6 ICMP - echo request (128) and echo reply (129)
        $fwcmd6 add pass ipv6-icmp from any to any icmptypes 128,129
# IPv6 ICMP - router solicitation (133) and router advertisement (134)
        $fwcmd6 add pass ipv6-icmp from any to any icmptypes 133,134  
# IPv6 ICMP - neighbour discovery solicitation (135) and advertisement (136)
        $fwcmd6 add pass ipv6-icmp from any to any icmptypes 135,136
\end{alltt}

\slide{IPv6 firewalls, cont. allowing services}

\begin{alltt}\small
#       Allow all established connections to persist (setup required
#       for new connections).
        $fwcmd6 add allow tcp from any to any established
        $fwcmd6 add allow tcp from any to any out setup
# allow access to my webserver and ssh
#       $fwcmd6 add allow tcp from any to any 80,443  setup
        $fwcmd6 add allow tcp from any to any $ssh  setup

# allow access to X11 forwarding over ::1
        $fwcmd6 add allow tcp from any to ::1 6010  setup

#       Politely rejects AUTH requests (e.g. email and ftp)
        $fwcmd6 add reset tcp from any to any 113 

#       Deny everything else ipv6
        $fwcmd6 add 65435 deny log ipv6 from any to any

\end{alltt}

% practical information

\slide{Practical information for your network}

\begin{list1}
\item IPv6 is already in your network - see next slide
\item Take control of IPv6, do not just block it \smiley
\vskip 1cm
\item Strategy and actions points
\begin{list2}
\item Collect information about IPv6 
\item Collect information about your network
\item Collect information about your hosts and services
\item Ask your providers for IPv6 plans
\item Experiment with IPv6 - today
\item Implement small proof of concept, in production!
\item Expand coverage
\end{list2}
\end{list1}

\slide{IPv6 is coming}

\hlkimage{2cm}{IPv6ready.png}

\begin{list1}
\item An important consideration is that IPv6 is quite likely to be already running on the enterprise network, whether that implementation was planned or not. Some important characteristics of IPv6 include:
\begin{list2}
\item IPv6 has a mechanism to automatically assign addresses so that end systems can easily establish communications.
\item IPv6 has several mechanisms available to ease the integration of the protocol into the network.
\item Automatic tunneling mechanisms can take advantage of the underlying IPv4 network and connect it to the IPv6 Internet.
\end{list2}
\end{list1}

Kilde:\\
{\footnotesize\link{http://www.cisco.com/en/US/prod/collateral/iosswrel/ps6537/ps6553/white_paper_c11-629391.html}}

\slide{Implications}

\hlkimage{2cm}{IPv6ready.png}

\begin{list1}
\item For an IPv4 enterprise network, the existence of an IPv6 overlay network has several of implications:
\begin{list2}
\item The IPv4 firewalls can be bypassed by the IPv6 traffic, and leave the security door wide open.
\item Intrusion detection mechanisms not expecting IPv6 traffic may be confused and allow intrusion
\item In some cases (for example, with the IPv6 transition technology known as 6to4), an internal PC can communicate directly with another internal PC and evade all intrusion protection and detection systems (IPS/IDS). Botnet command and control channels are known to use these kind of tunnels.
\end{list2}
\end{list1}

Kilde:\\
{\footnotesize\link{http://www.cisco.com/en/US/prod/collateral/iosswrel/ps6537/ps6553/white_paper_c11-629391.html}}


\slide{Collect information about IPv6}

\begin{list1}
\item \emph{Guidelines for the Secure Deployment of IPv6}, SP800-119, NIST\\
\link{http://csrc.nist.gov/publications/nistpubs/800-119/sp800-119.pdf}
\item \emph{The Second Internet: Reinventing Computer Networks with IPv6}, Lawrence E. Hughes, October 2010,\\ \link{http://www.secondinternet.org/}
\item \emph{IPv6 Network Administration}
af David Malone og Niall Richard Murphy
\item \link{http://www.ripe.net}
\item This presentation \smiley
\end{list1}


\slide{Collect information about your network}

\begin{list1}
\item devices - what is a network device?
\item switches - Layer 2 does not matter much, management by RFC-1918 IPv4 is OK
\item routers - most important, connectivity MUST support IPv6. Check vendor home page - do NOT assume support is ready
\item Security devices: firewalls, IDS/IPS, VPN - critical and support in general poor. Some vendors such as Cisco ASA and Juniper SRX has good support

\end{list1}

\vskip 1 cm
\centerline{\bf Remember to add IPv6 to list of requirements for new devices}

\slide{Collect information about your hosts and services}

\begin{list1}
\item servers and services, today everything is IPv4 - in Europe
\item clients - do you only support PCs running Windows? think again.\\
 Smart phones and tablets are the future
\item Desktop and laptop operating systems:\\
clients Windows, Linux and Mac OS X HAS great IPv6 support\\(Dont ask about Windows Xp and Vista, kill them on sight)
\item Mobile operating systems: support is rapidly increasing, iPhone/iPad - OK, Android 2.x yes, we think so but double check \smiley

\end{list1}


\slide{Ask your providers for IPv6 plans}

\begin{alltt}
Hvad er dit sp�rgsm�l
Hvorn�r forventer I at f� IPv6? Pt. er hastigheden jo
0Kb/s med IPv6 hos jer ;-)
Dit telefonnummer hos 3	20266000
Dit navn	Henrik Kramshoej
Kontaktnummer	20266000
E-mailadresse	hlk@solido.net
Har du kontaktet 3s kundeservice om denne henvendelse tidligere?	Nej
\end{alltt}

\vskip 1cm
\centerline{Question sent to Danish Mobile operator 3.dk - a 3G company}

\slide{Answer from Mobile operator 3.dk}

\hlkimage{25cm}{ipv6-hos-3.png}

\slide{IPv6 in the Nordic region, and Iceland}

\hlkimage{16cm}{ipv6-nordic.png}

\slide{Curent status Denmark}

\begin{list1}
\item Too little interest - less than 100 people thinking about IPv6?
\item Some providers have some IPv6 connectivity
\item NO ISPs have IPv6 to consumers
\item NO ISPs market IPv6 as a product, except me perhaps :-)
\item Perceived NO NEEED
\vskip 2 cm
\pause
\item Free, a major French ISP rolled-out IPv6 at end of year 2007
\item XS4All As of August 2010 native IPv6 DSL connections became available to almost all their customers.
\end{list1}

Source: \link{http://en.wikipedia.org/wiki/IPv6_deployment}




\slide{Experiment with IPv6 - today}

\begin{list1}
\item Native IPv6 - available at some hosting providers in DK
\item Automatic tunnels 6to4, Teredo etc.
\begin{list2}
\item 6to4 benytter IPv4 infrastrukturen
\item Teredo sender IPv6 gennem IPv4/UDP pakker
\end{list2}
\item Configured tunnels and tunnelbrokers
\begin{list2}
\item \link{http://sixxs.net} IPv6 Deployment \& Tunnel Broker
\item \link{http://he.net} hurricane electric internet services
\end{list2}
\end{list1}

\centerline{ Implement small proof of concept, in production!}


\slide{Allocating IPv6 addresses} 

\begin{list1}
\item You have plenty!
\item Providers and LIRs will typically get /32
\item Providers will typically give organisations /48 or /56
\item Your /48 can be used for:
\begin{list2}
\item 65536 subnets - all host subnets are /64
\item Each subnet has $2^{64}$ addresses
\end{list2}
\end{list1}

\slide{Preparing an IPv6
Addressing Plan}

\hlkimage{20cm}{ipv6-address-plan-ripe.png}

{\footnotesize \link{http://www.ripe.net/training/material/IPv6-for-LIRs-Training-Course/IPv6_addr_plan4.pdf}}

\slide{Example adress plan input}

\hlkimage{22cm}{ipv6-linked-to-ipv4.png}

\centerline{Easy and coupled with VLAN IDs it will work \smiley}

\slide{Run IPv6 in production}

\begin{list1}
\item Make sure you establish IPv6 in {\bf production}
\item Enabling service on IPv6 without production - bad experience for users
\item Start by enabling your DNS servers for IPv6 - and DNSSEC - and DNS over TCP\\
Remember that your firewall might have problems with large DNS packets
\item Add a production IPv6 router - hardware device or generic server
\item Tunnels are OK, and SixXS consider their service production
\end{list1}



\slide{World IPv6 Day}
\begin{quote}
{\bf About World IPv6 Day}

On 8 June, 2011, Google, Facebook, Yahoo!, Akamai and Limelight Networks will be amongst some of the major organisations that will offer their content over IPv6 for a 24-hour "test flight". The goal of the Test Flight Day is to motivate organizations across the industry - Internet service providers, hardware makers, operating system vendors and web companies - to prepare their services for IPv6 to ensure a successful transition as IPv4 addresses run out.

Please join us for this test drive and help accelerate the momentum of IPv6 deployment.
\end{quote}

\centerline{\link{http://isoc.org/wp/worldipv6day/} and \link{http://test-ipv6.com/}}


\slide{IPv6 business case}


\begin{list2}
\item An almost unlimited scalability with a very large IPv6 address space ($2^128$ addresses), enabling IP addresses to each and every device.

\item Address self-configuration mechanisms, easing the deployment.

\item Improved security and authentication features, such as mandatory IPSec capacities and the possibility to use of the address space to include encryption keys.

\item Peer-to-peer connectivity, solving the NAT barrier with specific and permanent IP addresses for any device and/or user of the Internet.

\item Mobility features, enabling a seamless connexion when moving from one access point to another access point on the Internet.

\item Multi cast and any cast functionalities.

\item IPv6 will provide an easier remote interaction with each and every device with a {\bfseries direct integration to the Internet.} In other words, IPv6 will make possible to move from a network of servers, to a network of things.

\end{list2}

\centerline{ Business case for IPv6 is {\bf continuity}}


{\footnotesize Partial quote from http://www.smartipv6building.org/index.php/en/ipv6-potential}



\slide{Conclusion}

\begin{center}
\vskip 5mm
{\color{titlecolor}\LARGE \bf IPv6 is here already - use it}
\vskip 5mm


\link{http://www.ipv6actnow.org/}

\link{http://digitaliser.dk/group/374895}

\link{http://www.ipv6tf.dk}

\end{center}


\slide{Up and running with IPv6}

\begin{list1}
\item Join the fun - join the wireless network
\item Use ping/ping6 and traceroute to test connectivity
\item Try in your browser:
\begin{list2}
\item \link{http://www.kame.net} Dancing turtle
\item \link{http://www.ripe.net} RIPE, look for address up right corner 
\item \link{http://loopsofzen.co.uk/} Play a game
\item \link{https://www.sixxs.net/} Apply for IPv6 tunnel 
\end{list2}
\item Done \smiley
\end{list1}

\myquestionspage


\slide{More Information}

\hlkimage{18cm}{twitter-security-feed.png}

\begin{list1}
\item Twitter has become an important new ressource for lots of stuff
\item Twitter has replaced RSS for me 
\end{list1}

\slide{Ressources}

\begin{list1}
\item \emph{Guidelines for the Secure Deployment of IPv6}, SP800-119, NIST\\
\link{http://csrc.nist.gov/publications/nistpubs/800-119/sp800-119.pdf}
\item \emph{The Second Internet: Reinventing Computer Networks with IPv6}, Lawrence E. Hughes, October 2010,\\ \link{http://www.secondinternet.org/}
\item \emph{IPv6 Network Administration}
af David Malone og Niall Richard Murphy
 - god til real-life admins, typisk
O'Reilly bog
\item \emph{IPv6 Essentials} af Silvia Hagen, O'Reilly 2nd edition (May 17, 2006)
	god reference om emnet
\item \emph{IPv6 Core Protocols Implementation}
af Qing Li, Tatuya Jinmei og Keiichi Shima
\item \emph{IPv6 Advanced Protocols Implementation}
af Qing Li, Jinmei Tatuya og Keiichi Shima
\item - flere andre
\end{list1}



\slide{Danish resources - get involved}

\hlkimage{10cm}{taskforce-logo.jpg}

\centerline{ Danish IPv6 task force - unofficial
\link{http://www.ipv6tf.dk}}


\slide{VikingScan.org - free portscanning}

\hlkimage{18cm}{vikingscan.png}
%\vskip 1cm 
%\centerline{\link{http://www.vikingscan.org}}


\hlkprofiluk


\end{document}
