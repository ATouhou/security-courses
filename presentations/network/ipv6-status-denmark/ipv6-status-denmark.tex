\documentclass[20pt,landscape,a4paper,footrule]{foils}
%\usepackage{solido-network-slides}
\usepackage{zencurity-slides}
\usepackage[normalem]{ulem}

%Dette indlæg giver en kort status over projekter som er indført i Danmark med udgangspunkt i managed hosting for store danske websteder og infrastrukturer for danske virksomheder. Henrik Kramshøj har arbejdet med IPv6 siden midten af 1990'erne og har opdateret flere netværk til at understøtte IPv6. Forhåbentlig kan der også afsløres store danske websites som i anledning af "IPv6 World day" den 8. juni kører IPv6.
%v/ Henrik Lund Kramshøj, Solido Networks ApS


\begin{document}
\selectlanguage{english}
\mytitlepage{IPv6 Current Status Denmark}


\vskip 2cm
\centerline{\footnotesize Slides are available as PDF kramshoej@Github}

\slide{Goal}
\hlkimage{6cm}{kame-noanime-small.png}

\begin{list1}
\item IPv6 Status Denmark
\item Enabled providers and sites
\item How to get your site on IPv6
\end{list1}

%\centerline{Participate and demonstrate IPv6 working - make the turtle dance!}

\slide{Internet today}

\hlkimage{14cm}{images/server-client.pdf}

\begin{list1}
\item Clients and servers
\item Rooted in academic networks
\item Protocols which are more than 20 years old, moved to TCP/IP in 1981
\end{list1}


%%%%%%%%%%%%%%%%%%%%%%%%%%%%%%%%%%%%%%%%%%%%%%%%%%%%%%%%%%%%%%%%%%%%%%%
%% Node 1: UCLA (30 August, hooked up 2 September)
%% Node 2: Stanford Research Institute (SRI) (1 October)
%% Node 3: University of California Santa Barbara (UCSB) (1 November)
%% Node 4: University of Utah (December)
%% RFC 4: Network Timetable
%% http://www.zakon.org/robert/internet/timeline/

\slide{Internetworking: history}

\begin{list2}
\item[1960s]  L. Kleinrock, MIT packet-switching theory,  J. C. R. Licklider, MIT - notes
  Paul Baran: On Distributed Communications
\item[1969]  ARPANET 4 nodes
\item[1971]  14 nodes
\item[1974]  TCP/IP: Cerf/Kahn: A protocol for Packet
        Network Interconnection
\item[1983]  Switching from NCP to IP/TCP
\item[1983]  EUUG $\rightarrow$ DKUUG/DIKU forbindelse
\item[1988]  About 60.000 systems on the internet -
        The Morris Worm hits about 10\%
\item[2010] IANA reserved blocks 7\% (Maj 2010) - \link{http://www.potaroo.net/tools/ipv4/}
\item[2011] February 3 IANA pool ran out - last 5 /8 allocated to RIRs
\item[2011] April 15 APNIC ran into their last /8 and started a more restrictive policy
\end{list2}

\slide{How to use IPv6}

\begin{center}
\vskip 3 cm
\hlkbig
www.zencurity.com

hlk@zencurity.com
\end{center}

\slide{Really how to use IPv6?}

\begin{list1}
\item Get IPv6 address and routing
\item Add AAAA (quad A) records to your DNS - Done
\end{list1}

\begin{alltt}
www     IN A       185.129.60.130
        IN AAAA    2a06:d380:0:3065::80
mail    IN AAAA    2a02:9d0:10::216

www.zencurity.com has address 185.129.60.130
www.zencurity.com has IPv6 address 2a06:d380:0:3065::80
mail.zencurity.com has IPv6 address 2a02:9d0:10::216
\end{alltt}





\slide{IPv6 Status Denmark}

\begin{list1}
\item \sout{IT- og Telestyrelsen are becoming more active}\\
They are killed, nobody in government does much with IPv6 anymore
\item Unofficial IPv6 task force at \link{http://www.ipv6tf.dk/}\\
Service Temporarily Unavailable, my web site - had almost no content
\item Major providers are ready on back bones
\item Internet Providers are increasingly becoming ready, {\bf slowly}
\item Other initiatives \link{http://world-ipv6-day.dk/}
\item Sites previously on IPv6 are not anymore, like Version2.dk and Computerworld :-(
\end{list1}

TL;DR We are getting IPv6, but find a tunnelbroker unless you are on Altibox, Fullrate, Gigabit or Bolig:net - \link{https://tunnelbroker.net/}

\slide{Enabled IPv6 ISPs in Denmark}

\hlkimage{26cm}{dk-isp-ipv6-2016-1.png}
Source: \link{https://ipv6-adresse.dk/} Emil Stahls excellent web site

\slide{Enabled IPv6 ISPs in Denmark}

\hlkimage{20cm}{dk-isp-ipv6-2016-2.png}
Source: \link{https://ipv6-adresse.dk/}

Also check \link{http://ripeness.ripe.net/5star/DK.html}

\centerline{}
\slide{IPv6 in the Nordic region}

\hlkimage{16cm}{ipv6-nordic-2016.png}

%\link{http://v6asns.ripe.net/v/6?s=_ALL;s=DK;s=SE;s=NO;s=NL}

\slide{IPv6 in the Denmark according to google}

\hlkimage{15cm}{ipv6-status-denmark-google.png}

Source:
\link{https://www.google.com/intl/en/ipv6/statistics.html}

\slide{How to get your site on IPv6}

Practical information for your network

\begin{list1}
\item Strategy and actions points
\begin{list2}
\item Collect information about IPv6
\item Collect information about your network
\item Collect information about your hosts and services
\item Ask your providers for IPv6 plans
\item Experiment with IPv6 - today
\item Implement small proof of concept, in production!
\item Expand coverage
\end{list2}
\end{list1}

footnote, I bought two dirt cheap TP-Link managed switches recently, but supposedly has IPv6 routing and even ACLs! T2600G-28TS ~1300kr eks moms\\ {\footnotesize\link{http://www.tp-link.com/en/products/details/cat-39_T2600G-28TS-(TL-SG3424).html}}

No excuses that your hardware in your enterprise does not support IPv6!

\slide{BTW IPv6 is already in your network}

\hlkimage{2cm}{IPv6ready.png}

\begin{list1}
\item For an IPv4 enterprise network, the existence of an IPv6 overlay network has several of implications:
\begin{list2}
\item The IPv4 firewalls can be bypassed by the IPv6 traffic, and leave the security door wide open.
\item Intrusion detection mechanisms not expecting IPv6 traffic may be confused and allow intrusion
\item In some cases (for example, with the IPv6 transition technology known as 6to4), an internal PC can communicate directly with another internal PC and evade all intrusion protection and detection systems (IPS/IDS). Botnet command and control channels are known to use these kind of tunnels.
\end{list2}
\end{list1}

Source: From 2010!\\
{\footnotesize\link{http://www.cisco.com/en/US/prod/collateral/iosswrel/ps6537/ps6553/white_paper_c11-629391.html}}


\slide{Allocating IPv6 addresses}

\begin{list1}
\item You have plenty!
\item Providers and LIRs will typically get /32
\item Providers will typically give organisations /48 or /56
\item Your /48 can be used for:
\begin{list2}
\item 65536 subnets - all host subnets are /64
\item Each subnet has $2^{64}$ addresses
\end{list2}
\end{list1}

\slide{Preparing an IPv6 Addressing Plan}

\hlkimage{19cm}{ipv6-linked-to-ipv4.png}

\centerline{Easy and coupled with VLAN IDs it will work \smiley}

Source:
{\footnotesize \link{https://www.ripe.net/support/training/material/IPv6-for-LIRs-Training-Course/Preparing-an-IPv6-Addressing-Plan.pdf}}



\slide{Run IPv6 in production}

\begin{list1}
\item Make sure you establish IPv6 in {\bf production}
\item Enabling service on IPv6 without production - bad experience for users
\item Start by enabling your DNS servers for IPv6 - and DNSSEC - and DNS over TCP\\
Remember that your firewall might have problems with large DNS packets
\item Add a production IPv6 router - hardware device or generic server
\item Tunnels are OK, and SixXS consider their service production
\end{list1}


\slide{Conclusion}

\begin{center}
\vskip 5mm
{\color{titlecolor}\LARGE \bf IPv6 is here already - use it}
\vskip 5mm


\link{http://www.ipv6actnow.org/}

\link{http://www.ipv6tf.dk} - anyone want to reboot it?

\end{center}

\begin{list1}
%\item Join the fun - join the wireless network
\item Use ping/ping6 and traceroute to test connectivity
\item Try in your browser:
\begin{list2}
\item \link{http://www.kame.net} Dancing turtle
\item \link{http://www.ripe.net} RIPE, look for address up right corner
\item \link{http://loopsofzen.co.uk/} Play a game
\item \link{https://www.sixxs.net/} Apply for IPv6 tunnel
\end{list2}
\item Done \smiley
\end{list1}

\end{document}
