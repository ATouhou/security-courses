

\slide{Internet today}

\hlkimage{14cm}{images/server-client.pdf}

\begin{list1}
\item Clients and servers
\item Rooted in academic networks
\item Protocols which are more than 20 years old, moved to TCP/IP in 1981 
\end{list1}


\slide{Hvad er Internet}

\begin{list1}
\item Kommunikation mellem mennesker!
\item Baseret p� TCP/IP
\begin{list2}
\item best effort
\item packet switching (IPv6 kalder det packets, ikke datagram)
\item forbindelsesorienteret, \emph{connection-oriented}
\item forbindelsesl�s, \emph{connection-less}
\end{list2}
\end{list1}

RFC-1958:
\begin{quote}
 A good analogy for the development of the Internet is that of
 constantly renewing the individual streets and buildings of a city,
 rather than razing the city and rebuilding it. The architectural
 principles therefore aim to provide a framework for creating
 cooperation and standards, as a small "spanning set" of rules that
 generates a large, varied and evolving space of technology.
\end{quote}


\slide{Hvad er Internet}

\begin{list1}
\item 80'erne IP/TCP starten af 80'erne
\item 90'erne IP version 6 udarbejdes
  \begin{list2}
  \item IPv6 ikke brugt i Europa og US
  \item IPv6 er ekstremt vigtigt i Asien 
  \item historisk f� adresser tildelt til 3.verdenslande
  \item St�rre Universiteter i USA har ofte st�rre allokering end Kina!
  \end{list2}
\item 1991 WWW "opfindes" af Tim Berners-Lee hos CERN
\item E-mail var hovedparten af traffik
  - siden overtog web/http f�rstepladsen
\end{list1}


% IP-adresser

\slide{F�lles adresserum}

\vskip 2 cm
\hlkimage{17cm}{IP-address.pdf}

\begin{list1}
\item Hvad kendetegner internet idag
\item Der er et f�lles adresserum baseret p� 32-bit adresser
\item En IP-adresse kunne v�re 10.0.0.1
\end{list1}


\slide{IPv6 addresser og skrivem�de}

\hlkimage{20cm}{ipv6-address-1.pdf}

\begin{list2}
\item 128-bit adresser, subnet prefix n�sten altid 64-bit
\item skrives i grupper af 4 hexcifre ad gangen adskilt af kolon :
\item foranstillede 0 i en gruppe kan udelades, en r�kke 0 kan erstattes med ::
\item dvs 0:0:0:0:0:0:0:0 er det samme som \\
0000:0000:0000:0000:0000:0000:0000:0000
\item Dvs min webservers IPv6 adresse kan skrives som: 
2001:16d8:ff00:12f::2
\item Specielle adresser:
::1 localhost/loopback og
::  default route
\item L�s mere i RFC-3513
\end{list2}

\slide{IPv6 header - RFC-2460}


\begin{alltt}
\footnotesize

   +-+-+-+-+-+-+-+-+-+-+-+-+-+-+-+-+-+-+-+-+-+-+-+-+-+-+-+-+-+-+-+-+
   |Version| Traffic Class |           Flow Label                  |
   +-+-+-+-+-+-+-+-+-+-+-+-+-+-+-+-+-+-+-+-+-+-+-+-+-+-+-+-+-+-+-+-+
   |         Payload Length        |  Next Header  |   Hop Limit   |
   +-+-+-+-+-+-+-+-+-+-+-+-+-+-+-+-+-+-+-+-+-+-+-+-+-+-+-+-+-+-+-+-+
   |                                                               |
   +                                                               +
   |                                                               |
   +                         Source Address                        +
   |                                                               |
   +                                                               +
   |                                                               |
   +-+-+-+-+-+-+-+-+-+-+-+-+-+-+-+-+-+-+-+-+-+-+-+-+-+-+-+-+-+-+-+-+
   |                                                               |
   +                                                               +
   |                                                               |
   +                      Destination Address                      +
   |                                                               |
   +                                                               +
   |                                                               |
   +-+-+-+-+-+-+-+-+-+-+-+-+-+-+-+-+-+-+-+-+-+-+-+-+-+-+-+-+-+-+-+-+
\end{alltt}


\slide{IPv6 addressing RFC-4291}

\begin{list1}
\item Addresses are always 128-bit identifiers for interfaces and sets of
   interfaces 
\item Unicast:   An identifier for a {\bf single interface}.\\
A packet sent to a
               unicast address is delivered to the interface identified
               by that address.
\item Anycast:   An identifier for a {\bf set of interfaces} (typically
               belonging to different nodes).\\  A packet sent to an
               anycast address is {\bf delivered to one} of the interfaces
               identified by that address (the "nearest" one, according
               to the routing protocols' measure of distance).

\item Multicast: An identifier for a {\bf set of interfaces} (typically
               belonging to different nodes). \\ A packet sent to a
               multicast address is {\bf delivered to all interfaces
               identified by that address}.
\end{list1}

\slide{Core, Distribution og Access net}

\hlkimage{20cm}{core-dist.pdf}

\centerline{Det er ikke altid man har pr�cis denne opdeling, men den er ofte brugt}


\sout{Where are the NAT gateways?}

\slide{Pakker i en datastr�m}

\hlkimage{23cm}{ethernet-frame-1.pdf}
\begin{list1}
\item Ser vi data som en datastr�m er pakkerne blot et m�nster lagt henover data 
\item Netv�rksteknologien definerer start og slut p� en frame
\item Fra et lavere niveau modtager vi en pakke, eksempelvis 1500-bytes fra Ethernet driver
\end{list1}


\slide{Collect information about your network}

\begin{list1}
\item devices - what is a network device?
\item switches - Layer 2 does not matter much, management by RFC-1918 IPv4 is probably wise
\item routers - most important, connectivity MUST support IPv6. Check vendor home page - do NOT assume support is ready
\item Security devices: firewalls, IDS/IPS, VPN - critical and support in general poor. Some vendors such as Cisco ASA and Juniper SRX has good support

\end{list1}

\slide{Hello neighbors}

\begin{alltt}\small
$ ping6 -w -I en1 ff02::1
PING6(72=40+8+24 bytes) fe80::223:6cff:fe9a:f52c%en1 --> ff02::1
30 bytes from fe80::223:6cff:fe9a:f52c%en1: bigfoot
36 bytes from fe80::216:cbff:feac:1d9f%en1: mike.kramse.dk.
38 bytes from fe80::200:aaff:feab:9f06%en1: xrx0000aaab9f06
34 bytes from fe80::20d:93ff:fe4d:55fe%en1: harry.local
36 bytes from fe80::200:24ff:fec8:b24c%en1: kris.kramse.dk.
31 bytes from fe80::21b:63ff:fef5:38df%en1: airport5
32 bytes from fe80::216:cbff:fec4:403a%en1: main-base
44 bytes from fe80::217:f2ff:fee4:2156%en1: Base Station Koekken 
35 bytes from fe80::21e:c2ff:feac:cd17%en1: arnold.local
\end{alltt}


\slide{Vigtigste protokoller}


\begin{list1}
\item ARP Address Resolution Protocol
\item IP og ICMP Internet Control Message Protocol
\item UDP User Datagram Protocol
\item TCP Transmission Control Protocol
\item DHCP Dynamic Host Configuration Protocol 
\item DNS Domain Name System
\end{list1}
\vskip 1cm
\centerline{Ovenst�ende er omtrent minimumskrav for at komme p� internet}

% allerede gennemg�et ovenfor
%\slide{ICMP}

%\begin{list1}
%\item 	Internet Control Message Protocol 
%	Defineret i RFC-792

%\end{list1}


\slide{UDP User Datagram Protocol}
\hlkimage{20cm}{udp-1.pdf}
\begin{list1}
\item Forbindelsesl�s RFC-768, \emph{connection-less} - der kan tabes pakker
\item Kan benyttes til multicast/broadcast - flere modtagere
\end{list1}



\slide{TCP Transmission Control Protocol}
\hlkimage{20cm}{tcp-1.pdf}

\begin{list1}
\item Forbindelsesorienteret RFC-791 September 1981, \emph{connection-oriented}
\item Enten overf�res data eller man f�r fejlmeddelelse
\end{list1}




\slide{TCP three way handshake}

\hlkimage{7cm}{images/tcp-three-way.pdf}

\begin{list2}
\item {\bfseries TCP SYN half-open} scans
\item Tidligere loggede systemer kun n�r der var etableret en fuld TCP
  forbindelse - dette kan/kunne udnyttes til \emph{stealth}-scans
\item Hvis en maskine modtager mange SYN pakker kan dette fylde
  tabellen over connections op - og derved afholde nye forbindelser
  fra at blive oprette - {\bfseries SYN-flooding}
\end{list2}

\slide{Well-known port numbers}

\hlkimage{10cm}{iana1.jpg}

\begin{list1}
\item IANA vedligeholder en liste over magiske konstanter i IP
\item De har lister med hvilke protokoller har hvilke protokol ID m.v.
\item En liste af interesse er port numre, hvor et par eksempler er:
\begin{list2}
\item Port 25 SMTP Simple Mail Transfer Protocol
\item Port 53 DNS Domain Name System
\item Port 80 HTTP Hyper Text Transfer Protocol over TLS/SSL
\item Port 443 HTTP over TLS/SSL
\end{list2}
\item Se flere p� \link{http://www.iana.org}
\end{list1}



\slide{DHCP Dynamic Host Configuration Protocol}

\hlkimage{13cm}{dhcp-1.pdf}

\begin{list1}
\item Hvordan f�r man information om default gateway
\item Man sender et DHCP request og modtager et svar fra en DHCP server
\item Dynamisk konfiguration af klienter fra en centralt konfigureret server
\item Bruges til IP adresser og meget mere
\end{list1}

\slide{IPv6 router advertisement daemon}

\begin{alltt}
/etc/rtadvd.conf:
en0:
      :addrs#1:addr="2001:1448:81:b00f::":prefixlen#64:
en1:
      :addrs#1:addr="2001:1448:81:beef::":prefixlen#64:

root# /usr/sbin/rtadvd -Df en0 en1
root# sysctl -w net.inet6.ip6.forwarding=1
net.inet6.ip6.forwarding: 0 -> 1
\end{alltt}

\begin{list1}
\item Stateless autoconfiguration er en stor ting i IPv6
\item Kommandoen starter den i debug-mode og i forgrunden\\
- normalt vil man starte den fra et script 
\item Typisk skal forwarding aktiveres, som vist med BSD sysctl kommando
\item NB: de fleste clients vil idag implementere IPv6 privacy addresses
\end{list1}

\slide{IPv6 autoconfiguration}

\hlkimage{24cm}{modified-eui64.pdf}

\begin{list1}
\item DHCPv6 is available, but {\bfseries stateless autoconfiguration} is king
\item Routers announce subnet prefix via {\bfseries router advertisements}
\item Individual nodes then combine this with their EUI64 identifier
\end{list1}

%\link{http://www.cisco.com/web/about/ac123/ac147/archived_issues/ipj_7-2/ipv6_autoconfig.html}

\slide{Router advertisement daemon}

\hlkimage{14cm}{ipv6-router-advertisement.png}




% 802.1q
\slide{VLAN Virtual LAN}

\hlkimage{10cm}{vlan-portbased.pdf}

\begin{list1}
\item Nogle switche tillader at man opdeler portene
\item Denne opdeling kaldes VLAN og portbaseret er det mest simple
\item Port 1-4 er et LAN
\item De resterende er et andet LAN
\item Data skal omkring en firewall eller en router for at krydse fra VLAN1 til VLAN2
\end{list1}

\slide{IEEE 802.1q}

\hlkimage{19cm}{vlan-8021q.pdf}

\begin{list1}
\item Nogle switche tillader konfiguration med 802.1q VLAN tagging p� Ethernet niveau
\item Data skal omkring en firewall eller en router for at krydse fra VLAN1 til VLAN2
\item VLAN trunking giver mulighed for at dele VLANs ud p� flere switches
\item Der findes administrationsv�rkt�jer der letter dette arbejde: OpenNAC FreeNAC, Cisco VMPS
\end{list1}


%DNS
\slide{Domain Name System}

\hlkimage{12cm}{dns-1.pdf}

\begin{list1}
\item Gennem DHCP f�r man typisk ogs� information om DNS servere
\item En DNS server kan sl� navne, dom�ner og adresser op
\item Foreg�r via query og response med datatyper kaldet resource records
\item DNS er en distribueret database, s� opslag kan resultere i flere opslag
\end{list1}

\slide{Mere end navneopslag}

\begin{list1}
  \item best�r af resource records med en type:
    \begin{list2}
\item adresser A-records
\item IPv6 adresser AAAA-records
\item autoritative navneservere NS-records
\item post, mail-exchanger MX-records
\item flere andre: md ,  mf ,  cname ,  soa ,
                  mb , mg ,  mr ,  null ,  wks ,  ptr ,
                  hinfo ,  minfo ,  mx ....
\end{list2}
\end{list1}
\begin{alltt}
ns1     IN      A       217.157.20.130
        IN      AAAA    2001:618:433::1
www     IN      A       217.157.20.131
        IN      AAAA    2001:618:433::14
        IN      MX      10      mail.security6.net.
        IN      MX      20      mail2.security6.net.
\end{alltt}


