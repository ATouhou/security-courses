\slide{Kryptografi}


\begin{list1}
\item Kryptografi er l�ren om, hvordan man kan kryptere data
\item Kryptografi benytter algoritmer som sammen med n�gler giver en
  ciffertekst - der kun kan l�ses ved hj�lp af den tilh�rende n�gle
\item privat-n�gle kryptografi (eksempelvis AES) benyttes den samme
  n�gle til kryptering og dekryptering 
\item offentlig-n�gle kryptografi (eksempelvis RSA) benytter to
  separate n�gler til kryptering og dekryptering
\end{list1}


\slide{Kryptografiske principper}

\begin{list1}
\item Algoritmerne er kendte
\item N�glerne er hemmelige
\item N�gler har en vis levetid - de skal skiftes ofte
\item Et successfuldt angreb p� en krypto-algoritme er enhver genvej
  som kr�ver mindre arbejde end en gennemgang af alle n�glerne 
\item Nye algoritmer, programmer, protokoller m.v. skal gennemg�s n�je!
\item Se evt. Snake Oil Warning Signs:
Encryption Software to Avoid 
\link{http://www.interhack.net/people/cmcurtin/snake-oil-faq.html}
\end{list1}

\slide{DES, Triple DES og AES}

\hlkimage{15cm}{images/AES_head.png}

\begin{list1}
\item DES kryptering baseret p� den IBM udviklede Lucifer algoritme
  har v�ret benyttet gennem mange �r. 
\item Der er vedtaget en ny standard algoritme Advanced Encryption
  Standard (AES) som afl�ser Data Encryption Standard (DES)
\item Algoritmen hedder Rijndael og er udviklet
af Joan Daemen og Vincent Rijmen.
%\item \emph{Rijndael is available for free. You can use it for
%whatever purposes  you want, irrespective of whether
%it is accepted as AES or not.}

\item Kilde:
\link{http://csrc.nist.gov/encryption/aes/}\\
\href{http://www.esat.kuleuven.ac.be/~rijmen/rijndael/}
{http://www.esat.kuleuven.ac.be/\~{}rijmen/rijndael/}
\end{list1}


\slide{Form�let med kryptering}

\vskip 3 cm
\centerline{\hlkbig kryptering er den eneste m�de at sikre:}
\vskip 3 cm
\centerline{\hlkbig fortrolighed}
\vskip 3 cm
\centerline{\hlkbig autenticitet / integritet}


\slide{Secure protocols}

\begin{list1}
\item Securing e-mail
\begin{list2}
\item Pretty Good Privacy - Phil Zimmermann
\item OpenPGP = e-mail security
\end{list2}
\item Network sessions use SSL/TLS
\begin{list2}
\item Secure Sockets Layer SSL / Transport Layer Services TLS
\item Encrypting data sent and received
\item SSL/TLS already used for many protocols as a wrapper: POP3S, IMAPS, SSH, SMTP+TLS m.fl.
\end{list2}
\item Encrypting traffic at the network layer - Virtual Private Networks VPN
\begin{list2}
\item {\color{green}IPsec IP Security Framework, se ogs� L2TP}
\item {\color{red} PPTP Point to Point Tunneling Protocol - d�rlig og usikker, brug den ikke mere!}
\item OpenVPN uses SSL/TLS across TCP or UDP
\end{list2}
\end{list1}

\centerline{Note: SSL/TLS is not trivial to implement, key management!}


%%% Local Variables: 
%%% mode: latex
%%% TeX-master: "tcpip-security"
%%% End: 
