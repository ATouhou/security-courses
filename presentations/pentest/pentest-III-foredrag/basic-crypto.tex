\slide{Cryptography}


\begin{list1}
\item Cryptography or cryptology is the practice and study of techniques for secure communication
\item Modern cryptography is heavily based on mathematical theory and computer science practice; cryptographic algorithms are designed around computational hardness assumptions, making such algorithms hard to break in practice by any adversary
\item Symmetric-key cryptography refers to encryption methods in which both the sender and receiver share the same key, to ensure confidentiality, example algorithm AES
\item Public-key cryptography (like RSA) uses two related keys, a key pair of a public key and a private key. This allows for easier key exchanges, and can provide confidentiality, and methods for signatures and other services
\end{list1}

Source: \link{https://en.wikipedia.org/wiki/Cryptography}

\slide{Kryptografiske principper}

\begin{list1}
\item Algoritmerne er kendte
\item Nøglerne er hemmelige
\item Nøgler har en vis levetid - de skal skiftes ofte
\item Et successfuldt angreb på en krypto-algoritme er enhver genvej
  som kræver mindre arbejde end en gennemgang af alle nøglerne
\item Nye algoritmer, programmer, protokoller m.v. skal gennemgås nøje!
\item Se evt. Snake Oil Warning Signs:
Encryption Software to Avoid\\
\link{http://www.interhack.net/people/cmcurtin/snake-oil-faq.html}
\end{list1}

\slide{DES, Triple DES og AES}

\hlkimage{15cm}{images/AES_head.png}

\begin{list1}
\item DES kryptering baseret på den IBM udviklede Lucifer algoritme
  har været benyttet gennem mange år
\item Der blev i 2001 vedtaget en ny standard algoritme Advanced Encryption
  Standard (AES) som afløser Data Encryption Standard (DES)
\item Algoritmen hedder Rijndael og er udviklet
af Joan Daemen og Vincent Rijmen.
%\item \emph{Rijndael is available for free. You can use it for
%whatever purposes  you want, irrespective of whether
%it is accepted as AES or not.}
\item Se også \link{https://en.wikipedia.org/wiki/Advanced_Encryption_Standard}
\end{list1}


\slide{Formålet med kryptering}

\vskip 3 cm
\centerline{\hlkbig kryptering er den eneste måde at sikre:}
\vskip 3 cm
\centerline{\hlkbig fortrolighed}
\vskip 3 cm
\centerline{\hlkbig autenticitet / integritet}


%%% Local Variables:
%%% mode: latex
%%% TeX-master: "tcpip-security"
%%% End:
