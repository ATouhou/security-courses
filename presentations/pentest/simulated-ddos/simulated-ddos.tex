\documentclass[20pt,landscape,a4paper,footrule]{foils}
\usepackage{zencurity-slides}

%\externaldocument{unix-audit-security-oevelser}
\externaldocument{\jobname-exercises}



% DDoS Attacking, breaking the firewall infrastructure

% This presentation is about DDoS attack simulation, what it requires, what are the typical results from testing, and improvements after testing. If you don’t test, you haven’t got DDoS protection. The results and experience is gathered from during active penetration testing and DDoS simulation against largish customers such as banks.

% Bio, Henrik Kramshoej

% Henrik is a grumpy old internet samurai working with internet security.
% Henrik loves IPv6 and sending small packets to discover open ports and find security vulnerabilities

\begin{document}
\selectlanguage{danish}

\mytitlepage{Simulated DDoS Attacks, breaking the firewall infrastructure}


\slide{Goal}

\vskip 2 cm

\hlkimage{5cm}{dont-panic.png}
\centerline{\color{titlecolor}\LARGE Don't Panic!}

\begin{list1}
\item How to create DDoS simulations
\item Some actual experience with doing this
\item Evaluate how good is this, value
\end{list1}

\vskip 1cm
\centerline{I use Kali 2.0 Linux for this}

\slide{Kali Linux the new backtrack}

\hlkimage{\linewidth-2cm}{kali-linux.png}

\begin{list1}
\item BackTrack \link{http://www.backtrack-linux.org}
\item  Kali \link{http://www.kali.org/}
\end{list1}



\slide{hping3 packet generator}




\begin{alltt}\small
usage: hping3 host [options]
  -i  --interval  wait (uX for X microseconds, for example -i u1000)
      --fast      alias for -i u10000 (10 packets for second)
      --faster    alias for -i u1000 (100 packets for second)
      --flood	   sent packets as fast as possible. Don't show replies.
...
hping3 is fully scriptable using the TCL language, and packets
can be received and sent via a binary or string representation
describing the packets.
\end{alltt}


Hping3 packet generator is a very flexible tool to produce simulated DDoS traffic with specific charateristics

Home page: \link{http://www.hping.org/hping3.html}\\
Source repository \link{https://github.com/antirez/hping}


\slide{t50 packet generator}


\begin{alltt}\small
root@cornerstone03:~# t50 -?
T50 Experimental Mixed Packet Injector Tool 5.4.1
Originally created by Nelson Brito <nbrito@sekure.org>
Maintained by Fernando Mercês <fernando@mentebinaria.com.br>

Usage: T50 <host> [/CIDR] [options]

Common Options:
    --threshold NUM           Threshold of packets to send     (default 1000)
    --flood                   This option supersedes the 'threshold'
...
6. Running T50 with '--protocol T50' option, sends ALL protocols sequentially.
root@cornerstone03:~# t50 -? | wc -l
264
\end{alltt}

T50 packet generator, another high speed packet generator which can easily overload most firewalls by producing a randomized traffic with multiple protocols like IPsec, GRE, MIX \\
home page: \link{http://t50.sourceforge.net/resources.html}

\slide{Before testing: Smokeping}


\hlkimage{\linewidth-5cm}{smokeping-before-testing.png}

\centerline{Before DDoS testing  use Smokeping software}

\slide{Before testing: Pingdom}

\hlkimage{\linewidth-5cm}{forside-pingdom.png}

\centerline{Another external monitoring from Pingdom.com}


\slide{Process}

\begin{list2}
\item Start small, run with delays between packets
\item Turn up until it breaks,
\item Monitor speed of attack on your router interface pps/bandwidth
\item Give it all shes got\\
 \verb+hping3 --flood -1+ and \verb+hping3 --flood -2+
\item Have a common chat to talk about symptoms and things observed
\end{list2}

Comparable to real DDoS?

Tools are simple and widely available but are they actually producing same result as high-powered and advanced criminal botnets. We can confirm that the attack delivered in this test is, in fact, producing the traffic patterns very close to criminal attacks in real-life scenarios.

\slide{Running hping3}

\begin{alltt}\small
# export CUST_IP=192.0.2.1
# date;time hping3 -q -c 1000000  -i u60 -S -p 80  $CUST_IP
 \end{alltt}

\begin{alltt}\small
#  date;time hping3 -q -c 1000000  -i u60 -S -p 80  $CUST_IP
Thu Jan 21 22:37:06 CET 2016
HPING 192.0.2.1 (eth0 192.0.2.1): S set, 40 headers + 0 data bytes

--- 192.0.2.1 hping statistic ---
1000000 packets transmitted, 999996 packets received, 1% packet loss
round-trip min/avg/max = 0.9/7.0/1005.5 ms

real	1m7.438s
user	0m1.200s
sys	0m5.444s
\end{alltt}

\vskip 1cm
\centerline{Dont forget to do a killall hping3 when done \smiley }

\slide{Experiences from testing}

How much bandwidth can big danish companies handle?
\begin{list2}
\item A) 10-100Mbps
\item B) 100Mbps -1Gbit
\item C) Up to 5Gbit easily
\end{list2}

How much abuse in pps can big danish companies handle?
\begin{list2}
\item A) 10.000 - 50.000 pps
\item B) 50 - 500k pps
\item C) Up to 5 million pps
\end{list2}

\slide{Running the tools}

A minimal test would be:
\begin{list2}
\item TCP SYN flooding
\item TCP other flags, PUSH-ACK, RST, ACK, FIN
\item ICMP flooding
\item UDP flooding
\item Spoofed packets src=dst=target \smiley
\item Small fragments
\item Bad fragment offset
\item Bad checksum
\item Be creative
\item Mixed packets - like \verb+t50 --protocol T50+
\item Perhaps esoteric or unused protocols, GRE, IPSec
\end{list2}


\slide{Rocky Horror Picture Show - 1}

\hlkimage{20cm}{smokeping-1.png}

\centerline{Really does it break from 50.000 pps SYN attack?}

\slide{Rocky Horror Picture Show - 2}

\hlkimage{20cm}{smokeping-2.png}

\centerline{Oh no 500.000 pps UDP attacks work?}

\slide{Rocky Horror Picture Show - 3}

\centerline{Oh no spoofing attacks work?}

\hlkimage{20cm}{smokeping-3.png}



\slide{Experiences from testing}

How much bandwidth can big danish companies handle!
\begin{list2}
\item B) {\bf 100Mbps -1Gbit}
\end{list2}

How much abuse in pps can big danish companies handle!
\begin{list2}
\item B) {\bf 50.000 - 500k pps} TCP attacks
\item B) {\bf 500.000 - 1mill pps} UDP or ICMP attacks
\item Ohhh and often we can spoof using their addresses in the first test
\end{list2}

Even the DDoS protection services are a bit too small, can handle perhaps 10G?

and also multiple times admins lost access to network, VPN, log overflow etc.

\vskip 2cm
Note: attackers can send full 10Gbit 14mill pps from Core i7 with 3 cores ...

\slide{Improvements seen after testing}

\begin{list1}
\item Turning off unneeded features - free up resources
\item Tuning sesions, max sessions src / dst
\item Tuning firewalls, max sessions in half-open state, enabling services
\item Tuning network, drop spoofed src from inside net \smiley
\item Tuning network, can follow logs, manage network during attacks
\item ...
\item And organisation has better understanding of DDoS challenges
\item Including vendors, firewall consultants, ISPs etc.
\end{list1}

\vskip 1cm
\centerline{After tuning of {\bf existing devices/network} improves results 10-100 times}

\slide{Conclusion}

\hlkrightimage{15cm}{network-layers-1.png}
.
\begin{list1}
\item You really should try testing
\item Investigate your existing devices\\
all of them, RTFM, upgrade firmware
\item Choose which devices does which\\
part - discard early to free resources\\
for later devices to dig deeper
\end{list1}

\vskip 2cm
\centerline{And dont forget that DDoS testing is as much a firedrill for the organisation}

\slide{More application testing}

\hlkimage{12cm}{images/linux-tsung-size_rcv.png}

\begin{list1}
\item We covered only lower layers - but helpful layer 7 testing programs exist
\item Tsung can be used to stress HTTP, WebDAV, SOAP, PostgreSQL, MySQL, LDAP and Jabber/XMPP servers \link{http://tsung.erlang-projects.org/}
\end{list1}

\myquestionspage

\end{document}
