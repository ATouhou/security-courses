\documentclass[20pt,landscape,a4paper]{foils}

% HLK/ security6
%\usepackage{sec6slides}
\usepackage{solido-network-slides}
%\usepackage{crypto-slides}


%\externaldocument{unix-audit-security-oevelser}
\externaldocument{\jobname-exercises}

% HK Klubben hos Energinet.dk, har inviteret Henrik Kramshøj til at komme og fortælle om webbaserede angreb, netværkssikkerhed og kryptografi.

% Basic things that we need are below
\begin{document}
\selectlanguage{danish}

\mytitlepage
{Hacking today}{}


\slide{Formålet med foredraget}

\vskip 2 cm

\hlkimage{5cm}{dont-panic.png}
\centerline{\color{titlecolor}\LARGE Don't Panic!}

\begin{list1}
\item Skabe forståelse for hackerværktøjer
  samt penetrationstest metoder

\item webbaserede angreb
\item netværkssikkerhed
\item lidt kryptografi

\end{list1}


\slide{Hacker - cracker}

{\bfseries Det korte svar - drop diskussionen}

%Det lidt længere svar:\\
Det havde oprindeligt en anden betydning, men medierne har taget
udtrykket til sig - og idag har det begge betydninger.

{\color{red}\hlkbig Idag er en hacker stadig en der bryder ind i systemer!}

ref. Spafford, Cheswick, Garfinkel, Stoll, ...
- alle kendte navne indenfor sikkerhed

Hvis man vil vide mere kan man starte med:
\begin{list2}
\item \emph{Cuckoo's Egg: Tracking a Spy Through the Maze of Computer
 Espionage},  Clifford Stoll
\item \emph{Hackers: Heroes of the Computer Revolution},
Steven Levy
\item \emph{Practical Unix and Internet Security},
Simson Garfinkel, Gene Spafford, Alan Schwartz
\end{list2}



\slide{Hacking er magi}

\hlkimage{7cm}{wizard_in_blue_hat.png}

\vskip 1 cm

\centerline{Hacking ligner indimellem  magi}


\slide{Hacking er ikke magi}

\hlkimage{17cm}{ninjas.png}

\vskip 1 cm
\centerline{Hacking kræver blot lidt ninja-træning}

\slide{Movie:Kryptonite lock - old}

\hlkimage{16cm}{youtube-bic-lock.png}

\begin{list1}
\item Just search for: kryptonite lock bic pen
\item \link{https://www.youtube.com/watch?v=LahDQ2ZQ3e0}
\end{list1}



\slide{Hacking eksempel - det er ikke magi}

\begin{list1}
\item MAC filtrering på trådløse netværk
\item Alle netkort har en MAC adresse - BRÆNDT ind i kortet fra fabrikken
\item Mange trådløse Access Points kan filtrere MAC adresser
\item Kun kort som er på listen over godkendte adresser tillades adgang til netværket
\pause
\item Det virker dog ikke \smiley
\item De fleste netkort tillader at man overskriver denne adresse midlertidigt
\item Derudover har der ofte været fejl i implementeringen af MAC filtrering
\end{list1}

\slide{Myten om MAC filtrering}

\begin{list1}
\item Eksemplet med MAC filtrering er en af de mange myter
\item Hvorfor sker det?
\item Marketing - producenterne sætter store mærkater på æskerne
\item Manglende indsigt - forbrugerne kender reelt ikke koncepterne
\item Hvad \emph{er} en MAC adresse egentlig
\item Relativt få har forudsætningerne for at gennemskue dårlig sikkerhed
\item Løsninger?
\pause
\item Udbrede viden om usikre metoder til at sikre data og computere
\item Udbrede viden om sikre metoder til at sikre data og computere
\end{list1}

\slide{MAC filtrering}

\hlkimage{15cm}{stupid-security.jpg}




\slide{Attack overview}

\hlkimage{19cm}{sicherheitstacho.png}

{\small\link{http://www.sicherheitstacho.eu/?lang=en}}




\slide{Heartbleed CVE-2014-0160}

\hlkimage{22cm}{heartbleed-com.png}

Source: \link{http://heartbleed.com/}


\slide{Heartbleed hacking}

\begin{alltt}\footnotesize
  06b0: 2D 63 61 63 68 65 0D 0A 43 61 63 68 65 2D 43 6F  -cache..Cache-Co
  06c0: 6E 74 72 6F 6C 3A 20 6E 6F 2D 63 61 63 68 65 0D  ntrol: no-cache.
  06d0: 0A 0D 0A 61 63 74 69 6F 6E 3D 67 63 5F 69 6E 73  ...action=gc_ins
  06e0: 65 72 74 5F 6F 72 64 65 72 26 62 69 6C 6C 6E 6F  ert_order&billno
  06f0: 3D 50 5A 4B 31 31 30 31 26 70 61 79 6D 65 6E 74  =PZK1101&payment
  0700: 5F 69 64 3D 31 26 63 61 72 64 5F 6E 75 6D 62 65  _id=1&{\bf card_numbe}
  0710: XX XX XX XX XX XX XX XX XX XX XX XX XX XX XX XX  {\bf r=4060xxxx413xxx}
  0720: 39 36 26 63 61 72 64 5F 65 78 70 5F 6D 6F 6E 74  {\bf 96&card_exp_mont}
  0730: 68 3D 30 32 26 63 61 72 64 5F 65 78 70 5F 79 65  {\bf h=02&card_exp_ye}
  0740: 61 72 3D 31 37 26 63 61 72 64 5F 63 76 6E 3D 31  {\bf ar=17&card_cvn=1}
  0750: 30 39 F8 6C 1B E5 72 CA 61 4D 06 4E B3 54 BC DA  {\bf 09}.l..r.aM.N.T..
\end{alltt}

\begin{list2}
\item Obtained using Heartbleed proof of concepts - Gave full credit card details
\item "can XXX be exploited" - yes, clearly! PoCs ARE needed\\
without PoCs even Akamai wouldn't have repaired completely!
\item The internet was ALMOST fooled into thinking getting private keys from Heartbleed was not possible - scary indeed.
\end{list2}


\slide{Why is heartbleed different?}

\hlkimage{3cm}{heartbleed.png}
\begin{list1}
\item Great PR, name, web site, logo
\item OpenSSL is very widespread
\item OpenSSL has been criticized before
\item The spotlight is now on a lot of products, infrastructure
\item BOTH Open Source products and Proprietary products hurt by this
\item TL;DR\\ OpenSSL is everywhere and an example of our dependency on weak components
\end{list1}

\slide{Key points after heartbleed}

\hlkimage{16cm}{ssl-tls-breaks-timeline.png}
Source: picture source\\ {\footnotesize\link{https://www.duosecurity.com/blog/heartbleed-defense-in-depth-part-2}}
\begin{list2}
\item Writing SSL software and other secure crypto software is hard
\item Configuring SSL is hard\\
check you own site \link{https://www.ssllabs.com/ssltest/}
\item SSL is hard, finding bugs "all the time"
\link{https://armoredbarista.blogspot.dk/2013/01/a-brief-chronology-of-ssltls-attacks.html}
\item Rekeying is hard - slow, error prone, manual proces - Automate!
\item Proof of concept programs exist - good or bad?
\end{list2}


\slide{Most vulnerable operating systems in 2014}

\hlkimage{18cm}{GFI-vulns-2014-OS-chart.jpg}

\begin{quote}
An average of 19 vulnerabilities per day were reported in 2014, according to the data from the National Vulnerability Database (NVD).
\end{quote}

Source:\\
{\footnotesize
\link{https://www.gfi.com/blog/most-vulnerable-operating-systems-and-applications-in-2014/}}


\slide{Most vulnerable applications in 2014}

\hlkimage{16cm}{gfi-vulns-application-chart.jpg}

\begin{quote}\small
Not surprisingly at all, web browsers continue to have the most security vulnerabilities because they are a popular gateway to access a server and to spread malware on the clients.
\end{quote}

Source:\\
{\footnotesize
\link{https://www.gfi.com/blog/most-vulnerable-operating-systems-and-applications-in-2014/}}



\slide{Hackerværktøjer}
% mæske til reference afsnit?

\begin{list1}
\item Vi benytter en del værktøjer:
\begin{list2}
\item Nmap - \link{http://www.insecure.org} portscanner
\item Wireshark - \link{https://www.wireshark.org/} avanceret netværkssniffer
\item Kali Linux \link{https://www.kali.org/}
\item Burp is a highly recommended commercial Web proxy EUR 275/user/year ~2.000DKK\\
\link{https://portswigger.net/burp/help/suite_gettingstarted.html}
\end{list2}

\end{list1}


Hackerværktøjer er dem som gør noget anderledes for at opnå fordel

\slide{OSI og Internet modellerne}

\hlkimage{14cm,angle=90}{images/compare-osi-ip.pdf}


\slide{Wireshark - grafisk pakkesniffer}

\hlkimage{20cm}{images/wireshark-website.png}

\centerline{\link{https://www.wireshark.org}}
\centerline{både til Windows og UNIX}

\slide{Wireshark usage}
\hlkimage{18cm}{wireshark-http.png}

Wireshark: Filters, hexdump, protocol dissection, overview, coloring, advanced features

\demo{Wireshark}


\slide{Network mapping}

\hlkimage{23cm}{images/network-example.pdf}

\begin{list1}
\item Ved brug af traceroute og tilsvarende programmer kan man ofte
  udlede topologien i det netværk man undersæger
\end{list1}


\slide{Portscan med Zenmap GUI}

\hlkimage{16cm}{nmap-zenmap.png}



\demo{Armitage og Metasploit}


\slide{FREAK March 2015}


\begin{quote}
"A group of cryptographers at INRIA, Microsoft Research and IMDEA have discovered some serious vulnerabilities in OpenSSL (e.g., Android) clients and Apple TLS/SSL clients (e.g., Safari) that allow a 'man in the middle attacker' to downgrade connections from 'strong' RSA to 'export-grade' RSA. These attacks are real and exploitable against a shocking number of websites -- including government websites. Patch soon and be careful."
\end{quote}

Source: Matthew Green, cryptographer and research professor at Johns Hopkins Univ\\
{\tiny\link{http://blog.cryptographyengineering.com/2015/03/attack-of-week-freak-or-factoring-nsa.html}
\link{https://www.smacktls.com/} \link{https://freakattack.com/}
}


OpenSSL, LibreSSL, Apple SSL flaw exit exit exit!, Android SSL, certs certs!!!111, SSLv3, Heartbleed, MS TLS

%\vskip 1cm
%\centerline{F this I'm going out drinking beer *drops mic*}



\vskip 1cm
PS From now on its TLS! Not SSL anymore, any SSLv2, SSLv3 is old and vulnerable

\slide{ Wi-Fi Protected Setup, WPS hacking - Reaver}

\begin{quote}
How Reaver Works\\
Now that you've seen how to use Reaver, let's take a quick overview of how Reaver works. The tool takes advantage of a vulnerability in something called Wi-Fi Protected Setup, or WPS. It's a feature that exists on many routers, intended to provide an easy setup process, and it's tied to a PIN that's hard-coded into the device. Reaver exploits a flaw in these PINs; the result is that, with enough time, it can reveal your WPA or WPA2 password.
\end{quote}

\centerline{Hvad betyder ease of use?}

Source: \\
\link{https://code.google.com/p/reaver-wps/}\\
{\footnotesize \link{https://lifehacker.com/5873407/how-to-crack-a-wi+fi-networks-wpa-password-with-reaver}}

\slide{WPS Design Flaws used by Reaver }

\hlkimage{22cm}{wps-design-flaw-1.png}

\centerline{Pin only, no other means necessary}

Source:\\
\link{https://sviehb.files.wordpress.com/2011/12/viehboeck_wps.pdf}



\slide{WPS Design Flaws used by Reaver }

\hlkimage{14cm}{wps-design-flaw-2.png}

\centerline{Reminds me of NTLM cracking, crack parts independently}

Source:\\
\link{https://sviehb.files.wordpress.com/2011/12/viehboeck_wps.pdf}




\slide{Cracking passwords}

\begin{list2}
\item Hashcat is the world's fastest CPU-based password recovery tool.
\item oclHashcat-plus is a GPGPU-based multi-hash cracker using a brute-force attack (implemented as mask attack), combinator attack, dictionary attack, hybrid attack, mask attack, and rule-based attack.
\item oclHashcat-lite is a GPGPU cracker that is optimized for cracking performance. Therefore, it is limited to only doing single-hash cracking using Markov attack, Brute-Force attack and Mask attack.
\item John the Ripper password cracker old skool men stadig nyttig
\end{list2}

Source:\\
\link{https://hashcat.net/wiki/}\\
\link{http://www.openwall.com/john/}

\slide{Parallella John}

\hlkimage{20cm}{parallella-john.png}

\link{https://twitter.com/solardiz/status/492037995080712192}

Warning: FPGA hacking - not finished part of presentation

\slide{Stacking Parallella boards}
\hlkimage{16cm}{4BoardStack.jpg}

\link{https://www.parallella.org/power-the-parallella/}


\slide{Bettercrypto.org pretty good advise}

SSL settings for nginx
\hlkimage{15cm}{bettercrypto-nginx.png}

Overview
\begin{quote}
"This whitepaper arose out of the need for system administrators to have an updated,
solid, well researched and thought-through guide for configuring SSL, PGP, SSH and
other cryptographic tools in the post-Snowden age. ... This guide is specifically
written for these system administrators."
\end{quote}

\link{https://bettercrypto.org/}


\slide{February and March 2015, Security Onion updates}

\hlkimage{10cm}{security-onion.png}

\begin{list1}
\item Security Onion 12.04.5.1 ISO image now available, plus
Suricata IDS engine 2.0.7
\end{list1}

\centerline{Learn NSM with Security Onion today - its free}

Source:\\
\link{http://blog.securityonion.net/}
%\link{http://blog.securityonion.net/2015/02/security-onion-120451-iso-image-now.html}\\
%\link{http://blog.securityonion.net/2015/03/suricata-207.html}


% Suricata, Logstash, Elasticsearch, D3JShttp://d3js.org/
\slide{Kibana 4 february 2015}

\hlkimage{14cm}{kibanascreenshothomepagebannerbigger.jpg}

\centerline{Highly recommended for a lot of data visualisation}

Source:
\link{https://www.elastic.co/products/kibana}

\slide{Focus for the near future}

\begin{list2}
\item Walk through your infrastructure\\
get a detailed view of data, flows, protocols, bandwidth, ports and services

\item Create a list of critical phone numbers and contacts, enter it in your phone
\item Automate updates for both clients and servers, goal update everything in hours
\item Learn to run Nmap and Metasploit scripts - identify vulnerable servers
\end{list2}

\vskip 2cm
\centerline{consider the fact we have multiple overlapping critical security incidents now!}

\vskip 2cm
How many incidents can your organisation handle in parallel?


\myquestionspage

\end{document}
