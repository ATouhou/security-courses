\documentclass[20pt,landscape,a4paper,footrule]{foils}
\usepackage{solido-network-slides}
\usepackage{pdf14}
%\usepackage{ulem}

% Basic things that we need are below
\selectlanguage{danish}

%\externaldocument{unix-audit-security-oevelser}
\externaldocument{\jobname-exercises}


%\slide{Pause}
%Er det tid til en lille pause?
%\hlkimage{15cm}{300px-Fozziecurtain.JPG}

% Der fokuseres p� hacking/kompromittering i statsskala. 
% F�rst gennemg�s teknik og baggrund, inden der demonstreres sm� hands on-eksempler. 
% Stikord er: kryptering, bagd�re, hackerangreb og 0-days. S� kom og f� praktiske eksempler p�, hvordan du bedre beskytter dig og v�rkt�jer til at teste IT-sikkerheden hos dig selv. 

\begin{document}





\mytitlepage{Ins@fe Training Seminar}
{Secure browsing, plugins and privacy tools}
\vskip 1 cm

Key words: multiple browsers, crypto, Torproject and how to protect yourself


\slide{Democracy now: Why do we bother?}

\hlkimage{22cm}{jacob-appelbaum.png}
{\large
In a democracy we need the citizens with freedom that can act without constant surveillance

Democracy requires that we can actively select which personal data to give up and to whom
}


\vskip 2 cm
\centerline{\large Cryptography is peaceful protest against blanket surveillance}

\slide{Why think of security?}

\hlkimage{8cm}{1984-not-instruction-manual.jpg}


\begin{quote}
	Privacy is necessary for an open society in the electronic age. Privacy is not secrecy. A private matter is something one doesn't want the whole world to know, but a secret matter is something one doesn't want anybody to know. Privacy is the power to selectively reveal oneself to the world. ~A Cypherpunk's Manifesto by Eric Hughes, 1993
\end{quote}

Copied from \link{https://cryptoparty.org/wiki/CryptoParty}


\slide{Security is not magic}

%\vskip 2cm

\hlkimage{4cm}{wizard_in_blue_hat.png}

\hlkrightimage{5cm}{003scawebgoshindomanicon.png}
.
%{\large Superheltegerninger}

\begin{list1}
\item Think security, it may seem like magic - but it is not
\item Follow news about security
\item Support communities, join and learn
\end{list1}


\slide{Hacking is not magic}

\hlkimage{17cm}{ninjas.png}

\vskip 1 cm
\centerline{Hacking requires some ninja training}

\slide{Dive into the bitstream}

\hlkimage{19cm}{wireshark-http.png}

\centerline{Anyone in the network path can sniff this!}


\slide{Wall of sheep}

\hlkimage{16cm}{wall-of-sheep-defcon.jpg}

\begin{center}
Defcon Wall of Sheep - play nice!\\
\end{center}

What is YOUR password?

\slide{Hacker tools BackTrack and Kali}

\hlkimage{16cm}{kali-linux.png}

\begin{list1}
\item Hacking is fun and you can learn a lot
\item Remember to do it in a controlled environment lab setups
\item I recommend using Kali in a virtual environment, VMware Player, Virtualbox or similar
\item Kali Linux \link{http://www.kali.org/}
\item Then you can run other insecure virtual machines like Metasploitable for learning
\end{list1}


\slide{First advice use the modern operating systems}

\begin{list1}
\item Newer versions of Microsoft Windows, Mac OS X and Linux
\begin{list2}
\item Buffer overflow protection
\item Stack protection, non-executable stack
\item Heap protection, non-executable heap
\item \emph{Randomization of parameters} stack gap m.v.
\end{list2}
\item Note: these still have errors and bugs, but are better than older versions
%\item OpenBSD has shown the way in many cases\\ \link{http://www.openbsd.org/papers/}
\end{list1}

\vskip 1cm

\centerline{Always try to make life worse and more costly for attackers}


\slide{Multiple browsers}

\hlkimage{20cm}{multi-browser-strategy.png}

\begin{list2}
\item Multiple browsers: one for facebook, one for net banking applications
\item More strict secure settings and NoScripts for web surfing
\item One browser with loose settings for: Netflix, and \emph{trusted sites}
\item Remember to install critical plugins like HTTPS Everywhere, NoScript, CertPatrol m.fl.
\end{list2}

\slide{HTTPS Everywhere}

\hlkimage{5cm}{HTTPS_Everywhere_new_logo.jpg}
\begin{quote}
HTTPS Everywhere is a Firefox extension produced as a collaboration between The Tor Project and the Electronic Frontier Foundation. It encrypts your communications with a number of major websites.
\end{quote}

\centerline{\link{http://www.eff.org/https-everywhere}}


Also in Chrome web store!

\slide{NoScript and NotScripts}

\hlkimage{15cm}{noscript.png}

\begin{quote}
NotScripts\\
A clever extension that provides a high degree of 'NoScript' like control of javascript, iframes, and plugins on Google Chrome.
\end{quote}

NoScripts for Firefox or NotScripts for Chrome\\
Only allow scripting and active content on pages where it is required

Pro tip: you can avoid lots of advertisements

\slide{CertPatrol - which site uses which certificate}


\hlkimage{13cm}{certificate-patrol-400x300.jpg}

\begin{quote}
An add-on formerly considered paranoid: CertPatrol implements ''pinning'' for Firefox/Mozilla/SeaMonkey roughly as now recommended in the User Interface Guidelines of the World Wide Web Consortium (W3C).
\end{quote}

\centerline{\link{http://patrol.psyced.org/}}


\slide{Email is insecure}

\hlkimage{20cm}{email-uden-krypterin.png}

\centerline{Email without encryption is like an open post card}



\slide{Email with encryption - sending}

\hlkimage{18cm}{email-med-kryptering.png}


\centerline{Sending a secure email is not hard}

\slide{Encrypted in transit}

\hlkimage{11cm}{modtaget-email-med-kryptering.png}

\centerline{A secure email is protected while being transported}


\slide{Kryptering: Cryptography Engineering}

\hlkimage{8cm}{book-ce-150w.jpg}

\emph{Cryptography Engineering} by
Niels Ferguson, Bruce Schneier, and Tadayoshi Kohno
\link{https://www.schneier.com/book-ce.html}
\vskip 8mm
\centerline{Cryptography ensures confidentiality and integrity of messages}


\slide{FileZilla - underst�tter SFTP}

\hlkimage{22cm}{filezilla.png}
\centerline{\link{http://filezilla-project.org/}}

\vskip 8mm
\centerline{Stop using unencrypted FTP, use SSL/TLS or SFTP}

\slide{Tor project anonymized web browsing and services}

\hlkimage{16cm}{sample-tor-site.png}

\centerline{Sample site Secure Drop whistleblower site}


\centerline{There are alternatives, but Tor is well-known}
\link{https://www.torproject.org/}

\slide{Tor project - how it works 1}

\hlkimage{21cm}{how-tor-works-1.png}

\centerline{pictures from \link{https://www.torproject.org/about/overview.html.en}}

\slide{Tor project - how it works 2}

\hlkimage{21cm}{how-tor-works-2.png}

\centerline{pictures from \link{https://www.torproject.org/about/overview.html.en}}

\slide{Tor project - how it works 3}

\hlkimage{21cm}{how-tor-works-3.png}

\centerline{pictures from \link{https://www.torproject.org/about/overview.html.en}}

\slide{Tor project install}

\hlkimage{18cm}{tor-project.png}

Multiple Tor tools, like Torbutton on/off for Firefox 

We recommend starting out with the Torbrowser bundles from\\ \link{https://www.torproject.org/}

\slide{Torbrowser - outdated}

\hlkimage{20cm}{torbrowser-outdated.png}

\centerline{\color{red}Missing an update!}

\slide{Torbrowser - anonym browser}

\hlkimage{20cm}{torbrowser-main-window.png}

\centerline{\color{titlecolor} More anonymized browsing - uses Firefox in disguise}


\slide{Torbrowser - sample site}

\hlkimage{18cm}{sample-tor-site.png}

\centerline{\color{titlecolor} .onion er Tor adresser - hidden sites}

\footnotesize{Example SecureDrop from Radio24syv, danish media\\ \link{http://www.radio24syv.dk/dig-og-radio24syv/securedrop/}}



\slide{Whonix - Tor to the max!}

\hlkimage{17cm}{400px-Whonix.jpg}

\begin{quote}
Whonix is an operating system focused on anonymity, privacy and security. It's based on the Tor anonymity network[5], Debian GNU/Linux[6] and security by isolation. DNS leaks are impossible, and not even malware with root privileges can find out the user's real IP. \link{https://www.whonix.org/}

\end{quote}

\centerline{Torbrowser is nice, but limited, Whonix Tor-encrypts all traffic}



\slide{Secure your mobile}

\hlkimage{20cm}{the-guardian-project.pdf}

\centerline{Don't forget your mobile platforms \link{https://guardianproject.info/}}




\slide{VPN a tunnel for all your traffic}

\hlkimage{12cm}{openvpn-gui-systray.png}

\begin{list1}
\item Virtual Private Networks are useful - or even required when traveling
\item VPN \link{http://en.wikipedia.org/wiki/Virtual_private_network}
\item SSL/TLS VPN - Multiple incompatible vendors: OpenVPN, Cisco, Juniper, F5 Big IP
\end{list1}


\slide{DNS censorship in Denmark}

\hlkimage{6cm}{Censored_rubber_stamp.png}

There is a lot of DNS tampering today, you can change to:
Censurfridns.dk UncensoredDNS
\begin{list2}
\item ns1.censurfridns.dk / 89.233.43.71 / 2002:d596:2a92:1:71:53::
\item ns2.censurfridns.dk / 89.104.194.142 / 2002:5968:c28e::53
\end{list2}
also see the web site and blog for more information:\\
\link{http://www.censurfridns.dk} and \link{blog.censurfridns.dk} 

\vskip 1cm

\centerline{\large DNS tampering is unacceptable}

\slide{DNSSEC trigger}

\hlkimage{7cm}{dnssec-trigger.png}

Using DNSSEC and DANE will help

I use DNSSEC-trigger secure local DNS server for your Windows or Mac laptop.

\begin{list2}
\item DNSSEC Validator for Firefox\\ \link{https://addons.mozilla.org/en-us/firefox/addon/dnssec-validator/}
\item OARC tools \link{https://www.dns-oarc.net/oarc/services/odvr}
\item \link{http://www.nlnetlabs.nl/projects/dnssec-trigger/}
\end{list2}


\slide{Follow Twitter Safety news}

%\hlkimage{18cm}{twitter-security-feed.png}
\hlkimage{10cm}{twitter-safety.png}

\begin{list1}
\item Twitter has become an important new resource for lots of stuff
\item Twitter has replaced RSS for me 
\item The Twitter team actively promotes good security practives
\end{list1}


\slide{Checklist}

\begin{list2}
\item BIOS kodeord, pin for your phone 
\item Firewall - especially on laptops
\item Install anti virus and anti-spyware if using Windows
\item Use multiple browsers with different security settings
\item Use OpenPGP and email encryption
\item Use Password Safe, Keychain Access (OSX) or other password saving programs
\item Consider using hard disk or file encryption like Truecrypt \link{http://www.truecrypt.org/}
\item Keep systems and applications updated with security fixes
\item {\bf Backup your data} - perhaps the single most important point
\item Secure wipe your devices when you stop using them
\end{list2}

\slide{Work together}

\hlkimage{14cm}{Shaking-hands_web.jpg}

\begin{list1}
\item Team up!
\item We need to share security information freely
\item We often face the same threats, so we can work on solving these together
\end{list1}


\slide{Be careful - questions?}

\hlkimage{5cm}{michael-conrad.jpg}
\centerline{\Large Hey, Lets be careful out there!}
\vskip 2 cm

\begin{center}
\myname

%\myweb
\end{center}

\vskip 2cm
Source: Michael Conrad \link{http://www.hillstreetblues.tv/}


\end{document}

